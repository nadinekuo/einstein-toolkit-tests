\documentclass{article}
%\usepackage{../../../../doc/ThornGuide/cactus}
\usepackage{../../../../../doc/latex/cactus}
\newlength{\tableWidth} \newlength{\maxVarWidth} \newlength{\paraWidth} \newlength{\descWidth} \begin{document}

% The title of the document (not necessarily the name of the Thorn)
\title{The {\tt GRHydro} code: three-dimensional relativistic hydrodynamics}

% The author of the documentation - on one line, otherwise it does not work
\author{Original authors: Luca Baiotti, Ian Hawke, Pedro Montero \cr Contributors: 
  Sebastiano Bernuzzi, Giovanni Corvino, Toni Font, Joachim Frieben, \cr Roberto De Pietri, Thorsten
  Kellermann, Frank L\"offler, Christian D. Ott, \cr Luciano Rezzolla, Nikolaos Stergioulas, Aaryn
  Tonita, \cr {\it and many others,} \cr {\it especially those who were in the European Network ``Sources of
    Gravitational Waves'' } }

% the date your document was last changed, if your document is in CVS, 
% please use:
\date{Date: 2009-12-07 19:20:47 -0600 (Mon, 07 Dec 2009)}
\maketitle

% *======================================================================*
%  Cactus Thorn template for ThornGuide documentation
%  Author: Ian Kelley
%  Date: Sun Jun 02, 2002
%  $Header$                                                             
%
%  Thorn documentation in the latex file doc/documentation.tex 
%  will be included in ThornGuides built with the Cactus make system.
%  The scripts employed by the make system automatically include 
%  pages about variables, parameters and scheduling parsed from the 
%  relevent thorn CCL files.
%  
%  This template contains guidelines which help to assure that your     
%  documentation will be correctly added to ThornGuides. More 
%  information is available in the Cactus UsersGuide.
%                                                    
%  Guidelines:
%   - Do not change anything before the line
%       % START CACTUS THORNGUIDE",
%     except for filling in the title, author, date etc. fields.
%        - Each of these fields should only be on ONE line.
%        - Author names should be sparated with a \\ or a comma
%   - You can define your own macros are OK, but they must appear after
%     the START CACTUS THORNGUIDE line, and do not redefine standard 
%     latex commands.
%   - To avoid name clashes with other thorns, 'labels', 'citations', 
%     'references', and 'image' names should conform to the following 
%     convention:          
%       ARRANGEMENT_THORN_LABEL
%     For example, an image wave.eps in the arrangement CactusWave and 
%     thorn WaveToyC should be renamed to CactusWave_WaveToyC_wave.eps
%   - Graphics should only be included using the graphix package. 
%     More specifically, with the "includegraphics" command. Do
%     not specify any graphic file extensions in your .tex file. This 
%     will allow us (later) to create a PDF version of the ThornGuide
%     via pdflatex. |
%   - References should be included with the latex "bibitem" command. 
%   - use \begin{abstract}...\end{abstract} instead of \abstract{...}
%   - For the benefit of our Perl scripts, and for future extensions, 
%     please use simple latex.     
%
% *======================================================================* 
% 
% Example of including a graphic image:
%    \begin{figure}[ht]
%       \begin{center}
%          \includegraphics[width=6cm]{/home/runner/work/einstein-toolkit-tests/einstein-toolkit-tests/arrangements/EinsteinEvolve/GRHydro/doc/MyArrangement_MyThorn_MyFigure}
%       \end{center}
%       \caption{Illustration of this and that}
%       \label{MyArrangement_MyThorn_MyLabel}
%    \end{figure}
%
% Example of using a label:
%   \label{MyArrangement_MyThorn_MyLabel}
%
% Example of a citation:
%    \cite{MyArrangement_MyThorn_Author99}
%
% Example of including a reference
%   \bibitem{MyArrangement_MyThorn_Author99}
%   {J. Author, {\em The Title of the Book, Journal, or periodical}, 1 (1999), 
%   1--16. {\tt http://www.nowhere.com/}}
%
% *======================================================================* 

% If you are using CVS use this line to give version information
% $Header$

% Use the Cactus ThornGuide style file
% (Automatically used from Cactus distribution, if you have a 
%  thorn without the Cactus Flesh download this from the Cactus
%  homepage at www.cactuscode.org)

% Do not delete next line
% START CACTUS THORNGUIDE

% Add all definitions used in this documentation here 
%   \def\mydef etc

%\newcommand{\eqref}[1]{(\ref{#1})}

% Add an abstract for this thorn's documentation
\begin{abstract}
  {\tt GRHydro} is a fully general-relativistic three-dimensional hydrodynamics code. It evolves the
  hydrodynamics using High Resolution Shock Capturing methods and can
  work with realistic equations of state. The evolution of the
  spacetime can be done by any other ``appropriate'' thorn, such as
  the {\tt CCATIE} code, maintained and developed at the Albert-Einstein-Institut (Potsdam).
\end{abstract}

% The following sections are suggestive only.
% Remove them or add your own.

\section{Introduction}
\label{sec:intro}

The {\tt GRHydro}\footnote{GRHydro started as a branch of Whisky. 
  The name Whisky is
  due to Tom Goodale, who pointed out that the in the original Gaelic
  {\it uisge beatha} (from which {\tt Whisky} is derived) meant {\it water of
  life}. This name was chosen at a free and fair democratic ballot
  at the EU Network meeting in Southampton that gave the right answer
  thanks to a little bit of help from the authors of the code after
  some slight irregularities on the part of certain specialists in
  strange stars.} code is based upon the public version of the {\tt Whisky} code which
  is a product of the EU Network on Sources of
  Gravitational Radiation\footnote{http://www.eu-network.org},
  and was initially written by Luca Baiotti, Ian Hawke
  and Pedro Montero, based on the publicly available {\tt GR3D} code and
  with many other important contributors.
  With the help of large parts of the community, the {\tt GRHydro} code got
  improved, extended and included into the Einstein Toolkit.

\section{Using This Thorn}
\label{sec:use}

What follows is a brief introduction to using {\tt GRHydro}. It assumes that
you know the required physics and numerical methods, and also the
basics of Cactus\footnote{http://www.cactuscode.org}. If you don't, then skip this section and come back
to it after reading the rest of this ThornGuide of Cactus. For more details such
as thornlists and parameter files, take a look at the Einstein Toolkit web page
which is currently stored at
\begin{verbatim}
http://www.einsteintoolkit.org
\end{verbatim}

{\tt GRHydro} uses the hydro variables defined in {\tt HydroBase}
and provides own ``conserved'' hydro variables and methods to evolve them. It
does not provide any information about initial data or equations of
state. For these, other thorns are required. A minimal list of thorns
for performing a shock-tube test is given in the shock-tube test
parameter file, found at
\begin{verbatim}
GRHydro/test/GRHydro_test_shock.par
\end{verbatim}
and will include the essential thorns
\begin{verbatim}
GRHydro EOS_Omni ADMBase ADMCoupling MoL
\end{verbatim}
Initial data for shocks can be set using
\begin{verbatim}
GRHydro_Init_Data 
\end{verbatim}
Initial data for spherically symmetric static stars (with
perturbations or multiple ``glued'' stars) can be set by
\begin{verbatim}
TOVSolver
\end{verbatim}

The actual evolution in time is controlled by the Method of Lines
thorn MoL. For full details see the relevant ThornGuide. For the
purposes of {\tt GRHydro} at least two parameters are relevant; {\tt
  ode\_method} and {\tt mol\_timesteps}. If second-order accuracy is
all that is required then {\tt ode\_method} can be set to either {\tt
  "rk2"} (second-order TVD Runge-Kutta evolution) or {\tt "icn"}
(Iterative Crank Nicholson, number of iterations controlled by {\tt
  mol\_timesteps}, defaults to three), and {\tt mol\_timesteps} can be
ignored. A more generic (and hence less efficient) method can be
chosen by setting {\tt ode\_method} to {\tt "genrk"} which is a
Shu-Osher type TVD Runge-Kutta evolution. Then the parameter {\tt
  mol\_timesteps} controls the number of intermediate steps and hence
the order of accuracy. First to seventh order are currently supported.

{\tt GRHydro} currently uses a Reconstruction-Evolution type method. The type
of reconstruction is controlled by the parameter {\tt recon\_method}.
The currently supported values are {\tt "tvd"} for slope limited TVD
reconstruction, {\tt "ppm"} for the Colella-Woodward PPM method, and
{\tt "eno"} for the Essentially Non-Oscillatory method of Harten et
al. Each of these has further controlling parameters. For example
there are a number of slope limiters controlled by the keyword {\tt
  tvd\_limiter}, the PPM method supports shock detection by the
Boolean {\tt ppm\_detect}, and the ENO method can have various orders
of accuracy controlled by {\tt eno\_order}. Note that the higher-order
methods such as PPM and ENO require the stencil size to be increased
using {\tt GRHydro\_stencil} {\bf and} {\tt driver::ghost\_size}.

For the evolution various approximate Riemann solvers are available,
controlled by {\tt riemann\_solver}. Currently implemented are {\tt
  "HLLE"}, {\tt "Roe"} and {\tt "Marquina"}. For the Roe and Marquina
methods there are added options to choose which method is used for
calculating the left eigenvectors. This now defaults to the efficient
methods of the Valencia group, but the explicit matrix inversion is
still there for reference.

For the equations of state, two ``types'' are recognized, controlled
by the parameter {\tt GRHydro\_eos\_type}. These are {\tt "Polytype"}
where the pressure is a function of the rest-mass density, $P=P(\rho)$, and the
more generic {\tt "General"} type where the pressure is a function
of the rest-mass density and the specific internal energy, $P=P(\rho, \epsilon)$. For the
{\tt Polytype} equations of state one fewer equation is evolved and
the specific internal energy is set directly from the rest-mass density. The
actual equation of state used is controlled by the parameter {\tt
  GRHydro\_eos\_table}. Polytype equations of state include {\tt
  "2D\_Polytrope"} and general equations of state include {\tt
  "Ideal\_Fluid"}. 

\subsection{Obtaining This Thorn}

The public version of GRHydro can be found on the
website {\tt http://www.einsteintoolkit.org}. 

\subsection{Basic Usage}

The simplest way to start using {\tt GRHydro} would be to download some
example parameter files from the web page to try it. There are a number
of essential parameters which might reward some experimentation. These
include:
\begin{itemize}
\item Reconstruction methods:
  \begin{itemize}
  \item {\tt recon\_method}: The type of reconstruction method to
    use. {\tt tvd} is the standard. {\tt ppm} is more accurate but it requires 
    more resources. {\tt eno} gives in theory
    arbitrary order of accuracy but it is practically unworthy to go beyond fifth order.
  \item {\tt tvd\_limiter}: When using {\tt tvd} reconstruction the
    choice of limiter can have a large effect. {\tt vanleerMC2} is
    probably the best to use, but the extremes of {\tt minmod} and {\tt
      Superbee} are also interesting.
  \item {\tt ppm\_detect}: When using {\tt ppm} reconstruction with
    shocks, the shock detection algorithm can notably sharpen the
    profile.
  \end{itemize}
\item Riemann solvers: {\tt Marquina} is the standard solver
  used. {\tt HLLE} is significantly faster, but sometimes provides cruder approximation.
\item Equations of state: These are controlled by the {\tt
    GRHydro\_eos\_type} and {\tt GRHydro\_eos\_table} parameters. Changing
  these parameters will depend on which equation of state thorns you
  have compiled in.
\item Initial data parameters: {\tt GRHydro\_rho\_central} is inherited by many
  initial data thorns to set the central density of compact fluid
  objects such as single stars. However, this parameter is depreciated.
\item Atmosphere parameters: Many of these are listed in
  section~\ref{sec:atmosphere}. 
\end{itemize}

\subsection{Special Behaviour}

Although in theory {\tt GRHydro} can deal with conformal metrics as well as
physical metrics, this part of the code is completely untested as we
don't have conformal initial data (although this would not be hard -
we just haven't had the incentive).

\subsection{Interaction With Other Thorns}

{\tt GRHydro} provides the appropriate contribution to the stress energy
through the {\tt TmunuBase} interface. Those spacetime evolvers that
use this interface can use {\tt GRHydro} without change. 
%To pass the required variables across {\tt GRHydro} is a friend of {\tt ADMCoupling}.

{\tt GRHydro} uses the {\tt MoL} thorn to perform the actual time
evolution. This means that if all other evolution thorns are also
using {\tt MoL} then the complete evolution will have the accuracy of
the {\tt MoL} evolution method without change. This (currently) allows
for up to fourth-order accuracy in time without any changes to {\tt GRHydro}.

For the general equations of state {\tt GRHydro} uses the {\tt EOS\_Omni}
interface. This returns the necessary hydrodynamical quantities, such
as the pressure and derivatives with general function calls. The
parameter {\tt GRHydro\_eos\_table} controls which equation of state is
used during evolution.

For the metric quantities {\tt GRHydro} uses the standard {\tt
  CactusEinstein} arrangement, especially {\tt ADMBase}. This allows
the standard thorns to be used for the calculation of constraint
violations, emission of gravitational waves, location of the apparent
horizon, and more.

\subsection{Support and Feedback}

{\tt GRHydro} is part of the Einstein Toolkit, with its web page located at
\begin{verbatim}
http://www.einsteintoolkit.org
\end{verbatim}
It contains information on obtaining the code, together with
thornlists and sample parameter files. For questions, send an email
to the Einstein Toolkit mailing list {\tt users@einsteintoolkit.org}.

\section{Physical System}
\label{sec:phys}

The equations of general relativistic hydrodynamics can be written in
the flux conservative form
\begin{equation}
  \label{eq:consform1}
  \partial_t {\bf q} + \partial_{x^i} {\bf f}^{(i)} ({\bf q}) = {\bf
    s} ({\bf q}),
\end{equation}
where ${\bf q}$ is a set of {\it conserved variables}, ${\bf f}^{(i)}
({\bf q})$ the fluxes and ${\bf s} ({\bf q})$ the source
terms.
The 8 conserved variables are labeled $D$, $S_i$, $\tau$, and $\mathcal B^k$,
where
$D$ is the generalized particle number density, $S_i$ are the generalized
momenta in each direction, $\tau$ is an internal energy term, and $\mathcal
B^k$ is the densitized magnetic field.
These conserved variables are composed from a set of {\it primitive variables},
which are $\rho$, the rest-mass density, $p$, the
pressure, $v^i$, the fluid 3-velocities, $\epsilon$, the specific internal
energy, $B^k$ the magnetic field in the lab frame,
and $W$, the Lorentz factor, via the following relations
% from GRHydro/src/Prim2con.F90
%  w = 1.d0 / sqrt(1.d0 - (gxx*dvelx*dvelx + gyy*dvely*dvely + gzz &
%       &*dvelz*dvelz + 2*gxy*dvelx*dvely + 2*gxz*dvelx *dvelz + 2*gyz&
%       &*dvely*dvelz))
% 
%  dpress = (GRHydro_eos_gamma - 1.d0) * drho * deps
%
%  ddens = sqrt(det) * drho * w
%  dsx = sqrt(det) * (drho*(1+deps)+dpress)*w*w * (gxx*dvelx + gxy&
%       &*dvely + gxz*dvelz)
%  dsy = sqrt(det) * (drho*(1+deps)+dpress)*w*w * (gxy*dvelx + gyy&
%       &*dvely + gyz*dvelz)
%  dsz = sqrt(det) * (drho*(1+deps)+dpress)*w*w * (gxz*dvelx + gyz&
%       &*dvely + gzz*dvelz)
%  dtau = sqrt(det) * ((drho*(1+deps)+dpress)*w*w - dpress) - ddens
\
\begin{eqnarray}
  \label{eq:prim2con}
   D &=& \sqrt{\gamma}W\rho \nonumber \\
   S_i &=& \sqrt{\gamma} \left(\rho h^* W^{\,2} v_j-\alpha b^0b_j\right) \nonumber \\
   \tau &=& \sqrt{\gamma}\left(\rho h^* W^2 - p^*-(\alpha b^0)^2\right) - D, \nonumber \\
   \mathcal B^k &=& \sqrt{\gamma} B^k,
\end{eqnarray}
where $\gamma$ is the determinant of the spatial 3-metric $\gamma_{ij}$ and 
$h^* \equiv 1+\epsilon+\left(p + b^2\right)/\rho$, $p^* = p + b^2/2$.
$b^\mu$ is the magnetic field in the fluid's rest frame $b^\mu = u_\nu
^*F^{\mu\nu}$ where $^*F^{\mu\nu}$ is the dual of the Faraday tensor. It is
related to $B^k$ via
\begin{eqnarray}
b^0&=&\frac{WB^kv_k}{\alpha}\,\,,\\
b^i&=&\frac{B^i}{W}+W(B^kv_k)\left(v^i-\frac{\beta^i}{\alpha}\right)\,\,,\\
b^2&=&\frac{B^iB_i}{W^2}+(B^iv_i)^2\,\,.
\end{eqnarray}

Only five of the primitive fluid variables are
independent. The Lorentz factor is defined in terms of the
velocities and the metric as $W = (1-\gamma_{ij}v^i v^j)^{-1/2}$.  
Also one of the pressure, rest-mass density or specific internal energy terms is given in 
terms of the other two by an {\it equation of state}.

The fluxes are usually defined in terms of both the conserved
variables and the primitive variables:
%
\begin{eqnarray}
{\bf F}^i({\bf U}) &=& [D(\alpha v^i - \beta^i), S_j(\alpha v^i -
  \beta^i) + p\delta^i_j, \tau(\alpha v^i - \beta^i) + p
  v^i, \mathcal B^k (\alpha v^i - \beta^i) - \mathcal B^i (\alpha v^k - \beta^k)]^T\ .
\end{eqnarray}
%
The source terms are
%
\begin{eqnarray} \label{source_terms}
{\bf s}({\bf U}) = \Big [0, T^{\mu\nu}\big (\partial_\mu g_{\nu j} +
  \Gamma^\delta_{\mu\nu} g_{\delta j}\big ), \alpha \big (T^{\mu
    0}\partial_\mu \ln \alpha - T^{\mu\nu}\Gamma^0_{\nu\mu} \big), 0 \Big ]\ .
\end{eqnarray}
%
Note that the source terms do not contain differential operators
acting on the stress-energy tensor and that this is important for the
numerical preservation of the hyperbolicity character of the system.
Also note that in a curved spacetime the equations are not in a
strictly-homogeneous conservative form, which is recovered only in flat
spacetime. This conservative form of the relativistic Euler equations
was first derived by the group at the Universidad de Valencia
\cite{Banyuls97} and therefore was named the {\it Valencia
formulation}. 

The stress energy tensor is in turn given by
\begin{eqnarray}
T^{\mu \nu}
  &=&  \left( \rho +  \rho \epsilon + p + b^2 \right) u^\mu u^\nu +
       \left(p + \frac{b^2}{2} \right) g^{\mu \nu} - b^\mu b^\nu
  \\\nonumber
  &\equiv& \rho h^*u^\mu u^\nu + P^* g^{\mu \nu} -
           b^\mu b^\nu.  \label{mhd-stress-energy-tensor}
\end{eqnarray}

%Describe possible equations of state.

%Any other physics points (there'll be lots).

For more detail see the review of Font~\cite{livrevgrrfd} and the GRHydro
paper~\cite{Moesta:2013dna}.

\section{Numerical Implementation}

TODO: describe MHD scheme in particular constrained transport and con2prim
method used.

\section{High Resolution Shock Capturing methods}
\label{sec:hrsc}

The numerical evolution of a hydrodynamical problem is complicated by
the occurrence of {\it shocks}, {\it i.e.} genuine nonlinear discontinuities that
will generically form. It is also complicated by the conservation
constraint. In a High Resolution Shock Capturing (HRSC) method the
requirement of conservation is used to ensure the correct evolution of
a shock. A HRSC method also avoids spurious oscillations at shocks
which are known as Gibbs' phenomena, while retaining a high order of
accuracy over the majority of the domain.

For a full introduction to HRSC methods see~\cite{laney}, \cite{toro},
\cite{leveque}, \cite{livrevsrrfd} and \cite{livrevgrrfd}.

In the {\tt GRHydro} code it was decided to use the {\it method of lines} as a
base for the HRSC scheme. The method of lines is a way of turning a
partial differential equation such as~(\ref{eq:consform1}) into an
ordinary differential equation. For the {\tt GRHydro} code the following steps
are required.
\begin{itemize}
\item Partition the domain of interest into {\it cells}. For
  simplicity we shall assume a regular Cartesian partitioning. This is
  not necessary for the method of lines, but it is for {\tt GRHydro}.
\item Over a given cell with Cartesian coordinates $(x^1_i, x^2_j, x^3_k)$,
  integrate equation~(\ref{eq:consform1}) in space to find the
  ordinary differential equation
  \begin{eqnarray}
    \label{eq:molform1} \nonumber
    \frac{{\rm d} \,}{{\rm d} t} {\bf q} & = & \int \!\!\!\! \int \!\!\!\!
      \int {\bf s} \,{\rm d}^3 x + \int_{x^2_{j-1/2}}^{x^2_{j+1/2}}
    \int_{x^3_{k-1/2}}^{x^3_{k+1/2}} {\bf f}^{(1)} ({\bf q}
    (x^1_{i-1/2}, y, z)) {\rm d} y \, {\rm d} z - \\
    &&    \int_{x^2_{j-1/2}}^{x^2_{j+1/2}} 
    \int_{x^3_{k-1/2}}^{x^3_{k+1/2}} {\bf f}^{(1)} ({\bf q}
    (x^1_{i+1/2}, y, z))
    {\rm d} y \, {\rm d} z + \cdots
  \end{eqnarray}
  where the boundaries of the Cartesian cells are given by $x^1_{i \pm
  1/2}$ and so on.
\item If we define $\bar{\bf q}$ as the integral average of
  ${\bf q}$ over the cell, after dividing (\ref{eq:molform1}) by the volume of the cell, we get an ordinary
  differential equation for $\bar{\bf q}$, in terms of the flux functions and the
  source terms as functions of the spatial differential of $\bar{{\bf
  q}}$. We note that, unlike the spatial differential of ${\bf q}$,
  the spatial differential of $\bar{{\bf q}}$ is well defined in a
  cell containing a discontinuity. 
\end{itemize}

This ordinary differential equation can be solved by the Cactus thorn
MoL. All that {\tt GRHydro} has to do is to return the values of the discrete
spatial differential operator
\begin{eqnarray}
  \label{eq:molrhs1} \nonumber
  {\bf L}({\bf q}) & = & \int \!\!\!\! \int \!\!\!\!
      \int {\bf s} \,{\rm d}^3 x + \int_{x^2_{j-1/2}}^{x^2_{j+1/2}}
    \int_{x^3_{k-1/2}}^{x^3_{k+1/2}} {\bf f}^{(1)} ({\bf q}
    (x^1_{i-1/2}, y, z)) {\rm d} y \, {\rm d} z - \\
    &&    \int_{x^2_{j-1/2}}^{x^2_{j+1/2}} 
    \int_{x^3_{k-1/2}}^{x^3_{k+1/2}} {\bf f}^{(1)} ({\bf q}
    (x^1_{i+1/2}, y, z)) {\rm d} y \, {\rm d} z + \cdots
\end{eqnarray}
given the data ${\bf q}$, and to supply the boundary conditions that will
be required to calculate this right hand side at the next time level.
We note that in the current implementation of MoL the solution to the
ordinary differential equation~(\ref{eq:molform1}) will be $N^{\rm
  th}$-order accurate provided that the time integrator used by MoL is $N^{\rm
  th}$-order accurate in time, and that the discrete operator ${\bf L}$ is
$N^{\rm th}$-order accurate in space and {\it first}-order (or better)
accurate in time.  For more details on the method of lines, and the
options given with the time integration for MoL, see the relevant
chapter in the ThornGuide.

In this implementation of {\tt GRHydro} the right hand side operator ${\bf L}$
will be simplified considerably by approximating the integrals by the
midpoint rule to get
\begin{equation}
  \label{eq:molrhs2}
  {\bf L}({\bf q}) = {\bf s}_{i,j,k} + {\bf f}^{(1)}_{i-1/2,j,k} -
    {\bf f}^{(1)}_{i+1/2,j,k} + \cdots
\end{equation}
where the dependence of the flux on ${\bf q}$ and spatial position is
implicit in the notation. Given this simplification, the calculation
of the right hand side operator splits simply into the following two parts:
\begin{enumerate}
\item Calculate the source terms ${\bf s}({\bf q}(x^1_i, x^2_j,
  x^3_k))$:

  Given that $\bar{{\bf q}}$ is a second-order accurate approximation
  to ${\bf q}$ at the midpoint of the cell, which is precisely $(x^1_i, x^2_j,
  x^3_k)$, for second-order accuracy it is sufficient to use ${\bf
  s}(\bar{{\bf q}}_{i,j,k})$.
\item In each coordinate direction $x^a$, calculate the {\it intercell
    flux} ${\bf f}^{(a)}({\bf q}_{i+1/2,j,k})$:
  
  From the initial data $\bar{{\bf q}}$ given at time $t^n$ we can
  reconstruct the data at the cell boundary, $({\bf q}_{i+1/2,j,k})$
  to any required accuracy in space (except in the vicinity of a
  shock, where only first-order accuracy is guaranteed).
% - FIXME: check,
%  Ian thinks the ENO theorems say that for linear problems get high order
%  even with discty). 
  However this will only be zeroth-order accurate
  in time. To ensure first-order accuracy in time, we have to find
  $({\bf q}_{i+1/2,j,k})(t)$ while retaining the high spatial order
  of accuracy. This requires two steps:
  \begin{enumerate}
  \item {\it Reconstruct} the data ${\bf q}$ over the cells adjacent
    to the required cell boundary. This reconstruction should use the
    high spatial order of accuracy. This gives two values of
    $({\bf q}_{i+1/2,j,k})$, which we call ${\bf q}_L$ and ${\bf q}_R$,
    where ${\bf q}_L$ is obtained from cell $i$ (left cell) and ${\bf q}_R$
    from cell $i+1$ (right cell).
  \item The values ${\bf q}_{L,R}$ are used as initial data for the
    {\it Riemann problem}. This is the initial value problem given by
    the partial differential equation~(\ref{eq:consform1}) with
    semi-infinite piecewise constant initial data ${\bf q}_{L,R}$. As
    the true function ${\bf q}$ is probably not piecewise constant we
    will not get the exact solution of the general problem even if we
    solve the local Riemann problem exactly. However, it will be first-order 
    accurate in time and retain the high order of accuracy in
    space which, as explained in the documentation to the MoL thorn,
    is sufficient to ensure that the method as a whole has a high
    order of accuracy. 
  \end{enumerate}
\end{enumerate}

So, the difficult part of {\tt GRHydro} is expressed in two routines. One that
reconstructs the function ${\bf q}$ at the boundaries of a
computational cell given the cell average data $\bar{{\bf q}}$, and
another that calculates the intercell flux ${\bf f}$ at this cell
boundary. 

\section{Reconstruction}
\label{sec:recon}

In the reduction of all of {\tt GRHydro} to two routines in the last section
one point was glossed over. That is, in order for the numerical method
to be consistent and convergent it must retain conservation and not
introduce spurious oscillations. Up to this point all the steps have
either been exact or have neither changed the conservation properties
or the profile of the function. Also, the calculation of the intercell
flux from the Riemann problem can be made to be ``exactly correct''.
That is, even though as explained above it may not be the true flux
for the real function ${\bf q}$, it will be the exact physical
solution for the values ${\bf q}_{L,R}$ given by the reconstruction
routine, so the intercell flux cannot violate conservation or
introduce oscillations.  Unphysical effects such as these can only be
introduced by an incorrect reconstruction.

For a full explanation of reconstruction methods see
Laney~\cite{laney}, Toro~\cite{toro}, Leveque~\cite{leveque}. For the
moment we will concentrate on the simplest methods that are better
than first-order accurate in space, the TVD slope-limited methods.
More complex methods such as ENO will be introduced later.

In the late 1950's Godunov proved a theorem that, in this context,
says
\begin{quote}
Any {\bf linear} reconstruction method of higher-than-first-order
accuracy may introduce spurious oscillations.
\end{quote}
For this theorem {\it linear} meant that the reconstruction method was
independent of the data it was reconstructing. Simple centred
differencing is a linear second-order method and is oscillatory near a
shock. Instead we must find a reconstruction method that depends on
the data ${\bf q}$ being reconstructed.

Switching our attention to conservation, we note that there is
precisely one conservative first-order reconstruction method,
\begin{equation}
  \label{eq:reconfirst}
  {\bf q}^{{\rm First}}(x) = \bar{{\bf q}}_i, \qquad x \in [
  x_{i-1/2}, x_{i+1/2} ], 
\end{equation}
and that any second-order conservative method can be written in terms
of a {\it slope} or rather \emph{difference} $\Delta_i$ as
\begin{equation}
  \label{eq:reconsecond}
  {\bf q}^{{\rm Second}}(x) = \bar{{\bf q}}_i + \frac{x -
  x_i}{x_{i+1/2} - x_{i-1/2}} \Delta_i, \qquad x \in [ x_{i-1/2}, x_{i+1/2} ].
\end{equation}

\subsection{TVD Reconstruction}
\label{sec:tvd}

As we want a method that is accurate ({\it i.e.}, at least to second order)
while being stable ({\it i.e.}, only first order or nonlinear at shocks)
then the obvious thing to do is to use some second-order method in the
form of equation~(\ref{eq:reconsecond}) in smooth regions but which
changes to the form of equation~(\ref{eq:reconfirst}) smoothly near
shocks. 

In the articles describing the {\tt GRAstro\_Hydro}
code\footnote{http://wugrav.wustl.edu/ASC/internal/asccodes.html},
this was described as an average of the two reconstructions, 
\begin{equation}
  {\bf q}^{{\rm TVD}}(x) = \phi({\bf q}) {\bf q}^{{\rm Second}} + (1 -
  \phi({\bf q})) {\bf q}^{{\rm First}},
  \label{First_qTVD}
\end{equation}
where $\phi \in [0,1]$ varies smoothly in some sense, and is zero near
a shock and 1 in smooth regions. In Toro's notation~\cite{toro} (which we usually adopt here) 
the slope limiter function $\phi$ (having the same attributes as above)
directly multiplies the slope, giving
\begin{equation}
  {\bf q}^{{\rm TVD}}(x) = \bar{{\bf q}}_i + \frac{x -
  x_i}{x_{i+1/2} - x_{i-1/2}} \phi({\bf q}) \Delta_i, \qquad x \in [
  x_{i-1/2}, x_{i+1/2} ]. 
  \label{Toro_qTVD}
\end{equation}
Equations (\ref{First_qTVD}) and (\ref{Toro_qTVD}) are equivalent.

For details on how to construct a limiter, on their stability regions and on 
the explicit expressions for the limiters used here,
see~\cite{toro}. The {\tt GRHydro} code implements the {\tt minmod} limiter
(the most diffusive and the default), the Van Leer Monotonized Centred
(MC) ({\tt VanLeerMC}) limiter in a number of forms (which should give equivalent results), 
and the {\tt Superbee} limiter. The limiter specified by the parameter value {\tt VanLeerMC2} 
is the recommended one.

As an example, we show how TVD with the minmod limiter is implemented
in the code. First, we define the minmod function:
\begin{equation}
  \label{eq:tvdminmod}
  \mathrm{\mathbf{minmod}}(a,b) = \left\{ \begin{array}{c l} 
      \text{min}(|a|,|b|) & \text{if } (a b > 0)\\
      0 & \text{otherwise}. \end{array}\right.
\end{equation}
For reconstructing $\mathbf{q}$ %at the cell interfaces $i+\frac{1}{2}$ and $i-\frac{1}{2}$, 
we choose two differences
\begin{equation}
  \begin{array}{lcl}
    % tex4ht fails unless \mathbf is in a group
    \Delta_{\mathrm{upw}} & = & {\mathbf{q}}_{i} - {\mathbf{q}}_{i-1}\\
    \Delta_{\mathrm{loc}} & = & {\mathbf{q}}_{i+1} - {\mathbf{q}}_{i}\,,\\
  \end{array}
\end{equation}
and write
\begin{equation}
  {\bf q}^{{\rm TVD,minmod}}(x) = \bar{{\bf q}}_i + \frac{x -
  x_i}{x_{i+1/2} - x_{i-1/2}} \mathbf{minmod}(\Delta_{\mathrm{upw}},\Delta_{\mathrm{loc}}), 
    \qquad x \in [
  x_{i-1/2}, x_{i+1/2} ]. 
\end{equation}
\subsection{PPM reconstruction}
\label{sec:ppm}

The piecewise parabolic method (PPM) of Colella and
Woodward~\cite{ppm} is a rather more complex method that requires a
number of steps. The implementation in the {\tt GRHydro} code is specialized
to use evenly spaced grids. Also, some of the more complex features are not
implemented; in particular, the dissipation algorithm is only the
simplest given in the original article. Here we just give the implementation
details. For more details on the method we refer to the original
article.

Again we assume we are reconstructing a scalar function $q$ as a
function of $x$ in one dimension on an evenly spaced grid, with spacing $\Delta
x$. The first step is to interpolate a quadratic polynomial to the cell
boundary, 
\begin{equation}
  \label{eq:ppm1}
  q_{i+1/2} = \frac{1}{2} \left( q_{i+1} + q_i \right) + \frac{1}{6}
  \left( \delta_m q_i - \delta_m q_{i+1} \right), 
\end{equation}
where
\begin{equation}
  \label{eq:ppmdm1}
  \delta_m q_i = \left\{ \begin{array}{c l} \text{min}(|\delta q_i|,
      2|q_{i+1} - q_i|, 2|q_i - q_{i-1}|) \text{ sign}(\delta q_i) &
      \text{if } (q_{i+1} - q_i)(q_i - q_{i-1}) > 0, \\
      0 & \text{otherwise}. \end{array} \right.,
\end{equation}
and
\begin{equation}
  \label{eq:ppmd1}
  \delta q_i = \frac{1}{2}(q_{i+1} - q_{i-1}).
\end{equation}
At this point we set both left and right states at the interface to be
equal to this,
\begin{equation}
  \label{eq:ppmset1}
  q_i^R = q_{i+1}^L = q_{1+1/2}.
\end{equation}

This reconstruction will be oscillatory near shocks. To preserve
monotonicity, the following replacements are made:
\begin{eqnarray}
  \label{eq:ppmmonot}
  q_i^L = q_i^R = q_i & \text{if} & (q_i^R - q_i)(q_i - q_i^L) \leq
  0 \\
  q_i^L = 3 q_i - 2q_i^R & \text{if} & (q_i^R - q_i^L)\left( q_i -
  \frac{1}{2} (q_i^L + q_i^R) \right) > \frac{1}{6}(q_i^R - q_i^L)^2
  \\ 
  q_i^R = 3 q_i - 2q_i^L & \text{if} & (q_i^R - q_i^L)\left( q_i -
  \frac{1}{2} (q_i^L + q_i^R) \right) < -\frac{1}{6}(q_i^R -
  q_i^L)^2. 
\end{eqnarray}

However, before applying the monotonicity preservation two other steps
may be applied. Firstly we may steepen discontinuities. This is to
ensure sharp profiles and is only applied to contact
discontinuities. This may be switched on or off using the parameter
{\tt ppm\_detect}. This part of the method replaces the cell boundary
reconstructions of the density with
\begin{eqnarray}
  \label{eq:ppmdetect}
  \rho_i^L & = & \rho_i^L (1-\eta) + \left(\rho_{i-1} + \frac{1}{2}
    \delta_m \rho_{i-1} \right) \eta \\
  \rho_i^R & = & \rho_i^R (1-\eta) + \left(\rho_{i+1} - \frac{1}{2}
    \delta_m \rho_{i+1} \right) \eta
\end{eqnarray}
where $\eta$ is only applied if the discontinuity is mostly a contact
(see~\cite{ppm} for the details) and is defined as
\begin{equation}
  \label{eq:ppmeta}
  \eta = \text{max}(0, \text{min}(1, \eta_1 (\tilde{\eta} - \eta_2))),
\end{equation}
where $\eta_1,\eta_2$ are positive constants and
\begin{equation}
  \label{eq:ppmetatilde}
  \tilde{\eta} = \left\{ \begin{array}{c l} 
  \frac{\rho_{i-2} - \rho_{i+2} + 4 \delta\rho_i}{12\delta\rho_i} &
  \text{ if } \left\{
\begin{array}{l}
  \delta^2\rho_{i+1}\delta^2\rho_{i-1} < 0\\
  (\rho_{i+1} - \rho_{i-1}) - \epsilon \text{min}(|\rho_{i+1}|,|\rho_{i-1}|) >
  0\nonumber
\end{array} \right. \\
  0 & \text{otherwise} \end{array} \right.,
\end{equation}
with $\epsilon$ another positive constant and
\begin{equation}
  \label{eq:ppmd2rho}
  \delta^2\rho_i = \frac{\rho_{i+1} - 2\rho_i + \rho_{i-1}}{6\Delta
    x^2}.  
\end{equation}

Another step that is performed before monotonicity enforcement is to
flatten the zone structure near shocks. This adds simple dissipation
and is always in the code. In short, the reconstructions are again
altered to
\begin{equation}
  \label{eq:ppmflatten}
  q_i^{L,R} = \nu_i q_i^{L,R} + (1 - \nu_i) q_i,
\end{equation}
where
\begin{equation}
  \label{eq:ppmflatten2}
  \nu_i = \left\{ \begin{array}{c l} {\rm max}(0, 1 - \text{max}(0, \omega_2
      (\frac{p_{i+1} - p_{i-1}}{p_{i+2} - p_{i-2}} - \omega_1))) & \text{
        if } \omega_0 \text{min}(p_{i-1}, p_{i+1}) < |p_{i+1} - p_{i - 1}|
      \text{ and } v^x_{i-1} - v^x_{i+1} > 0 \\
      1 & \text{otherwise} \end{array} \right.
\end{equation}
and $\omega_0, \omega_1,\omega_2$ are constants.

The above flattening procedure is not the one in the original article of Colella and Woodward, but
it has been adapted from it in order to have a stencil of three points.  The original flattening
procedure is also implemented in {\tt GRHydro}. Instead of \ref{eq:ppmflatten}, it consists in the formula
\begin{equation}
  \label{eq:ppmflatten-stencil4}
  q_i^{L,R} = \tilde \nu_i q_i^{L,R} + (1 - \tilde \nu_i) q_i,
\end{equation}
where
\begin{eqnarray}
\tilde \nu_i &=& {\rm max}\Big(\nu_i,\nu_{i+{\rm sign}(p_{i-1}-p_{i+1})}\Big)
\end{eqnarray}
and $\nu_i$ is given by \ref{eq:ppmflatten2}. This can be activated by setting the parameter {\tt
ppm\_flatten} to {\tt stencil\_4}. Formula \ref{eq:ppmflatten-stencil4}, despite requiring
more computational resources (especially when mesh refinement is used), usually gives very similar
results to \ref{eq:ppmflatten}, so we routinely use \ref{eq:ppmflatten}.


\subsection{ENO Reconstruction}
\label{sec:eno}

An alternative way of getting higher-than-second-order accuracy is the implementation of the
Essentially Non-Oscillatory methods of Harten et.al~\cite{eno}. The
essential idea is to alter the stencil to use those points giving the
smoothest reconstruction. The only restriction is that the stencil
must include the cell to be reconstructed (for stability). Here we
describe the simplest ENO type reconstruction: piecewise polynomial
reconstruction using the (un)divided differences to measure the
smoothness. 

Let $k$ be the order of the reconstruction. Suppose we are
reconstructing the scalar function $q$ in cell $i$. We start with the
cell $I_i$. We then add to the stencil cell $I_j$, where $j = i \pm
1$, where we choose $j$ to minimize the Newton divided differences
\begin{eqnarray}
  \label{enodd}
  q \left[ x_{i-1}, x_i \right] & = & \frac{q_i - q_{i-1}}{x_{i+1/2}
    - x_{i-3/2}} \\
  q \left[ x_i, x_{i+1} \right] & = & \frac{q_{i+1} - q_i}{x_{i+3/2}
    - x_{i-1/2}}.
\end{eqnarray}
\noindent We then recursively add more cells, minimizing the higher-order 
Newton divided differences $q \left[ x_{i-k}, \dots, x_{i+j} \right]$
defined by 
\begin{equation}
  \label{enodd2}
  q \left[ x_{i-k}, \dots, x_{i+j} \right] = \frac{ q \left[ x_{i-k+1},
  \dots, x_{i+j} \right] - q \left[ x_{i-k}, \dots, x_{i+j-1} \right]
  }{x_{i+j} - x_{i-k}}.
\end{equation}
\noindent The reconstruction at the cell boundary is given by a
standard $k^{\text{th}}$-order polynomial interpolation on the chosen
stencil. 

\cite{shueno} has outlined an elegant way of calculating the cell
boundary values solely in terms of the stencil and the known data. If
the stencil is given by
\begin{equation}
  \label{enostencil1}
  S(i) = \left\{ I_{i-r}, \dots, I_{i+k-r-1} \right\},
\end{equation}
\noindent for some integer $r$, then there exist constants $c_{rj}$
depending only on the grid $x_i$ such that the boundary
values for cell $I_i$ are given by
\begin{equation}
  \label{enoc1}
  q_{i+1/2} = \sum_{j=0}^{k-1} c_{rj} q_{i-r+j}, \qquad q_{i-1/2} =
  \sum_{j=0}^{k-1} c_{r-1,j} q_{i-r+j}. 
\end{equation}
\noindent The constants $c_{rj}$ are given by the rather complicated
formula 
\begin{equation}
  \label{enoc2}
  c_{rj} = \left\{ \sum_{m=j+1}^k \frac{ \sum_{l=0, l \neq m}^k
  \prod_{q=0, q \neq m, l}^k \left(  x_{i+1/2} - x_{i-r+q-1/2} \right)
  }{ \prod_{l=0, l \neq m}^k \left(  x_{i-r+m-1/2} - x_{i-r+L-1/2}
  \right) }  \right\} \Delta x_{i-r+j}. 
\end{equation}
\noindent This simplifies considerably if the grid is even. The
coefficients for an even grid are given (up to seventh order)
by~\cite{shueno}.

\section{Riemann Problems}
\label{sec:riemann}

Given the reconstructed data, we then solve a local Riemann problem in
order to get the intercell flux. The Riemann problem is specified by
an equation in flux conservative homogeneous form,
\begin{equation}
  \label{eq:homconsform1}
  \partial_t {\bf q} + \partial_{x^i} {\bf f}^{(i)} ({\bf q}) = 0
\end{equation}
with piecewise constant initial data ${\bf q}_{_{L,R}}$ separated by a
discontinuity at $x^{(1)}=0$. Flux terms for the other directions are
given similarly.  There is no intrinsic scale to this problem and so
the solution must be a self similar solution with similarity variable
$\xi = x^{(1)}/t$. The solution is given in terms of {\it waves} which
separate different {\it states}, with each state being constant. The
waves are either {\it shocks}, across which all hydrodynamical
variables change discontinuously, {\it rarefactions}, across which all
the variables change continuously (the wave is not a single value of
$\xi$ for a rarefaction, but spreads across a finite range), or {\it
contact or tangential} discontinuities, across which some but not all
of the hydrodynamical variables change discontinuously and the rest
are constant. The characteristics of the matter evolution converge and
break at a shock, diverge at a rarefaction and are parallel at the
other linear discontinuities.

The best references for solving the Riemann problem either exactly or
approximately are~\cite{leveque}, \cite{toro}, \cite{laney}. Here, we start by 
giving a simple outline. We
start by considering the $N$ dimensional linear problem in one
dimension given by
\begin{equation}
%  \label{eq:linsys1}
  \label{lsrp}
  \partial_t {\bf q} + A \partial_{x} {\bf q} = 0  \ ,
\end{equation}
where $A$ is a $N\times N$ matrix with constant coefficients. We
define the eigenvalues $\lambda^j$ with associated right and left
eigenvectors ${\bf r}^j,{\bf l}_j$, where the eigenvectors are
normalized to each other ({\it i.e.}, their dot product is
$\delta^i_j$). 
%%%%This was sort of written twice
%Defining the characteristic variables ${\bf w}_i$ as
%\begin{equation}
%  \label{eq:charvar1}
%  {\bf w}_i = {\bf l}^j_i \cdot {\bf q}_j \ ,
%\end{equation}
%then equation (\ref{eq:linsys1}) transforms to a set of uncoupled
%linear equations
%\begin{equation}
%  \label{eq:linsys2}
%  \partial_t {\bf w}_i + \Lambda \partial_{x} {\bf w}_i = 0   \ ,
%\end{equation}
%where the matrix $\Lambda$ contains the eigenvalues on the diagonals
%and is zero elsewhere. By convention the eigenvalues are ordered
%$\lambda_1 < \dots < \lambda_N$.
%
%The solution to one of the uncoupled equations along the cell boundary
%is simply that the characteristic variable is given by
%\begin{equation}
%  \label{eq:linflux}
%  {\bf w}_i = \left\{ \begin{array}[c]{r c l} ({\bf w}_i)_L & {\rm if}
%      & \lambda_i > 0 \\ 
%      ({\bf w}_i)_R & {\rm if} & \lambda_i < 0. \end{array}\right\},
%\end{equation}
%where, as before, subscripts $L$ and $R$ are used to denote quantities
%obtained from cells on the left and right, respectively, of the intercell 
%interface.
%To get the intercell flux we transform the solution back to the
%conserved variables by taking the dot product with the right
%eigenvectors and then multiplying by the matrix $A$. This can be written
%in a number of ways; the form we use is
%\begin{equation}
%  \label{eq:linsysflux}
%  {\bf f}_{i+1/2} = \frac{1}{2} \left[ A({\bf q}_{_L}) + A({\bf q}_{_R})
%  - \Sigma_{j=1}^N |\lambda_j| \Delta {\bf w}_j {\bf r} \right], 
%\end{equation}
%where $\Delta {\bf w}^j$ is the jump in the appropriate characteristic
%variable across the $j^{{\rm th}}$ wave. We note that the
%characteristic jumps are easily calculated from
%\begin{equation}
%  \label{eq:charjump}
%  \Delta {\bf w}_i = {\bf l}_i^j \left[ ({\bf q}_R)_j - ({\bf q}_L)_j
%  \right]. 
%\end{equation}
%
%The Roe flux is simply given by locally assuming that the conserved
%variables are constant. Thus the matrix $A$ is simply given by
%evaluating the Jacobian matrix $\partial {\bf f}({\bf q}) / \partial
%{\bf q}$ at some average state. Equation~(\ref{eq:linsysflux}) is then
%used to evaluate the intercell flux. The only questions are
%\begin{itemize}
%\item What is an appropriate intermediate state?
%\item When is this approximate (in)valid?
%\end{itemize}
%
%\subsection{Linear system}
%%
%Let ${\bf l}_i$, ${\bf r}^i$ be the left and right eigenvectors,
%respectively, of the matrix $A$ with eigenvalues $\lambda_i$,
%normalized such that ${\bf l}_i \cdot {\bf r}^j = \delta_i^j$. 
%
\label{sec:linsys}
We shall assume that the eigenvectors span the space. The characteristic
variables ${\bf w}_i$ are defined by
\begin{equation}
  \label{charvar}
  {\bf w}_i = {\bf l}_i \cdot {\bf q}.
\end{equation}
\noindent Then equation (\ref{lsrp}) when written in terms of the
characteristic variables becomes
\begin{equation}
  \label{charvarrp}
  \partial_t {\bf w} + \Lambda \partial_x {\bf w} = 0,
\end{equation}
\noindent where $\Lambda$ is the matrix containing the eigenvalues
$\lambda_i$ on the diagonals and zeros elsewhere. Hence each
characteristic variable ${\bf w}^i$ obeys the linear advection
equation with velocity $a = \lambda_i$. This solves the Riemann
problem in terms of characteristic variables.

In order to write the solution in terms of the original variables
${\bf q}$ we order the variables in such a way  that $\lambda_1 \leq \dots \leq
\lambda_N$. We also define the differences in the characteristic
variables $\Delta {\bf w}_i = ({\bf w}_i)_L - ({\bf w}_i)_R$ across
the $i^{{\rm th}}$ characteristic wave. These differences are single
numbers (`scalars'). We note that these differences can easily be
found from the initial data using
\begin{equation}
  \label{dw}
  \Delta {\bf w}_i = {\bf l}_i \cdot \left( {\bf q}_L - {\bf q}_R
  \right). 
\end{equation}
\noindent As the change in the solution across each wave is precisely
the difference in the associated characteristic variable, the solution
of the Riemann problem in terms of characteristic variables can be
written as either
\begin{equation}
  \label{lsrpsol1}
  {\bf w}_i = ({\bf w}_i)_L + \sum_{j=1}^M \Delta {\bf w}_j {\bf e}^j \quad
  {\rm if}\ \lambda_M < \xi < \lambda_{M+1},
\end{equation}
\noindent or 
\begin{equation}
  \label{lsrpsol2}
  {\bf w} = ({\bf w}_i)_R - \sum_{j=M+1}^N \Delta {\bf w}_j {\bf e}^j
  \quad {\rm if}\ \lambda_M < \xi < \lambda_{M+1},
\end{equation}
\noindent or as the average
\begin{equation}
  \label{lsrpsol3}
  {\bf w}_i = \frac{1}{2} \left( ({\bf w}_i)_L + ({\bf w}_i)_R +
  \sum_{j=1}^M \Delta 
  {\bf w}_j {\bf e}^j - \sum_{j=M+1}^N \Delta {\bf w}_j {\bf e}^j 
  \right) \quad {\rm if}\ \lambda_M < \xi < \lambda_{M+1},
\end{equation}
\noindent where ${\bf e}^i$ is the column vector $({\bf e}^i)_j =
\delta^i_j$. 

Converting back to the original variables ${\bf q}$ we have the
solution
\begin{equation}
  \label{lsrpsol4}
  {\bf q} = \frac{1}{2} \left( {\bf q}_L + {\bf q}_R + \sum_{i=1}^M \Delta
  {\bf w}_i {\bf r}^i - \sum_{i=M+1}^N \Delta {\bf w}_i {\bf r}^i
  \right) \quad {\rm if}\ \lambda_M < \xi < \lambda_{M+1}.
\end{equation} 
\noindent In the case where we are only interested in the flux
along the characteristic $\xi = 0$ we can write the solution in the
simple form
\begin{equation}
  \label{lsrpsol6}
  {\bf f}({\bf q}) = \frac{1}{2} \left( {\bf f}({\bf q}_L) + {\bf f}({\bf
  q}_R) - \sum_{i=1}^N | \lambda_i | \Delta {\bf w}_i {\bf r}^i \right).
\end{equation} 

All exact Riemann solvers have to solve at least an implicit
equation and so are computationally very expensive. As the 
solution of Riemann problems takes a large portion of
the time to run in a HRSC code, approximations that speed the
calculation of the intercell flux are often essential. This is
especially true in higher dimensions (>1), where the solution of the
ordinary differential equation to give the relation across a
rarefaction wave makes the use of an exact Riemann solver impractical.

Approximate Riemann solvers can have problems, as shown in depth by
Quirk~\cite{Quirk}. Hence it is best to compare the
results of as many different solvers as possible. Here we shall describe
the three approximate solvers used in this code, starting with the
simplest.

\subsection{HLLE solver}
\label{sec:hlle}

The Harten-Lax-van Leer-Einfeldt (HLLE) solver of Einfeldt~\cite{Einfeldt88} is a
simple two-wave approximation. We assume that the maximum and minimum wave speeds
$\xi_{\pm}$ are known. The solution of the Riemann problem is then
given by requiring conservation to hold across the waves. The solution
takes the form
\begin{equation}
  \label{hlle1}
  {\bf q} = \left\{ \begin{array}[c]{r c l} {\bf q}_L & {\rm if} & \xi
        < \xi_- \\  {\bf  q}_* & {\rm if} & \xi_- < \xi < \xi_+ \\
        {\bf q}_R & {\rm if} & \xi   > \xi_+, \end{array}\right. 
\end{equation}
\noindent and the intermediate state ${\bf q}_*$ is given by
\begin{equation}
  \label{hlle2}
  {\bf q}_* = \frac{\xi_+ {\bf q}_R - \xi_- {\bf q}_L - {\bf f}({\bf
  q}_R) + {\bf f}({\bf q}_L)}{\xi_+ - \xi_-}.
\end{equation}
\noindent If we just want the numerical flux along the boundary then
this takes the form
\begin{equation}
  \label{hlleflux}
  {\bf f}({\bf q}) = \frac{\widehat{\xi}_+{\bf f}({\bf q}_L) -
  \widehat{\xi}_-{\bf f}({\bf q}_R) + \widehat{\xi}_+ \widehat{\xi}_-
  ({\bf q}_R - {\bf q}_L)}{\widehat{\xi}_+ - \widehat{\xi}_-},
\end{equation}
\noindent where
\begin{equation}
  \label{hlle3}
  \widehat{\xi}_- = {\rm min}(0, \xi_-), \quad \widehat{\xi}_+ =
  {\rm max}(0, \xi_+). 
\end{equation}

Knowledge of the precise minimum and maximum characteristic velocities
$\xi_{\pm}$ requires knowing the solution of the Riemann problem.
Instead, the characteristic velocities are usually found from the
eigenvalues of the Jacobian matrix $\partial {\bf f} / \partial {\bf
  q}$ evaluated at some intermediate state. To ensure that the maximum
and minimum eigenvalues over the entire range between the left and
right states are found, we evaluate the Jacobian in both the left and
right states and take the maximum and minimum over all eigenvalues.
This ensures, for the systems of equations considered here, that the
real maximum and minimum characteristic velocities are contained
within $[\xi_-, \xi_+]$.

If we set $\alpha = {\rm max}(|\xi_-|, |\xi_+|)$ and replace the
characteristic velocities $\xi_{\pm}$ with $\pm \alpha$, we find the
Lax--Friedrichs flux ({\it cf.} also Tadmor's semi-discrete scheme~\cite{Tadmor00})
\begin{equation}
  \label{lfflux}
  {\bf f}({\bf q}) = \frac{1}{2} \left[ {\bf f}({\bf q}_L) + {\bf f}({\bf
  q}_R) + \alpha ({\bf q}_L -{\bf q}_R) \right].
\end{equation}
\noindent This is very diffusive, but also very stable.


\subsection{Roe solver}
\label{sec:roe}

The linearized solver of Roe~\cite{Roe81} is probably the most popular
approximate Riemann solver. The simplest interpretation is that the
Jacobian $\partial {\bf f} / \partial {\bf q}$ is linearized about
some intermediate state. Then the conservation form reduces to the
linear equation
\begin{equation}
  \label{roe1}
  \partial_t {\bf q} + A \partial_x {\bf q} = 0,
\end{equation}
\noindent where $A$ is a constant coefficient matrix. This is
identical to equation (\ref{lsrp}) and so all the results of
section~\ref{sec:linsys} on linear systems hold. We reiterate that the
standard form for the flux along the characteristic ray $\xi=0$ is
\begin{equation}
  \label{roe2}
  {\bf f}({\bf q}) = \frac{1}{2} \left( {\bf f}({\bf q}_L) + {\bf f}({\bf
  q}_R) - \sum_{i=1}^N | \lambda_i | \Delta {\bf w}_i {\bf r}^i \right).
\end{equation} 

There is a choice of which intermediate state the Jacobian should be
evaluated at. Roe gives three criteria that ensure the consistency and
stability of the numerical flux:
\begin{enumerate}
\item $A({\bf q}_{{\rm Roe}}) \left( {\bf q}_R - {\bf q}_L \right) =
  {\bf f}({\bf q}_R) - {\bf f}({\bf q}_L)$,
\item $A({\bf q}_{{\rm Roe}})$ is diagonalizable with real
  eigenvalues, 
\item $A({\bf q}_{{\rm Roe}}) \rightarrow \partial {\bf f} / \partial
  {\bf q}$ smoothly as ${\bf q}_{{\rm Roe}} \rightarrow {\bf q}$.
\end{enumerate}
\noindent A true Roe average for relativistic hydrodynamics, {\it i.e.}, an
intermediate state that satisfies all these conditions, has been
constructed by Eulderink~\cite{Eulderink94}. However, frequently it is sufficient
to use
\begin{equation}
  \label{roe3}
  {\bf q}_{{\rm Roe}} = \frac{1}{2} \left( {\bf q}_R + {\bf q}_L \right),
\end{equation}
\noindent which satisfies only the last two conditions. For simplicity
we have implemented this arithmetic average.


\subsection{Marquina solver}
\label{sec:marq}

Unlike all the other Riemann solvers introduced so far, the Marquina
solver as outlined in \cite{Marquina1} does not solve the Riemann
problem completely. Instead, only the flux along the characteristic
ray $\xi=0$ is given. It can be seen as a generalized Roe solver, as
the results are the same except at sonic points. These points are
where the fluid velocity is equal to the speed of sound of the fluid.
In the context of Riemann problems, sonic points are found when the
ray $\xi=0$ is within a rarefaction wave.

Firstly define the left ${\bf l}({\bf q}_{L,R})$ and right ${\bf
  r}({\bf q}_{L,R})$ eigenvectors and the eigenvalues $\lambda({\bf
  q}_{L,R})$ of the Jacobian matrix $\partial {\bf f} / \partial {\bf
  q}$ evaluated at the left and right states. Next define left and
right characteristic variables ${\bf w}_{L,R}$ and fluxes
${\bf \phi}_{L,R}$ by
\begin{equation}
  \label{marq1}
  ({\bf w}_i)_{L,R} = {\bf l}_i({\bf q}_{L,R}) \cdot {\bf q}_{L,R}, \quad
  ({\bf \phi}_i)_{L,R} = {\bf l}_i({\bf q}_{L,R}) \cdot {\bf f}({\bf
  q}_{L,R}). 
\end{equation}

Then the algorithm chooses the correct-sided characteristic flux if
the eigenvalues $\lambda_i({\bf q}_L)$, $\lambda_i({\bf q}_R)$ have
the same sign, and uses a Lax--Friedrichs type flux if they change
sign. In full, the algorithm is given in figure~\ref{fig:marqcode}.
\begin{figure}[htbp]
  \begin{center}
    \leavevmode
\begin{equation}
  \label{marqalg}
  \begin{array}[l]{l}
    {\bf For}\ \, i = 1, \dots, N\ {\bf do} \\
    \qquad \begin{array}[c]{l}
      {\bf If}\ \, \lambda_i({\bf q}_L) \lambda_i({\bf q}_R) > 0
        \ \, {\bf then} \\
      \qquad \qquad \begin{array}[c]{l}
        {\bf If}\ \,  \lambda_i({\bf q}_L) > 0 \ \, {\bf then} \\
        \qquad \qquad \qquad \begin{array}[c]{r c l}
          {\bf \phi}^i_+ & = & {\bf \phi}^i_L \\
          {\bf \phi}^i_- & = & 0
        \end{array} \\
        {\bf else}  \\
        \qquad \qquad \qquad \begin{array}[c]{r c l}
          {\rm \phi}^i_+ & = & 0 \\
          {\rm \phi}^i_- & = & {\bf \phi}^i_R
        \end{array} \\
        {\bf end if}
      \end{array} \\
      {\bf else} \\
      \qquad \qquad \begin{array}[c]{r c l}
        \alpha^i & = & {\rm max}(|\lambda_i({\bf q}_L), \lambda_i({\bf
          q}_R)|) \\
        {\bf \phi}^i_+ & = & \frac{1}{2} \left( {\bf \phi}^i_L +
          \alpha^i {\bf w}^i_L \right) \\
        {\bf \phi}^i_- & = & \frac{1}{2} \left( {\bf \phi}^i_R -
          \alpha^i {\bf w}^i_R \right)
      \end{array} \\
      {\bf end if} \\
    \end{array} \\
    {\bf end do}
  \end{array}
\end{equation}
    \caption[The algorithm to calculate the Marquina flux]{The
      algorithm to calculate the Marquina flux.}
    \label{fig:marqcode}
  \end{center}
\end{figure}

Then the numerical flux is given by
\begin{equation}
  \label{marqflux}
  {\bf f}({\bf q}) = \sum_{i=1}^N \left[ {\bf \phi}^i_+ 
  {\bf r}^i ({\bf q}_L) + {\bf \phi}^i_- {\bf r}^i ({\bf q}_R)
  \right].
\end{equation}

The above implementation is based on \cite{Aloy99b}.


\section{Other points in {\tt GRHydro}}
\label{sec:misc}

There are a number of other things done by {\tt GRHydro} which, whilst not as
important as reconstruction and evolution, are still essential.


\subsection{Source terms}
\label{sec:sources}

In a curved spacetime the equations are not in homogeneous conservation-law 
form but also contain source terms. These are actually calculated
first, before the flux terms (it simplifies the loop very slightly).
There are a few points to note about the calculation of the sources.
\begin{itemize}
\item The calculation used here, taken from the {\tt GR3D} code, requires both the
  metric and the extrinsic curvature.
\item In order to calculate the Christoffel symbols the gauge and
  metric variables must be differenced. Currently centred differencing
  of second or fourth (we are safe to use this, as {\tt GRHydro} requires 
  always at least 2 ghost zones) order is hardwired in. The two differencings can be selected via
  the parameter {\tt GRHydro::sources\_spatial\_order}.
\item For numerical reasons, namely in order to avoid the presence of time derivatives
  in the source-term computation, the implemented form of the source terms is not
  \eqref{source_terms} directly, but it has been modified as shown in
  the following paragraphs (following a clever idea by Mark Miller (see the {\tt GR3D} code) 
\end{itemize}

%The actual derivation of the source terms is somewhat complex and not
%at all obvious. If anybody comes up with a better way of doing things
%than the below, please alter the documentation. 
In what follows Greek letters range from $0$ to $3$ and roman letters from $1$ to $3$.

For the following computations, we need the expression of some of the 4-Christoffel symbols
$\ {}^{(4)}\Gamma^\rho_{\mu\nu}$ applied to the 3+1 decomposed
variables. In order to remove time derivatives we will frequently make
use of the ADM evolution equation for the 3-metric in the form
%
\begin{equation}
  \label{eq:SourceADMg}
  \partial_t \gamma_{ij} = 2\left(- \alpha K_{ij} + \partial_{(i}
  \beta_{j)} - {}^{(3)}\Gamma^k_{ij} \beta_k \right)\ .
\end{equation}
%
As it is used in what follows, we also recall that $\nabla$ is the
covariant derivative associated with the spatial 3-surface and we note that it is
compatible with the 3-metric:
%
\begin{eqnarray}
\label{compatible_derivative}
\nabla_i\gamma^{jk}=\partial_i\gamma^{jk} + 2{}^{(3)}\Gamma^j_{il}\gamma^{lk} = 0 \ .
\end{eqnarray}
%
We start from the ${}^{(4)}\Gamma^0_{00}$ symbol:
%
\begin{eqnarray}
\label{eq:Gamma000}
{}^{(4)}\Gamma^0_{00} = \frac{1}{2\alpha^2}\Big[
-\partial_t\big(\beta_k\beta^k\big)+2\alpha\partial_t\alpha
+ 2\beta^i\partial_t\beta_i - \beta^i\partial_i\big(\beta_k\beta^k\big) + 2\alpha\beta^i\partial_i\alpha\Big]
\end{eqnarray}
%
and we expand the derivatives as
%
\begin{eqnarray} \label{de_t_beta2}
\partial_t\big(\beta_k\beta^k\big) &=&
\partial_t\big(\gamma_{jk}\beta^j\beta^k\big) = 2\gamma_{jk}\beta^j\partial_t\beta^k +
\beta^j\beta^k\partial_t\gamma_{jk} = \nonumber \\
 &=& 2\beta_k\partial_t\beta^k -2\alpha K_{jk}
\beta^j\beta^k + 2\beta^j\beta^k\partial_j\beta_k - 2{}^{(3)}\Gamma^i_{kj} \beta_i\beta^j\beta^k
\end{eqnarray}
%
and
%
\begin{eqnarray}  \label{de_i_beta2}
\partial_i\big(\beta_k\beta^k\big) =
\partial_i\big(\gamma^{jk}\beta_j\beta_k\big) =
2\gamma^{jk}\beta_j\partial_i\beta_k + \beta_j\beta_k\partial_i\gamma^{jk} =
2\beta_k\partial_i\beta_k -2{}^{(3)}\Gamma^j_{ik}\beta_j\beta^k \ ,
\end{eqnarray}
%
where we have used \eqref{eq:SourceADMg} and
\eqref{compatible_derivative}, respectively.
Inserting \eqref{de_t_beta2} and \eqref{de_i_beta2}, equation
\eqref{eq:Gamma000} becomes
%
\begin{eqnarray}
  \label{eq:Gamma000_final}
{}^{(4)}\Gamma^0_{00} = \frac{1}{\alpha}\Big(\partial_t\alpha +
\beta^i\partial_i\alpha + K_{jk}\beta^j\beta^k \Big)\ .
\end{eqnarray}
%
With the same strategy we then compute
%
\begin{eqnarray}
  \label{eq:SourceChr1a}
  {}^{(4)}\Gamma^0_{i0} & = & - \frac{1}{2\alpha^2} \Big[
      \partial_i (\beta^k \beta_k - \alpha^2) - \beta^j (\partial_i
      \beta_j - \partial_j \beta_i + \partial_t \gamma_{ij}) \Big] = - \frac{1}{\alpha} \Big(\partial_i\alpha - \beta^j K_{ij}\Big)
\end{eqnarray}
%
and
%
\begin{eqnarray}
\label{eq:SourceChr0ij}
  {}^{(4)}\Gamma^0_{ij} & = & - \frac{1}{2\alpha^2} \Big[
      \partial_i\beta_j + \partial_j\beta_i - \partial_t\gamma_{ij} - \beta^k
  (\partial_i\gamma_{kj} + \partial_j\gamma_{ki} - \partial_k\gamma_{ij})\Big] = -
  \frac{1}{\alpha} K_{ij}\ .
\end{eqnarray}
%
Other more straightforward calculations give
%
\begin{alignat}{3}
  \label{eq:SourceS3a}
  {}^{(4)}\Gamma_{00j} &=& {}^{(4)}\Gamma^\nu_{0j}g_{\nu 0} & =  \frac{1}{2} \partial_j \left( \beta_k
    \beta^k - \alpha^2 \right), \\
\nonumber\\
  \label{eq:SourceS3b}
  {}^{(4)}\Gamma_{l0j} &=& {}^{(4)}\Gamma^\nu_{lj}g_{\nu 0} & =  \alpha K_{lj} + \partial_l\beta_j + \partial_j\beta_l - \beta_k{}^{(3)}\Gamma^k_{lj}\ , \\
\nonumber\\
  \label{eq:SourceS3c}
  {}^{(4)}\Gamma_{0lj} &=& {}^{(4)}\Gamma^\nu_{0j}g_{\nu l} & =  -\alpha K_{jl} + \partial_l\beta_j - \beta_k {}^{(3)}\Gamma^k_{lj}\ , \\
\nonumber\\
  \label{eq:SourceS3d}
  {}^{(4)}\Gamma_{lmj} &=& {}^{(4)}\Gamma^\nu_{lj}g_{\nu m} & =  {}^{(3)}\Gamma_{lmj}\ ,
\end{alignat}
%
where~\eqref{eq:SourceADMg} was used to derive \eqref{eq:SourceS3b}
and~\eqref{eq:SourceS3c}.
\subsubsection{Source term for $S_k$}
Now we have all the expressions for calculating the source terms. The
ones for the variables $S_{\,k}$ are
%
\begin{equation}
  \label{eq:SourceS1}
  \big({\mathcal S}_{S_k}\big)_j = T^\mu_\nu \Gamma^\nu_{\mu j} = T^{\mu\nu} \Gamma_{\mu\nu j}\ .
\end{equation}
%
After expanding the derivative in
\eqref{eq:SourceS3a}, the coefficient of the $T^{\ 00}$ term in
\eqref{eq:SourceS1} becomes
%
\begin{eqnarray}
  \label{eq:SourceS4a}
  {}^{(4)}\Gamma_{00j} & = & \frac{1}{2} \beta^l \beta^m \partial_j
  \gamma_{lm} - \alpha \partial_j \alpha + \beta_m \partial_j \beta^m.
\end{eqnarray}
%
The coefficient of the $T^{\,0i}$ term 
% (or the $T^{i0}$ term, as the stress-energy tensor is symmetric) 
is
%
\begin{eqnarray}
  \label{eq:SourceS5a}
  {}^{(4)}\Gamma_{0ij} + {}^{(4)}\Gamma_{i0j} = \partial_j\beta_i = \beta^l \partial_i
  \gamma_{jl} + \gamma_{il} \partial_j \beta^l.
\end{eqnarray}
%
The coefficient of the $T^{\,lm}$ term is simply
%
\begin{eqnarray}
  \label{eq:SourceS6a} 
{}^{(3)}\Gamma_{lmj} =
  \frac{1}{2} \Big (\partial_j\gamma_{ml} + \partial_m\gamma_{jl} - \partial_l\gamma_{mj} \Big).
\end{eqnarray}
%
Finally, summing \eqref{eq:SourceS4a}--\eqref{eq:SourceS6a} we
find
%
\begin{eqnarray}
  \label{eq:SourceS2a} 
\big({\mathcal S}_{S_k}\big)_j & = &
  T^{00}\left( \frac{1}{2} \beta^l \beta^m \partial_j \gamma_{lm} -
  \alpha \partial_j \alpha \right) + T^{0i} \beta^l \partial_j
  \gamma_{il} + T^0_i\partial_j \beta^i + \frac{1}{2} T^{lm}
  \partial_j \gamma_{lm} \ ,
\end{eqnarray}
%
which is the expression implemented in the code. 
%This also has the advantage that all partial derivatives of the shift are collected in
%one term.
% ACTUALLY, NO PARTICULAR EFFORT WAS MADE TO COLLECT THE SHIFT DERIVATIVES...

\subsubsection{Source term for $\tau$}

The source term for $\tau$ is [{\it cf.} \eqref{source_terms}]
%
\begin{equation}
  \label{eq:SourceT1}
  {\mathcal S}_{\tau} = \alpha \left( T^{\mu 0} \partial_{\mu}
  \alpha - \alpha T^{\mu\nu} {}^{(4)}\Gamma^0_{\mu\nu}\right).
\end{equation}
%
For clarity, again we consider separately the terms containing as a
factor the different components of $T^{\mu\nu}$. From
\eqref{eq:Gamma000_final} we find the coefficient of $T^{\,00}$ to be
%
\begin{eqnarray}
  \label{eq:SourceT3a}
\alpha\big(\partial_t \alpha -\alpha {}^{(4)}\Gamma^0_{00}\big) =
-\alpha\big( \beta^i \partial_i \alpha + \beta^k \beta^l K_{kl}\big)\ .
\end{eqnarray}
%
The coefficient of the term $T^{\,0i}$ is given by
% Here we note that we it is T^{0i} + T^{i0}
%
\begin{eqnarray}
  \label{eq:SourceT4a}
  \alpha\big(\partial_i \alpha - 2 \alpha {}^{(4)}\Gamma^0_{i0}\big) =
  2 \alpha\beta^j K_{ij} - \alpha\partial_i \alpha
\end{eqnarray}
%
and, finally, the coefficient for $T^{\,ij}$ is
%
\begin{eqnarray}
  \label{eq:SourceT5a}
  -\alpha^2 {}^{(4)}\Gamma^0_{ij} = \alpha K_{ij}\ .
\end{eqnarray}
The final expression implemented in the code is thus
%
\begin{eqnarray}
  \label{eq:SourceT2a}
  {\mathcal S}_{\tau}  = \alpha\big[ T^{00}\left( \beta^i\beta^j K_{ij} - \beta^i
  \partial_i \alpha \right) +  T^{0i} \left( -\partial_i \alpha + 2
  \beta^j K_{ij} \right) 
+ T^{ij} K_{ij}\big]\ .
\end{eqnarray}

%% original version by Ian. It contains typos, Luca believes
%\subsubsection{Source term for S}
%
%The source terms for $S_j$ are 
%\begin{equation}
%  \label{eq:SourceS1}
%  {\cal S}_{S_j} = \alpha \sqrt{\gamma} T^{\mu\nu}
%  {}^{(4)}\Gamma_{\nu\mu j}.
%\end{equation}
%Ignoring the factor of $\alpha \sqrt{\gamma}$, these source terms are
%coded as
%\begin{eqnarray}
%  \label{eq:SourceS2a}
%  {\cal S}_{S_j} & = & T^{00}\left( \frac{1}{2} \beta^l \beta^m
%    \partial_j \gamma_{lm} - \alpha \partial_j \alpha \right) + \\
%  \label{eq:SourceS2b}
%  &  & T^{0i} \left(  \beta^l \partial_j \gamma_{il} \right) + \\
%  \label{eq:SourceS2c}
%  &  & \frac{1}{2} T^{lm} \partial_j \gamma_{lm} + \\
%  \label{eq:SourceS2d}
%  &  & \frac{\rho h W^2 v_l}{\alpha}\partial_j \beta^l . 
%\end{eqnarray}
%
%In order to get from the first expression to the expression in the
%code we need to calculate the 4-Christoffel symbols ${}^{(4)}\Gamma$
%applied to the 3+1 decomposed variables. In order to remove time
%derivatives we will frequently make use of the ADM evolution equation
%for the 3-metric in the form
%\begin{equation}
%  \label{eq:SourceADMg}
%  \partial_0 \gamma_{ij} = - 2 \left( \alpha K_{ij} + \partial_{(i}
%  \beta_{j)} - {}^{(3)}\Gamma^k_{ij} \beta_k \right),
%\end{equation}
%or equivalent forms.
%
%So, the heart of the calculation is to show that
%\begin{eqnarray}
%  \label{eq:SourceS3a}
%  {}^{(4)}\Gamma_{00j} & = & \frac{1}{2} \partial_j \left( \beta_k
%    \beta^k - \alpha^2 \right), \\
%  \label{eq:SourceS3b}
%  {}^{(4)}\Gamma_{l0j} & = & -\alpha K_{jl} - \beta_{j,l} - \beta_k
%  {}^{(3)}\Gamma^k_{lj}, \\
%  \label{eq:SourceS3c}
%  {}^{(4)}\Gamma_{0mj} & = & \alpha K_{mj} + {}^{(3)}\Gamma^k_{mj}
%  \beta_k, \\
%  \label{eq:SourceS3d}
%  {}^{(4)}\Gamma_{lmj} & = & {}^{(3)}\Gamma_{lmj}.
%\end{eqnarray}
%These are tedious but straightforward, where~(\ref{eq:SourceADMg}) was
%used in equations~(\ref{eq:SourceS3b}) and~(\ref{eq:SourceS3c}).
%
%Substituting these expressions into the original form of
%equation~(\ref{eq:SourceS1}) we find the coefficient of the $T^{00}$
%term to be
%\begin{eqnarray}
%  \label{eq:SourceS4a}
%  {}^{(4)}\Gamma_{00j} & = & \frac{1}{2} \beta^l \beta^m \partial_j
%  \gamma_{lm} - \alpha \partial_j \alpha \\
%  \label{eq:SourceS4b}
%  && + \beta_m \partial_j \beta^m.
%\end{eqnarray}
%Line~(\ref{eq:SourceS4a}) is equivalent to the
%line~(\ref{eq:SourceS2a}) in the code. 
%
%The coefficient of the $T^{0i}+T^{i0}$ term (as the stress-energy
%tensor is symmetric here) is
%\begin{eqnarray}
%  \label{eq:SourceS5a}
%  {}^{(4)}\Gamma_{0ij} + {}^{(4)}\Gamma_{i0j} & = & \beta^l \partial
%  \gamma_{jl} \\
%  \label{eq:SourceS5b}
%  && + \gamma_{il} \partial_j \beta^l.
%\end{eqnarray}
%Similarly to above, line~(\ref{eq:SourceS5a}) is equivalent to the
%line~(\ref{eq:SourceS2b}) in the code. 
%
%The coefficient of the $T^{lm}$ term is given by
%\begin{eqnarray}
%  \label{eq:SourceS6a}
%  {}^{(4)}\Gamma_{lmj} & = & {}^{(3)}\Gamma_{lmj} \\
%  \label{eq:SourceS6b}
%  & = & \frac{1}{2} \gamma_{ml,j} \\ 
%  \label{eq:SourceS6c}
%  && + \frac{1}{2} \left( \gamma_{jl,m} - \gamma_{mj,l} \right).
%\end{eqnarray}
%Again, line~(\ref{eq:SourceS6b}) is equivalent to the
%line~(\ref{eq:SourceS2b}) in the code. 
%
%In each of these sets of coefficients there is an extra line. In the
%case of the coefficient of $T^{lm}$ line~(\ref{eq:SourceS6c}) vanishes
%when contracted with $T^{lm}$ due to the symmetry of the stress-energy
%tensor and the anti-symmetry of line~(\ref{eq:SourceS6c}) with respect
%to $l,m$. However, lines~(\ref{eq:SourceS5b}) and (\ref{eq:SourceS4b})
%do not vanish. Instead, after contraction with the appropriate
%components of the stress-energy tensor, they simplify to form
%line~(\ref{eq:SourceS2d}) in the code. This gathers all partial
%derivatives of the shift in one place.
%
%\subsubsection{Source term for $\tau$}
%
%The derivation of the source term for $\tau$ is a bit more involved.
%
%The source term for $\tau$ is
%\begin{equation}
%  \label{eq:SourceT1}
%  {\cal S}_{\tau} = \alpha \sqrt{\gamma} \left( T^{\mu 0} \partial_{\mu}
%  \alpha - \alpha T^{\mu\nu} {}^{(4)}\Gamma^0_{\mu\nu}\right).
%\end{equation}
%Ignoring the factor of $\alpha \sqrt{\gamma}$, these source terms are
%coded as
%\begin{eqnarray}
%  \label{eq:SourceT2a}
%  {\cal S}_{\tau} & = & T^{00}\left( \beta^i\beta^j K_{ij} - \beta^i
%    \partial_i \alpha \right) + \\
%  \label{eq:SourceT2b}
%  &  & T^{0i} \left( -\partial_i \alpha + 2 \beta^j K_{ij} \right) + \\
%  \label{eq:SourceT2c}
%  &  &  T^{lm} K_{lm}.
%\end{eqnarray}
%
%We consider the coefficient of $T^{00}$ first. In what follows $D$ is
%the covariant derivative associated with the 3-surface. We note that
%it is compatible with the 3-metric, $D_i\gamma_{jk}=0$. Expanding the
%coefficient directly we get
%\begin{eqnarray}
%  \label{eq:SourceT3a}
%  \partial_0 \alpha - \alpha {}^{(4)}\Gamma^0_{00} & = & - \beta^i
%  \partial_i \alpha + \frac{1}{\alpha} \left( \beta^i \partial_0
%    \beta_i - \frac{1}{2} \partial_0 (\beta^k \beta_k) - \frac{1}{2}
%    \beta^i \partial_i (\beta^k \beta_k) \right) \\
%  \label{eq:SourceT3b}
%  & = & -\beta^i \partial_i \alpha + \frac{1}{2\alpha} \left( 2 \beta^i
%    \partial_0 \beta_i - \beta^k \partial_0 \beta_k - \beta_k \partial_0
%    \beta^k - \beta^i D_i (\beta^k \beta_k) \right) \\
%  \label{eq:SourceT3c}
%  & = & -\beta^i \partial_i \alpha + \frac{1}{2\alpha} \left(
%    \gamma_{kl} \beta^l \partial_0 \beta^k - \beta^k \partial_0
%    (\gamma_{kl} \beta^l) - \beta^i \left( \beta_k D_i \beta^k +
%      \beta^k D_i \beta_k \right) \right) \\
%  \label{eq:SourceT3d}
%  & = & -\beta^i \partial_i \alpha + \frac{1}{2\alpha} \left(
%    -\beta^k \beta^l \partial_0 \gamma_{kl} - \beta^i \left( \beta^l
%      D_i \beta_l + \beta^k D_i \beta_k + \beta_k \beta_l D_i
%      \gamma^{kl} \right) \right) \\
%  \label{eq:SourceT3e}
%  & = & -\beta^i \partial_i \alpha + \frac{1}{2\alpha} \left(
%    2 \alpha \beta^k \beta^l K_{kl} - \beta^k \beta^l (D_k \beta_l +
%    D_l \beta_k) + 2 \beta^i \beta^k D_i \beta_k \right) \\
%  \label{eq:SourceT3f}
%  & = & -\beta^i \partial_i \alpha + \beta^k \beta^l K_{kl}.
%\end{eqnarray}
%Again to go from line~(\ref{eq:SourceT3d}) to (\ref{eq:SourceT3e}) we
%used the evolution equation~(\ref{eq:SourceADMg}), expressed in terms
%of the covariant derivative $D$.
%
%To simplify the calculation of the other coefficients we first
%calculate the 4-Christoffel symbol ${}^{(4)}\Gamma^0_{i\nu}$. This is
%given by
%\begin{eqnarray}
%  \label{eq:SourceChr1a}
%  {}^{(4)}\Gamma^0_{i\nu} & = & - \frac{1}{2\alpha^2} \left\{ \left(
%      \partial_i (\beta^k \beta_k - \alpha^2) - \beta^j (\partial_i
%      \beta_j - \partial_j \beta_i + \partial_0 \gamma_{ij}) \right)
%    \delta^0_{\nu} + \right. \\
%    && \left. \left( (\partial_l \beta_i + \partial_i \beta_l -
%      \partial_0 \gamma_{il}) - 2 \beta^j {}^{(3)}\Gamma_{jil} \right)
%    \delta^l_{\nu} \right\} \\
%  \label{eq:SourceChr1b}
%  & = & - \frac{1}{2\alpha^2} \left\{ \left(
%      \partial_i (\beta^k \beta_k - \alpha^2) - \beta^j ( 2 \partial_i
%      \beta_j - 2 \alpha K_{ij} - 2 \beta_k {}^{(3)}\Gamma^k_{ij})
%      \right) \delta^0_{\nu} + \right. \\
%    && \left. \left( (2 \alpha K_{il} + 2 \beta_k
%      {}^{(3)}\Gamma^k_{il}) - 2 \beta^j {}^{(3)}\Gamma_{jil} \right)
%      \delta^l_{\nu} \right\}.
%\end{eqnarray}
%
%The coefficient of $T^{0i} + T^{i0}$ is expanded in a similar
%fashion. Here we note that we only pick up one partial derivative of
%the lapse (as there is a specific ordering on that term) but that
%symmetry gives us two Christoffel symbols,
%\begin{eqnarray}
%  \label{eq:SourceT4a}
%  \partial_i \alpha - 2 \alpha {}^{(4)}\Gamma^0_{i0} & = & \partial_i
%  \alpha + \frac{1}{\alpha} \left( (\beta^k \beta_k - \alpha^2)_{,i} -
%    2 \beta^j (\partial_i \beta_j - \beta_k {}^{(3)}\Gamma^k_{ij} -
%    \alpha K_{ij}) \right) \\
%  \label{eq:SourceT4b}
%  & = & 2 \beta^j K_{ij} - \partial_i \alpha + \frac{1}{\alpha} \left(
%    \partial_i (\beta^k \beta_k) - 2 \beta^j (\partial_i \beta_j -
%    \beta_k {}^{(3)}\Gamma^k_{ij} ) \right). 
%\end{eqnarray}
%We then have to show that the term in brackets vanishes identically.
%Just expanding gives
%\begin{eqnarray}
%  \label{eq:SourceT4c}
%  \left(\dots\right) & = & 2 \beta^k \partial_i \beta_k + \beta^k
%  \beta^l \partial_i \gamma_{kl} - 2 \beta^j \left( \gamma_{jk}
%    \partial_i \beta^k + \beta^k \partial_i \gamma_{jk} - \beta_k
%    \gamma^{kl} (\partial_i \gamma_{lj} + \partial_j \gamma_{il} -
%    \partial_l \gamma_{ij}) \right) \\
%  \label{eq:SourceT4d}
%  & = & \beta^j \beta^l (\partial_j \gamma_{il} - \partial_l
%  \gamma_{ij} )
%\end{eqnarray}
%which is zero by symmetry.
%
%Finally we have the $T^{lm}$ coefficient. Again a simple expansion
%gives
%\begin{eqnarray}
%  \label{eq:SourceT5a}
%  -\alpha {}^{(4)}\Gamma^0_{lm} & = & \frac{1}{2\alpha} \left( 2
%   \alpha K_{lm} + 2 \beta_k {}^{(3)}\Gamma^k_{lm} - 2 \beta^j
%   {}^{(3)} \Gamma_{jlm} \right) \\
%  \label{eq:SourceT5b}
%  & = & K_{lm}.
%\end{eqnarray}
%Once again we had to make use of equation~(\ref{eq:SourceADMg}).


\subsection{Conversion from conservative to primitive variables}
\label{sec:con2prim}

As noted in section~\ref{sec:phys} the variables that are evolved are
the conserved variables $D, S_i, \tau$. But in order to calculate the
fluxes and sources we require the primitive variables $\rho, v_i, P$.
Conversion from primitive to conservative is given analytically by
equation~(\ref{eq:prim2con}). Converting in the other direction is not
possible in a closed form except in certain special circumstances.

There are a number of methods for converting from conservative to
primitive variables; see~\cite{livrevsrrfd}. Here we use a
Newton-Raphson type iteration. If we are using a general equation of
state such as an ideal gas, then we find a root of the pressure
equation. Given an initial guess for the pressure $\bar{P}$ we find
the root of the function
\begin{equation}
  \label{eq:pressure1}
  f = \bar{P} - P(\bar{\rho}, \bar{\epsilon}),
\end{equation}
where the approximate density and specific internal energy are given
by 
\begin{eqnarray}
  \label{eq:press1gives}
  \bar{\rho} & = & \frac{\tilde{D}}{\tilde{\tau} + \bar{P} + \tilde{D}}
  \sqrt{ (\tilde{\tau} + \bar{P} + \tilde{D})^2 - S^2 }, \\
  \bar{W} & = & \frac{\tilde{\tau} + \bar{P} + \tilde{D}}{\sqrt{
      (\tilde{\tau} + \bar{P} + \tilde{D})^2 - S^2 }}, \\
  \bar{\epsilon} & = & \tilde{D}^{-1} \left( \sqrt{ (\tilde{\tau} +
      \bar{P} + \tilde{D})^2 - S^2 } - \bar{P} \bar{W} - \tilde{D}
  \right). 
\end{eqnarray}
Here the conserved variables are all ``undensitized'', {\it e.g.},
\begin{equation}
  \label{eq:undens}
  \tilde{D} = \gamma^{-1/2} D,
\end{equation}
where $\gamma$ is the determinant of the 3-metric, and $S^2$ is given
by 
\begin{equation}
  \label{eq:s2}
  S^2 = \gamma^{ij}\tilde{S}_i\tilde{S}_j.
\end{equation}

In order to perform a Newton-Raphson iteration we need the derivative
of the function with respect to the dependent variable, here the
pressure. This is given by
\begin{equation}
  \label{eq:df}
  f' = 1 - \frac{\partial P}{\partial \rho}\frac{\partial
  \rho}{\partial P} - \frac{\partial P}{\partial
  \epsilon}\frac{\partial \epsilon}{\partial P}, 
\end{equation}
where $\frac{\partial P}{\partial \rho}$ and $\frac{\partial
  P}{\partial \epsilon}$ given by calls to {\tt EOS\_Base}, and
\begin{eqnarray}
  \label{eq:df2}
  \frac{\partial \rho}{\partial P} & = & \frac{\tilde{D}
      S^2}{\sqrt{(\tilde{\tau} + \bar{P} + \tilde{D})^2 -
      S^2}(\tilde{\tau} + \bar{P} + \tilde{D})^2}, \\
  \frac{\partial \epsilon}{\partial P} & = & \frac{\bar{P}
      S^2}{\rho\left((\tilde{\tau} + \bar{P} + \tilde{D})^2 -
      S^2\right)(\tilde{\tau} + \bar{P} + \tilde{D})}. \\
\end{eqnarray}

For a polytropic type equation of state, the function is given by
\begin{equation}
  \label{eq:polyf}
  f = \bar{\rho}\bar{W} - \tilde{D},
\end{equation}
where $\bar{\rho}$ is the variable solved for, the pressure, specific
internal energy and enthalpy $\bar{h}$ are set from the EOS and the
Lorentz factor is found from
\begin{equation}
  \label{eq:polyw}
  \bar{W} = \sqrt{1 + \frac{S^2}{(\tilde{D}\bar{h})^2}}.
\end{equation}
The derivative is given by
\begin{equation}
  \label{eq:dpolyf}
  f' = \bar{W} - \frac{\bar{\rho}S^2 \bar{h}'}{\bar{W} \tilde{D}^2
  \bar{h}^3}, 
\end{equation}
where
\begin{equation}
  \label{eq:dpolyenth}
  \bar{h}' = \bar{\rho}^{-1}\frac{\partial P}{\partial \rho}.
\end{equation}

\subsection{A note on the Roe and Marquina Riemann Solvers}
\label{sec:rsnote}

Finding the Roe or Marquina fluxes as given is
section~\ref{sec:riemann} requires the left eigenvectors to either be
supplied analytically or calculated numerically. 

When this is done by inverting the matrix of right
eigenvectors, in the actual code this is combined with the calculation
of, {\it e.g.}, the characteristic jumps $\Delta {\bf w}$.
Normally the eigenvalues and vectors are ordered lexicographically.
However for the polytropic equation of state one of the equations is
redundant, so the matrix formed by these eigenvectors is linearly
dependent and hence singular. It turns out that this is only a minor
problem; by rearranging the order of the eigenvalues and vectors it is
possible to numerically invert the matrix. 
This means that no specific ordering of the eigenvalues should be
assumed. It also explains the slightly strange ordering in the
routines {\tt GRHydro\_EigenProblem*.F90}.

The current default is that the left eigenvectors are calculated
analytically - for the expressions see Font~\cite{livrevgrrfd}. For
both the Roe and the Marquina solvers an optimized version of the flux
calculation has been implemented. For more details on the analytical
form and the optimized flux calculation see Ib{\'a}{\~n}ez et
al.~\cite{Iban01}, Aloy et al.~\cite{Aloy99} and Frieben et
al.~\cite{Frie02}.

\subsection{The atmosphere}
\label{sec:atmosphere}

In simulations of compact objects, often the matter is located only on a (small) portion of the
numerical grid. In fact, over much of the evolved domain the physical situation is likely to be
sufficiently well approximated by vacuum. However, in the vacuum limit the continuity equations
describing the fluid break down. The speed of sound tends to the speed of light and everything fails
(especially the conversion from conserved to primitive variables).

To avoid this problem it is customary to introduce an atmosphere. In our implementation, this is a
low-density region surrounding the compact objects and initially it has no velocity and is in equilibrium. The
introduction of an atmosphere is managed by the initial data thorns.

However {\tt GRHydro} itself also knows about the atmosphere, of course. If the conserved variables
$D$ or $\tau$ are beneath some minimum value, or an evolution step might push them below such a
value, then the relevant cell is not evolved. Also, if the density should fall below a minimum value
in the routine that converts from conservative to primitive variables, all the variables are reset
to the values adopted for the atmosphere.

Probably the hardest part of using {\tt GRHydro} is to correctly set these
atmosphere values. In the current implementation the atmosphere is
used in three separate places. These are
\begin{enumerate}
\item {\bf Set up of the initial data}. Initial-data routines should set an atmosphere consistent
  with the one that will be evolved.
\item {\bf In the routine that converts from conserved variables to
    primitive variables}. This is where the majority of the atmosphere
  resets will occur. 

  If the equation of state is polytropic then an
  attempt is made to convert to primitive variables. If the iterative
  algorithm returns a negative (and hence unphysical) value of $\rho$,
  then $\rho$ is reset to the atmosphere value, the velocities are set
  to zero, and $P$, $\epsilon$, $S_i$ and $\tau$ are reset to be
  consistent with $\rho$ (and $D$). Note that even though the
  polytropic equation of state gives us sufficient information to
  calculate a consistent value of $D$, this is not done.

  If the equation of state is the more general type (such as that of an ideal fluid) and if $\rho$
  is less than the specified minimum, then, as above, we assume we are in the atmosphere and call
  the routine that changes from the conserved to the primitive variables for the polytrope.

\item {\bf When applying the update}. If the calculated update terms
  for a specific cell would lead to either $D$ or $\tau$ becoming
  negative, then two steps are taken. First, we do not update this
  specific cell. Second, the data in this cell is reset to be the
  atmosphere. 
\end{enumerate}

The reason why the routine that converts to the primitive variables
does not ensure that $D$ is consistent with the other variables is
practical rather than accurate. If the value of the variables is set
such that they all lie precisely on the atmosphere, then small errors
(typically initially of the order of $10^{-25}$ for a $64^3$-point TOV star
in octant symmetry) would move certain cells above the atmosphere
values. Combined with the necessary atmosphere treatment this leads to
high-frequency noise. This will lead to waves of matter falling onto
the star. Despite their extremely low density (typically only an
order of magnitude higher than the floor) they will result in visible
secondary overtones in the oscillations of, {\it e.g.}, the central
density. 

The parameters controlling the atmosphere are the following.
\begin{itemize}
\item {\tt GRHydro::rho\_abs\_min}. An absolute value for $\rho$ in the
  atmosphere. Defaults to -1. Any negative value will be ignored, and
  the value of {\tt rho\_rel\_min} used instead.
\item {\tt GRHydro::rho\_rel\_min}. A relative value for $\rho$ in the
  atmosphere. Defaults to $10^{-7}$. Only used if {\tt rho\_abs\_min}
  is negative, which is the default behaviour. The actual value of the
  atmosphere will be $\rho =${\tt rho\_rel\_min}$\times${\tt
    GRHydro\_rho\_max}, where {\tt GRHydro::GRHydro\_rho\_max}
  is a variable containing the maximum value of $\rho$ on the numerical grid at time zero.
\item {\tt initial\_rho\_abs\_min}. An absolute value for rho in the initial atmosphere. It is used
  only by initial data routines. Unused if negative.
\item {\tt initial\_rho\_rel\_min}. A relative (to the initial maximum rest-mass density) value for rho
  in the atmosphere. It is used only by initial data routines and only if it is positive and {\tt
  initial\_rho\_abs\_min} is negative.
\item {\tt initial\_atmosphere\_factor}. A relative (to the initial atmosphere) value for rho in the
  atmosphere. It is used only by initial data routines. It multiplies the atmosphere value used by
  the initial data solver. Unused if negative.
\item {\tt GRHydro\_atmo\_tolerance}. A parameter useful mostly in mesh-refined simulations. A point
  is set to the atmosphere values in the conservative to primitive routines if its rest-mass density
  is such that $\rho <$ {\tt GRHydro\_rho\_min}$*(1+${\tt GRHydro\_atmo\_tolerance}). This avoids
  occasional spurious oscillations in ({\tt Carpet}) buffer zones lying in the atmosphere (because
  prolongation happens on conserved variables).
\end{itemize}

The motivation for these parameters referring only to the initial data is that it is sometimes best
to set the initial atmosphere to slightly below the atmosphere cutoff used during evolution, as this
avoids truncation error and metric evolution leading to low density waves travelling across the
atmosphere.

The routines essential to the atmosphere are contained in {\tt
GRHydro\_Minima.F90, GRHydro\_Con2Prim.F90, GRHydro\_UpdateMask.F90}. 

\subsection{Advection of passive scalars ('tracers')}
\label{sec:tracer}

For some astrophysical problems it is necessary to advect passive
compositional scalars such as the electron fraction $Y_e$ (number of
electrons per baryon). For a generic tracer $X_k$, the evolution
equation looks like

%%%\begin{eqnarray}
%%%  \label{eq:tracer}
%%%   D &=& \sqrt{\gamma}W\rho \nonumber \\
%%%   S^i &=& \sqrt{\gamma} \rho h W^2 v^i \nonumber \\
%%%   \tau &=& \sqrt{\gamma}\left( \rho h W^2 - p\right) - D, 
%%%\end{eqnarray}

\begin{equation}
  \label{eq:tracer}
  \partial_t { ( D X_k )} + \partial_{x^j} {\bf f}^{(j)} ({D X_k}) = 0\, ,
\end{equation}
where $D$ is the generalized particle number density as defined in
Eq.~(\ref{eq:prim2con}). {\tt GRHydro} currently supports any number of
independent tracer variables. The following parameters have to be
set to use the tracers:

\begin{itemize}
  \item {\tt GRHydro::evolve\_tracer}. Boolean. Set to {\tt yes} if
    you want the tracers to be active.
  \item {\tt GRHydro::number\_of\_tracers}. Integer. Defaults to 0. To
    use tracers, set to at least 1.
\end{itemize}

Note, that your initial data thorn must set initial data for
{\tt GRHydro::tracer[k]} and {\tt GRHydro::cons\_tracer[k]} for all tracers
you want to advect. {\tt GRHydro::cons\_tracer[k]} stores $D X_k$.

\subsubsection{Implementation and Limitations}

\begin{itemize}
  \item Reconstruction: Currently only TVD and PPM reconstruction of the
    tracers $X_k$ are implemented. Since for most astrophysical problems one
    will associate the tracer with some compositional variable it might
    be better to reconstruct $\rho X_k$.

  \item Riemann Solvers: Only HLLE and Marquina are supported. In HLLE,
    the fluxes as given in Eq.~(\ref{eq:tracer}) are computed for each tracer.
    In the Marquina solver, we multiply the $D$-flux by each tracer to get
    the individual tracer fluxes according to the following prescription:

    \begin{equation}
      \label{eq:marquinatracer}
      \begin{array}[l]{l}
	{\bf If}\ \, F_{i+1/2} (D) > 0 \,\, {\bf then} \\
	  \qquad \qquad \begin{array}[c]{l}
	    \qquad \qquad F_{i+1/2} (D X_k) = X_{k_i}^+ F_{i+1/2} (D)
	  \end{array} \\
		 {\bf else} \\
		 \qquad \qquad \begin{array}[c]{r c l}
		   \qquad \qquad F_{i+1/2} (D X_k) = X_{k_{i+1}}^- F_{i+1/2} (D)
		 \end{array} \\
			{\bf end if} \\
      \end{array} \\
    \end{equation}

    The above was suggested by Miguel Aloy and first implemented and
    tested by Harry Dimmelmeier in his code (CoCoNuT), then by Christian~D.~Ott
    in {\tt GRHydro}.

\end{itemize}


\section{History}

The approximate time line is something like this:
\begin{itemize}
\item ~1995: Valencia group hydrodynamics code, fixed spacetime.
\item ~1997: Ported to Cactus, extensively rewritten for the Binary
  Neutron Star Grand Challenge. Primarily written by Mark Miller.
  Released as {\tt GR3D} as public domain code.
\item ~1998-: Developed as Cactus thorn {\tt MAHC} inside the
  GR{\tt Astro\_Hydro} arrangement at Washington University, primarily by
  Mark Miller.
\item 2002-2008: {\tt Whisky} written based on {\tt GR3D}.
\item 2008-: {\tt GRHydro} based on the public version of {\tt Whisky}
\end{itemize}

This is necessarily only a sketch; many people have contributed to
the history of this code, and the present authors were not around for most of it...

\subsection{Thorn Source Code}

This was initially written by Luca Baiotti, Ian Hawke and Pedro
Montero with considerable assistance from Luciano Rezzolla, Toni Font,
Nick Stergioulas and Ed Seidel. This led to the basic {\tt GRHydro} thorns,
{\tt GRHydro} itself, {\tt GRHydro\_Init\_Data} and {\tt GRHydro\_RNSID}.

Since then most of the maintenance has been done by Ian Hawke, Luca Baiotti and Frank L\"offler. Various
people have contributed to the development. In particular
\begin{itemize}
\item The PPM reconstruction methods were written by Ian Hawke heavily
  based on the code of Toni Font. They were later expanded by Christian D. Ott and Luca Baiotti.
\item The Roe and Marquina solvers were made considerably more
  efficient thanks to Joachim Frieben.
\item The current atmosphere algorithm is a mixture of ideas from the
  {\tt GR3D} code, Luciano Rezzolla, Toni Font and Nick
  Stergioulas. The current setup was written by Ian Hawke and Luca Baiotti.
\item The 1-dimensional TOV solver {\tt GRHydro\_TOVSolver} was written
  by Ian Hawke based on a short paper by Thomas Baumgarte.
%\item The use of the equation of state, in particular the routines to
%  ensure the initial hydrodynamic consistency, were due to the work of
%  Harald Dimmelmeier and Christian Ott.
\end{itemize}

\subsection{Thorn Documentation}

This documentation was first written largely by Ian Hawke and Scott Hawley in 2002. Long due,
rather necessary and considerably large updates were made in 2008 by Luca Baiotti.


\subsection{Acknowledgements}

As already mentioned, the history behind this code leads to a long list
of people to be acknowledged.

Firstly, without the work of the Valencia group this sort of code
would be impossible. 

Secondly, the incomparable work of Mark Miller and the Washington
University - AEI Collaboration in producing the {\tt GR3D} and {\tt
GRAstro\_Hydro} codes gave an essential benchmark to aim for, and
encouragement that it was possible!

Thirdly, the support of the Cactus team, especially Tom Goodale,
Gabrielle Allen and Thomas Radke made life a
lot easier.

Finally, for their work in coding, ideas and suggestions, or just
plain encouragement, we would like to thank all at the AEI and in the
EU Network, especially Toni Font, Luciano Rezzolla, Nick Stergioulas,
Ed Seidel, Carsten Gundlach and Jos\'e-Maria Ib{\'a}{\~n}ez.

Originally Ed Seidel and then Luciano Rezzolla and Gabrielle Allen and many others 
have been granting (in addition to valuable
scientific advice) financial support and human resources to the development of the code.

\begin{thebibliography}{20}

\bibitem{Aloy99b}
Aloy M.A., Ib{\'a}nez J.M., Mart\'\i J.M., M{\"u}ller E.
\newblock {\em Astroph. J. Supp.\/}, {\bf 122}: 151 (1999).

\bibitem{Aloy99} 
M.~A. Aloy, J.~A. Pons, and J.~M. Ib{\'a}{\~n}ez.
\newblock An efficient implementation of flux formulae in multidimensional
  relativistic hydrodynamical codes.
\newblock {\em Comput. Phys. Commun.}, {\bf 120}:\penalty0 115--121, 1999.

\bibitem{Banyuls97}
Banyuls F., Font J.A., Ib{\'a}nez J.M., Mart\'{\i} J.M., Miralles J.A.
\newblock {\em Astrophys. J.\/}, {\bf 476}: 221 (1997).

\bibitem{ppm}
P. Colella and P.~R. Woodward.
\newblock The {P}iecewise {P}arabolic {M}ethod ({PPM}) for
{G}as-{D}ynamical {S}imulations.
\newblock {\em J. Comput. Phys.}, {\bf 54}, 174--201, 1984.

\bibitem{Cook00}
G. Cook
\newblock Initial Data for Numerical Relativity
\newblock {\em Living Rev. Relativity}, {\bf 3}, 2000.
\newblock [Article in on-line journal], cited on 31/08/02,
  http://www.livingreviews.org/ Articles/Volume3/2000-5cook/index.html.

\bibitem{Marquina1}
R.~Donat and A.~Marquina.
\newblock Capturing shock reflections: An improved flux formula.
\newblock {\em J. Comput. Phys.}, {\bf 125}:\penalty0 42--58, 1996.

\bibitem{Einfeldt88}
Einfeldt B.
\newblock {\em Journal of Computational Physics\/}, {\bf 25}: 294 (1988).

\bibitem{Eulderink94}
Eulderink F., Mellema G.
\newblock {\em Astron. Astrophys.\/}, {\bf 284}: 652 (1994).

\bibitem{livrevgrrfd}
J.~A. Font.
\newblock Numerical hydrodynamics in {G}eneral {R}elativity.
\newblock {\em Living Rev. Relativity}, {\bf 3}, 2000.
\newblock [Article in on-line journal], cited on 31/07/01,
  http://www.livingreviews.org/ Articles/Volume3/2000-2font/index.html.

\bibitem{Frie02}
J. Frieben, J.~M. Ib{\'a}{\~n}ez, and J. Pons.
\newblock {\em in preparation}

\bibitem{eno}
A.~Harten, B.~Engquist, S.~Osher, and S.~R. Chakravarthy.
\newblock Uniformly high order accurate essentially non-oscillatory schemes,
  {III}.
\newblock {\em J. Comput. Phys.}, {\bf 71}:\penalty0 231--303, 1987.

\bibitem{Iban01}
J.~M. Ib{\'a}{\~n}ez et al.
\newblock in {\em Godunov Methods: Theory and Applications}.
\newblock New York, 485--503, (2001)

\bibitem{Tadmor00}
A. Kurganov and E. Tadmor.
\newblock {\em New high-resolution central schemes for nonlinear conservation laws and
  convection-diffusion equations}.
\newblock {\em J. Comput. Phys.}, {\bf 160}:241, 2000.

\bibitem{laney}
C.~B. Laney.
\newblock {\em Computational {G}asdynamics}.
\newblock Cambridge University Press, 1998.

\bibitem{leveque}
R.~J. LeVeque.
\newblock {N}onlinear conservation laws and finite volume methods for
  astrophysical fluid flow.
\newblock In O.~Steiner and A.~Gautschy, editors, {\em {C}omputational
  {M}ethods for {A}strophysical {F}luid {F}low}. Springer-Verlag, 1998.

\bibitem{livrevsrrfd}
J.~M. Mart{\'{\i}} and E.~M{\"u}ller.
\newblock Numerical hydrodynamics in {S}pecial {R}elativity.
\newblock {\em Living Rev. Relativity}, {\bf 2}, 1999.
\newblock [Article in on-line journal], cited on 31/7/01,
  http://www.livingreviews.org/Articles/Volume2/1999-3marti/index.html.

\bibitem{Quirk}
J.~J. Quirk.
\newblock A contribution to the great {R}iemann solver debate.
\newblock {\em Int. J. Numer. Methods Fluids}, {\bf 18}:\penalty0 555--574,
  1994.

\bibitem{Roe81}
Roe P.L.
\newblock {\em J. Comput. Phy.\/}, {\bf 43}: 357 (1981).

\bibitem{shueno}
C. Shu.
\newblock {H}igh {O}rder {ENO} and {WENO} {S}chemes for
{C}omputational {F}luid {D}ynamics.
\newblock In T.~J. Barth and H. Deconinck, editors {\em High-Order
  Methods for Computational Physics}. Springer, 1999.
\newblock A related on-line version can be found under {\em Essentially
  {N}on-{O}scillatory and {W}eighted {E}ssentially {N}on-{O}scillatory
  {S}chemes for {H}yperbolic {C}onservation {L}aws} at {\tt
  http://www.icase.edu/library/reports/rdp/97/97-65RDP.tex.refer.html}. 

\bibitem{toro}
E.~F. Toro.
\newblock {\em {R}iemann {S}olvers and {N}umerical {M}ethods for {F}luid
  {D}ynamics}.
\newblock {S}pringer-{V}erlag, 2nd edition, 1999.

\bibitem{Moesta:2013dna}
  P.~Mösta {\it et al.},
  `GRHydro: A new open source general-relativistic magnetohydrodynamics code for the Einstein Toolkit,''
  Class.\ Quant.\ Grav.\  {\bf 31}, 015005 (2014)
  doi:10.1088/0264-9381/31/1/015005
  [arXiv:1304.5544 [gr-qc]].

\end{thebibliography}

% Do not delete next line
% END CACTUS THORNGUIDE



\section{Parameters} 


\parskip = 0pt

\setlength{\tableWidth}{160mm}

\setlength{\paraWidth}{\tableWidth}
\setlength{\descWidth}{\tableWidth}
\settowidth{\maxVarWidth}{max\_magnetic\_to\_gas\_pressure\_ratio}

\addtolength{\paraWidth}{-\maxVarWidth}
\addtolength{\paraWidth}{-\columnsep}
\addtolength{\paraWidth}{-\columnsep}
\addtolength{\paraWidth}{-\columnsep}

\addtolength{\descWidth}{-\columnsep}
\addtolength{\descWidth}{-\columnsep}
\addtolength{\descWidth}{-\columnsep}
\noindent \begin{tabular*}{\tableWidth}{|c|l@{\extracolsep{\fill}}r|}
\hline
\multicolumn{1}{|p{\maxVarWidth}}{constrain\_to\_1d} & {\bf Scope:} private & BOOLEAN \\\hline
\multicolumn{3}{|p{\descWidth}|}{{\bf Description:}   {\em Set fluid velocities to zero for non-radial motion}} \\
\hline & & {\bf Default:} no \\\hline
\end{tabular*}

\vspace{0.5cm}\noindent \begin{tabular*}{\tableWidth}{|c|l@{\extracolsep{\fill}}r|}
\hline
\multicolumn{1}{|p{\maxVarWidth}}{use\_cxx\_code} & {\bf Scope:} private & BOOLEAN \\\hline
\multicolumn{3}{|p{\descWidth}|}{{\bf Description:}   {\em Use experimental C++ code?}} \\
\hline & & {\bf Default:} yes \\\hline
\end{tabular*}

\vspace{0.5cm}\noindent \begin{tabular*}{\tableWidth}{|c|l@{\extracolsep{\fill}}r|}
\hline
\multicolumn{1}{|p{\maxVarWidth}}{verbose} & {\bf Scope:} private & BOOLEAN \\\hline
\multicolumn{3}{|p{\descWidth}|}{{\bf Description:}   {\em Some debug output}} \\
\hline & & {\bf Default:} no \\\hline
\end{tabular*}

\vspace{0.5cm}\noindent \begin{tabular*}{\tableWidth}{|c|l@{\extracolsep{\fill}}r|}
\hline
\multicolumn{1}{|p{\maxVarWidth}}{apply\_h\_viscosity} & {\bf Scope:} restricted & BOOLEAN \\\hline
\multicolumn{3}{|p{\descWidth}|}{{\bf Description:}   {\em H viscosity is useful to fix the carbuncle instability seen at strong shocks}} \\
\hline & & {\bf Default:} no \\\hline
\end{tabular*}

\vspace{0.5cm}\noindent \begin{tabular*}{\tableWidth}{|c|l@{\extracolsep{\fill}}r|}
\hline
\multicolumn{1}{|p{\maxVarWidth}}{atmo\_falloff\_power} & {\bf Scope:} restricted & REAL \\\hline
\multicolumn{3}{|p{\descWidth}|}{{\bf Description:}   {\em The power at which the atmosphere level falls off as (atmo\_falloof\_radius/r)**atmo\_falloff\_power}} \\
\hline{\bf Range} & &  {\bf Default:} 0.0 \\\multicolumn{1}{|p{\maxVarWidth}|}{\centering 0:*} & \multicolumn{2}{p{\paraWidth}|}{Anything positive} \\\hline
\end{tabular*}

\vspace{0.5cm}\noindent \begin{tabular*}{\tableWidth}{|c|l@{\extracolsep{\fill}}r|}
\hline
\multicolumn{1}{|p{\maxVarWidth}}{atmo\_falloff\_radius} & {\bf Scope:} restricted & REAL \\\hline
\multicolumn{3}{|p{\descWidth}|}{{\bf Description:}   {\em The radius for which the atmosphere starts to falloff as (atmo\_falloff\_radius/r)**atmo\_falloff\_power}} \\
\hline{\bf Range} & &  {\bf Default:} 50.0 \\\multicolumn{1}{|p{\maxVarWidth}|}{\centering 0:*} & \multicolumn{2}{p{\paraWidth}|}{Anything positive} \\\hline
\end{tabular*}

\vspace{0.5cm}\noindent \begin{tabular*}{\tableWidth}{|c|l@{\extracolsep{\fill}}r|}
\hline
\multicolumn{1}{|p{\maxVarWidth}}{atmo\_tolerance\_power} & {\bf Scope:} restricted & REAL \\\hline
\multicolumn{3}{|p{\descWidth}|}{{\bf Description:}   {\em The power at which the atmosphere tolerance increases as (r/atmo\_tolerance\_radius)**atmo\_tolerance\_power}} \\
\hline{\bf Range} & &  {\bf Default:} 0.0 \\\multicolumn{1}{|p{\maxVarWidth}|}{\centering 0:*} & \multicolumn{2}{p{\paraWidth}|}{Anything positive} \\\hline
\end{tabular*}

\vspace{0.5cm}\noindent \begin{tabular*}{\tableWidth}{|c|l@{\extracolsep{\fill}}r|}
\hline
\multicolumn{1}{|p{\maxVarWidth}}{atmo\_tolerance\_radius} & {\bf Scope:} restricted & REAL \\\hline
\multicolumn{3}{|p{\descWidth}|}{{\bf Description:}   {\em The radius for which the atmosphere tolerance starts to increase as (r/atmo\_tolerance\_radius)**atmo\_tolerance\_power}} \\
\hline{\bf Range} & &  {\bf Default:} 50.0 \\\multicolumn{1}{|p{\maxVarWidth}|}{\centering 0:*} & \multicolumn{2}{p{\paraWidth}|}{Anything positive} \\\hline
\end{tabular*}

\vspace{0.5cm}\noindent \begin{tabular*}{\tableWidth}{|c|l@{\extracolsep{\fill}}r|}
\hline
\multicolumn{1}{|p{\maxVarWidth}}{avec\_gauge} & {\bf Scope:} restricted & KEYWORD \\\hline
\multicolumn{3}{|p{\descWidth}|}{{\bf Description:}   {\em Which gauge condition to use when evolving the vector potential}} \\
\hline{\bf Range} & &  {\bf Default:} lorenz \\\multicolumn{1}{|p{\maxVarWidth}|}{\centering algebraic} & \multicolumn{2}{p{\paraWidth}|}{Algebraic gauge} \\\multicolumn{1}{|p{\maxVarWidth}|}{\centering lorenz} & \multicolumn{2}{p{\paraWidth}|}{Lorenz gauge} \\\hline
\end{tabular*}

\vspace{0.5cm}\noindent \begin{tabular*}{\tableWidth}{|c|l@{\extracolsep{\fill}}r|}
\hline
\multicolumn{1}{|p{\maxVarWidth}}{bound} & {\bf Scope:} restricted & KEYWORD \\\hline
\multicolumn{3}{|p{\descWidth}|}{{\bf Description:}   {\em Which boundary condition to use - FIXME}} \\
\hline{\bf Range} & &  {\bf Default:} none \\\multicolumn{1}{|p{\maxVarWidth}|}{\centering flat} & \multicolumn{2}{p{\paraWidth}|}{Zero order extrapolation} \\\multicolumn{1}{|p{\maxVarWidth}|}{\centering none} & \multicolumn{2}{p{\paraWidth}|}{None} \\\multicolumn{1}{|p{\maxVarWidth}|}{\centering static} & \multicolumn{2}{p{\paraWidth}|}{Static, no longer supported} \\\multicolumn{1}{|p{\maxVarWidth}|}{\centering scalar} & \multicolumn{2}{p{\paraWidth}|}{Constant} \\\hline
\end{tabular*}

\vspace{0.5cm}\noindent \begin{tabular*}{\tableWidth}{|c|l@{\extracolsep{\fill}}r|}
\hline
\multicolumn{1}{|p{\maxVarWidth}}{c2p\_reset\_pressure} & {\bf Scope:} restricted & BOOLEAN \\\hline
\multicolumn{3}{|p{\descWidth}|}{{\bf Description:}   {\em If the pressure guess is unphysical should we recompute it?}} \\
\hline & & {\bf Default:} no \\\hline
\end{tabular*}

\vspace{0.5cm}\noindent \begin{tabular*}{\tableWidth}{|c|l@{\extracolsep{\fill}}r|}
\hline
\multicolumn{1}{|p{\maxVarWidth}}{c2p\_reset\_pressure\_to\_value} & {\bf Scope:} restricted & REAL \\\hline
\multicolumn{3}{|p{\descWidth}|}{{\bf Description:}   {\em The value to which the pressure is reset to when a failure occurrs in C2P}} \\
\hline{\bf Range} & &  {\bf Default:} 1.e-20 \\\multicolumn{1}{|p{\maxVarWidth}|}{\centering 0:} & \multicolumn{2}{p{\paraWidth}|}{greater than zero} \\\hline
\end{tabular*}

\vspace{0.5cm}\noindent \begin{tabular*}{\tableWidth}{|c|l@{\extracolsep{\fill}}r|}
\hline
\multicolumn{1}{|p{\maxVarWidth}}{c2p\_resort\_to\_bisection} & {\bf Scope:} restricted & BOOLEAN \\\hline
\multicolumn{3}{|p{\descWidth}|}{{\bf Description:}   {\em If the pressure guess is unphysical, should we try with bisection (slower, but more robust)}} \\
\hline & & {\bf Default:} no \\\hline
\end{tabular*}

\vspace{0.5cm}\noindent \begin{tabular*}{\tableWidth}{|c|l@{\extracolsep{\fill}}r|}
\hline
\multicolumn{1}{|p{\maxVarWidth}}{calculate\_bcom} & {\bf Scope:} restricted & BOOLEAN \\\hline
\multicolumn{3}{|p{\descWidth}|}{{\bf Description:}   {\em Calculate the comoving contravariant magnetic 4-vector b\^a?}} \\
\hline & & {\bf Default:} no \\\hline
\end{tabular*}

\vspace{0.5cm}\noindent \begin{tabular*}{\tableWidth}{|c|l@{\extracolsep{\fill}}r|}
\hline
\multicolumn{1}{|p{\maxVarWidth}}{check\_for\_trivial\_rp} & {\bf Scope:} restricted & BOOLEAN \\\hline
\multicolumn{3}{|p{\descWidth}|}{{\bf Description:}   {\em Do check for trivial Riemann Problem}} \\
\hline & & {\bf Default:} yes \\\hline
\end{tabular*}

\vspace{0.5cm}\noindent \begin{tabular*}{\tableWidth}{|c|l@{\extracolsep{\fill}}r|}
\hline
\multicolumn{1}{|p{\maxVarWidth}}{check\_rho\_minimum} & {\bf Scope:} restricted & BOOLEAN \\\hline
\multicolumn{3}{|p{\descWidth}|}{{\bf Description:}   {\em Should a check on rho {\textless} GRHydro\_rho\_min be performed and written as WARNING level 2?}} \\
\hline & & {\bf Default:} no \\\hline
\end{tabular*}

\vspace{0.5cm}\noindent \begin{tabular*}{\tableWidth}{|c|l@{\extracolsep{\fill}}r|}
\hline
\multicolumn{1}{|p{\maxVarWidth}}{clean\_divergence} & {\bf Scope:} restricted & BOOLEAN \\\hline
\multicolumn{3}{|p{\descWidth}|}{{\bf Description:}   {\em Use hyperbolic divergence cleaning}} \\
\hline & & {\bf Default:} no \\\hline
\end{tabular*}

\vspace{0.5cm}\noindent \begin{tabular*}{\tableWidth}{|c|l@{\extracolsep{\fill}}r|}
\hline
\multicolumn{1}{|p{\maxVarWidth}}{comoving\_attenuate} & {\bf Scope:} restricted & KEYWORD \\\hline
\multicolumn{3}{|p{\descWidth}|}{{\bf Description:}   {\em Which attenuation method for the comoving shift}} \\
\hline{\bf Range} & &  {\bf Default:} tanh \\\multicolumn{1}{|p{\maxVarWidth}|}{\centering sqrt} & \multicolumn{2}{p{\paraWidth}|}{Multiply by sqrt(rho/rho\_max)} \\\multicolumn{1}{|p{\maxVarWidth}|}{\centering tanh} & \multicolumn{2}{p{\paraWidth}|}{"Multiply by 1/2*(1+tanh(factor*( 
rho/rho\_max - offset)))"} \\\hline
\end{tabular*}

\vspace{0.5cm}\noindent \begin{tabular*}{\tableWidth}{|c|l@{\extracolsep{\fill}}r|}
\hline
\multicolumn{1}{|p{\maxVarWidth}}{comoving\_factor} & {\bf Scope:} restricted & REAL \\\hline
\multicolumn{3}{|p{\descWidth}|}{{\bf Description:}   {\em Factor multiplying the velocity for the comoving shift}} \\
\hline{\bf Range} & &  {\bf Default:} 0.0 \\\multicolumn{1}{|p{\maxVarWidth}|}{\centering 0.0:2.0} & \multicolumn{2}{p{\paraWidth}|}{[0,2] is allowed, but [0,1] is probably reasonable} \\\hline
\end{tabular*}

\vspace{0.5cm}\noindent \begin{tabular*}{\tableWidth}{|c|l@{\extracolsep{\fill}}r|}
\hline
\multicolumn{1}{|p{\maxVarWidth}}{comoving\_tanh\_factor} & {\bf Scope:} restricted & REAL \\\hline
\multicolumn{3}{|p{\descWidth}|}{{\bf Description:}   {\em The factor in the above tanh attenuation}} \\
\hline{\bf Range} & &  {\bf Default:} 10.0 \\\multicolumn{1}{|p{\maxVarWidth}|}{\centering (0.0:*} & \multicolumn{2}{p{\paraWidth}|}{Any positive number} \\\hline
\end{tabular*}

\vspace{0.5cm}\noindent \begin{tabular*}{\tableWidth}{|c|l@{\extracolsep{\fill}}r|}
\hline
\multicolumn{1}{|p{\maxVarWidth}}{comoving\_tanh\_offset} & {\bf Scope:} restricted & REAL \\\hline
\multicolumn{3}{|p{\descWidth}|}{{\bf Description:}   {\em The offset in the above tanh attenuation}} \\
\hline{\bf Range} & &  {\bf Default:} 0.05 \\\multicolumn{1}{|p{\maxVarWidth}|}{\centering 0.0:1.0} & \multicolumn{2}{p{\paraWidth}|}{Only makes sense in [0,1]} \\\hline
\end{tabular*}

\vspace{0.5cm}\noindent \begin{tabular*}{\tableWidth}{|c|l@{\extracolsep{\fill}}r|}
\hline
\multicolumn{1}{|p{\maxVarWidth}}{comoving\_v\_method} & {\bf Scope:} restricted & KEYWORD \\\hline
\multicolumn{3}{|p{\descWidth}|}{{\bf Description:}   {\em Which method for getting the radial velocity}} \\
\hline{\bf Range} & &  {\bf Default:} projected \\\multicolumn{1}{|p{\maxVarWidth}|}{\centering projected} & \multicolumn{2}{p{\paraWidth}|}{vr = x\_i . v\^i / r} \\\multicolumn{1}{|p{\maxVarWidth}|}{\centering components} & \multicolumn{2}{p{\paraWidth}|}{vr = sqrt(v\_i v\^i)} \\\hline
\end{tabular*}

\vspace{0.5cm}\noindent \begin{tabular*}{\tableWidth}{|c|l@{\extracolsep{\fill}}r|}
\hline
\multicolumn{1}{|p{\maxVarWidth}}{con2prim\_oct\_hack} & {\bf Scope:} restricted & BOOLEAN \\\hline
\multicolumn{3}{|p{\descWidth}|}{{\bf Description:}   {\em Disregard c2p failures in oct/rotsym90 boundary cells with xyz {\textless} 0}} \\
\hline & & {\bf Default:} no \\\hline
\end{tabular*}

\vspace{0.5cm}\noindent \begin{tabular*}{\tableWidth}{|c|l@{\extracolsep{\fill}}r|}
\hline
\multicolumn{1}{|p{\maxVarWidth}}{decouple\_normal\_bfield} & {\bf Scope:} restricted & BOOLEAN \\\hline
\multicolumn{3}{|p{\descWidth}|}{{\bf Description:}   {\em when using divergence cleaning properly decouple Bx,psidc subsystem}} \\
\hline & & {\bf Default:} yes \\\hline
\end{tabular*}

\vspace{0.5cm}\noindent \begin{tabular*}{\tableWidth}{|c|l@{\extracolsep{\fill}}r|}
\hline
\multicolumn{1}{|p{\maxVarWidth}}{enhanced\_ppm\_c2} & {\bf Scope:} restricted & REAL \\\hline
\multicolumn{3}{|p{\descWidth}|}{{\bf Description:}   {\em Parameter for enhancecd ppm limiter. Default from McCorquodale \& Colella 2011}} \\
\hline{\bf Range} & &  {\bf Default:} 1.25 \\\multicolumn{1}{|p{\maxVarWidth}|}{\centering *:*} & \multicolumn{2}{p{\paraWidth}|}{must be greater than 1. According to Colella\&Sekora 2008, enhanced ppm is insensitive to C in [1.25,5]} \\\hline
\end{tabular*}

\vspace{0.5cm}\noindent \begin{tabular*}{\tableWidth}{|c|l@{\extracolsep{\fill}}r|}
\hline
\multicolumn{1}{|p{\maxVarWidth}}{enhanced\_ppm\_c3} & {\bf Scope:} restricted & REAL \\\hline
\multicolumn{3}{|p{\descWidth}|}{{\bf Description:}   {\em Parameter for enhancecd ppm limiter. Default from McCorquodale \& Colella 2011}} \\
\hline{\bf Range} & &  {\bf Default:} 0.1 \\\multicolumn{1}{|p{\maxVarWidth}|}{\centering 0:*} & \multicolumn{2}{p{\paraWidth}|}{must be greater than 0.} \\\hline
\end{tabular*}

\vspace{0.5cm}\noindent \begin{tabular*}{\tableWidth}{|c|l@{\extracolsep{\fill}}r|}
\hline
\multicolumn{1}{|p{\maxVarWidth}}{eno\_order} & {\bf Scope:} restricted & INT \\\hline
\multicolumn{3}{|p{\descWidth}|}{{\bf Description:}   {\em The order of accuracy of the ENO reconstruction}} \\
\hline{\bf Range} & &  {\bf Default:} 2 \\\multicolumn{1}{|p{\maxVarWidth}|}{\centering 1:*} & \multicolumn{2}{p{\paraWidth}|}{Anything above 1, but above 5 is pointless} \\\hline
\end{tabular*}

\vspace{0.5cm}\noindent \begin{tabular*}{\tableWidth}{|c|l@{\extracolsep{\fill}}r|}
\hline
\multicolumn{1}{|p{\maxVarWidth}}{eos\_change} & {\bf Scope:} restricted & BOOLEAN \\\hline
\multicolumn{3}{|p{\descWidth}|}{{\bf Description:}   {\em Recalculate fluid quantities if changing the EoS}} \\
\hline & & {\bf Default:} no \\\hline
\end{tabular*}

\vspace{0.5cm}\noindent \begin{tabular*}{\tableWidth}{|c|l@{\extracolsep{\fill}}r|}
\hline
\multicolumn{1}{|p{\maxVarWidth}}{eos\_change\_type} & {\bf Scope:} restricted & KEYWORD \\\hline
\multicolumn{3}{|p{\descWidth}|}{{\bf Description:}   {\em Change polytropic K or Gamma?}} \\
\hline{\bf Range} & &  {\bf Default:} Gamma \\\multicolumn{1}{|p{\maxVarWidth}|}{\centering K} & \multicolumn{2}{p{\paraWidth}|}{Change the K} \\\multicolumn{1}{|p{\maxVarWidth}|}{\centering Gamma} & \multicolumn{2}{p{\paraWidth}|}{Change the Gamma} \\\multicolumn{1}{|p{\maxVarWidth}|}{\centering GammaKS} & \multicolumn{2}{p{\paraWidth}|}{Change K and Gamma, Shibata et al. 2004 3-D GR Core Collapse style} \\\hline
\end{tabular*}

\vspace{0.5cm}\noindent \begin{tabular*}{\tableWidth}{|c|l@{\extracolsep{\fill}}r|}
\hline
\multicolumn{1}{|p{\maxVarWidth}}{evolve\_tracer} & {\bf Scope:} restricted & BOOLEAN \\\hline
\multicolumn{3}{|p{\descWidth}|}{{\bf Description:}   {\em Should we advect tracers?}} \\
\hline & & {\bf Default:} no \\\hline
\end{tabular*}

\vspace{0.5cm}\noindent \begin{tabular*}{\tableWidth}{|c|l@{\extracolsep{\fill}}r|}
\hline
\multicolumn{1}{|p{\maxVarWidth}}{gradient\_method} & {\bf Scope:} restricted & KEYWORD \\\hline
\multicolumn{3}{|p{\descWidth}|}{{\bf Description:}   {\em Which method is used to set GRHydro::DiffRho?}} \\
\hline{\bf Range} & &  {\bf Default:} First diff \\\multicolumn{1}{|p{\maxVarWidth}|}{\centering First diff} & \multicolumn{2}{p{\paraWidth}|}{Standard first differences} \\\multicolumn{1}{|p{\maxVarWidth}|}{\centering Curvature} & \multicolumn{2}{p{\paraWidth}|}{Curvature based method of Paramesh / FLASH} \\\multicolumn{1}{|p{\maxVarWidth}|}{\centering Rho weighted} & \multicolumn{2}{p{\paraWidth}|}{Based on the size of rho} \\\hline
\end{tabular*}

\vspace{0.5cm}\noindent \begin{tabular*}{\tableWidth}{|c|l@{\extracolsep{\fill}}r|}
\hline
\multicolumn{1}{|p{\maxVarWidth}}{grhydro\_atmo\_tolerance} & {\bf Scope:} restricted & REAL \\\hline
\multicolumn{3}{|p{\descWidth}|}{{\bf Description:}   {\em A point is set to atmosphere in the Con2Prim's if its rho {\textless} GRHydro\_rho\_min *(1+GRHydro\_atmo\_tolerance). This avoids occasional spurious oscillations in carpet buffer zones lying in the atmosphere (because prolongation happens on conserved variables)}} \\
\hline{\bf Range} & &  {\bf Default:} 0.0 \\\multicolumn{1}{|p{\maxVarWidth}|}{\centering 0.0:} & \multicolumn{2}{p{\paraWidth}|}{Zero or larger. A useful value could be 0.0001} \\\hline
\end{tabular*}

\vspace{0.5cm}\noindent \begin{tabular*}{\tableWidth}{|c|l@{\extracolsep{\fill}}r|}
\hline
\multicolumn{1}{|p{\maxVarWidth}}{grhydro\_c2p\_failed\_action} & {\bf Scope:} restricted & KEYWORD \\\hline
\multicolumn{3}{|p{\descWidth}|}{{\bf Description:}   {\em what to do when we detect a c2p failure}} \\
\hline{\bf Range} & &  {\bf Default:} abort \\\multicolumn{1}{|p{\maxVarWidth}|}{\centering abort} & \multicolumn{2}{p{\paraWidth}|}{abort with error} \\\multicolumn{1}{|p{\maxVarWidth}|}{\centering terminate} & \multicolumn{2}{p{\paraWidth}|}{request termination} \\\hline
\end{tabular*}

\vspace{0.5cm}\noindent \begin{tabular*}{\tableWidth}{|c|l@{\extracolsep{\fill}}r|}
\hline
\multicolumn{1}{|p{\maxVarWidth}}{grhydro\_c2p\_reset\_eps\_tau\_hot\_eos} & {\bf Scope:} restricted & BOOLEAN \\\hline
\multicolumn{3}{|p{\descWidth}|}{{\bf Description:}   {\em As a last resort, reset tau}} \\
\hline & & {\bf Default:} no \\\hline
\end{tabular*}

\vspace{0.5cm}\noindent \begin{tabular*}{\tableWidth}{|c|l@{\extracolsep{\fill}}r|}
\hline
\multicolumn{1}{|p{\maxVarWidth}}{grhydro\_c2p\_warn\_from\_reflevel} & {\bf Scope:} restricted & INT \\\hline
\multicolumn{3}{|p{\descWidth}|}{{\bf Description:}   {\em Start warning on given refinement level and on higher levels}} \\
\hline{\bf Range} & &  {\bf Default:} (none) \\\multicolumn{1}{|p{\maxVarWidth}|}{\centering 0:} & \multicolumn{2}{p{\paraWidth}|}{Greater or equal to 0} \\\hline
\end{tabular*}

\vspace{0.5cm}\noindent \begin{tabular*}{\tableWidth}{|c|l@{\extracolsep{\fill}}r|}
\hline
\multicolumn{1}{|p{\maxVarWidth}}{grhydro\_c2p\_warnlevel} & {\bf Scope:} restricted & INT \\\hline
\multicolumn{3}{|p{\descWidth}|}{{\bf Description:}   {\em Warnlevel for Con2Prim warnings}} \\
\hline{\bf Range} & &  {\bf Default:} (none) \\\multicolumn{1}{|p{\maxVarWidth}|}{\centering 0:1} & \multicolumn{2}{p{\paraWidth}|}{Either 0 or 1} \\\hline
\end{tabular*}

\vspace{0.5cm}\noindent \begin{tabular*}{\tableWidth}{|c|l@{\extracolsep{\fill}}r|}
\hline
\multicolumn{1}{|p{\maxVarWidth}}{grhydro\_countmax} & {\bf Scope:} restricted & INT \\\hline
\multicolumn{3}{|p{\descWidth}|}{{\bf Description:}   {\em Maximum number of iterations for Con2Prim solve}} \\
\hline{\bf Range} & &  {\bf Default:} 100 \\\multicolumn{1}{|p{\maxVarWidth}|}{\centering 1:*} & \multicolumn{2}{p{\paraWidth}|}{Must be positive} \\\hline
\end{tabular*}

\vspace{0.5cm}\noindent \begin{tabular*}{\tableWidth}{|c|l@{\extracolsep{\fill}}r|}
\hline
\multicolumn{1}{|p{\maxVarWidth}}{grhydro\_countmin} & {\bf Scope:} restricted & INT \\\hline
\multicolumn{3}{|p{\descWidth}|}{{\bf Description:}   {\em Minimum number of iterations for Con2Prim solve}} \\
\hline{\bf Range} & &  {\bf Default:} 1 \\\multicolumn{1}{|p{\maxVarWidth}|}{\centering 0:*} & \multicolumn{2}{p{\paraWidth}|}{Must be non negative} \\\hline
\end{tabular*}

\vspace{0.5cm}\noindent \begin{tabular*}{\tableWidth}{|c|l@{\extracolsep{\fill}}r|}
\hline
\multicolumn{1}{|p{\maxVarWidth}}{grhydro\_del\_ptol} & {\bf Scope:} restricted & REAL \\\hline
\multicolumn{3}{|p{\descWidth}|}{{\bf Description:}   {\em Tolerance for primitive variable solve (absolute)}} \\
\hline{\bf Range} & &  {\bf Default:} 1.e-18 \\\multicolumn{1}{|p{\maxVarWidth}|}{\centering 0:} & \multicolumn{2}{p{\paraWidth}|}{Do we really want both tolerances?} \\\hline
\end{tabular*}

\vspace{0.5cm}\noindent \begin{tabular*}{\tableWidth}{|c|l@{\extracolsep{\fill}}r|}
\hline
\multicolumn{1}{|p{\maxVarWidth}}{grhydro\_enable\_internal\_excision} & {\bf Scope:} restricted & BOOLEAN \\\hline
\multicolumn{3}{|p{\descWidth}|}{{\bf Description:}   {\em Set this to 'false' to disable the thorn-internal excision.}} \\
\hline & & {\bf Default:} true \\\hline
\end{tabular*}

\vspace{0.5cm}\noindent \begin{tabular*}{\tableWidth}{|c|l@{\extracolsep{\fill}}r|}
\hline
\multicolumn{1}{|p{\maxVarWidth}}{grhydro\_eos\_hot\_eps\_fix} & {\bf Scope:} restricted & BOOLEAN \\\hline
\multicolumn{3}{|p{\descWidth}|}{{\bf Description:}   {\em Activate quasi-failsafe mode for hot EOSs}} \\
\hline & & {\bf Default:} no \\\hline
\end{tabular*}

\vspace{0.5cm}\noindent \begin{tabular*}{\tableWidth}{|c|l@{\extracolsep{\fill}}r|}
\hline
\multicolumn{1}{|p{\maxVarWidth}}{grhydro\_eos\_hot\_prim2con\_warn} & {\bf Scope:} restricted & BOOLEAN \\\hline
\multicolumn{3}{|p{\descWidth}|}{{\bf Description:}   {\em Warn about temperature workaround in prim2con}} \\
\hline & & {\bf Default:} yes \\\hline
\end{tabular*}

\vspace{0.5cm}\noindent \begin{tabular*}{\tableWidth}{|c|l@{\extracolsep{\fill}}r|}
\hline
\multicolumn{1}{|p{\maxVarWidth}}{grhydro\_eos\_rf\_prec} & {\bf Scope:} restricted & REAL \\\hline
\multicolumn{3}{|p{\descWidth}|}{{\bf Description:}   {\em Precision to which root finding should be carried out}} \\
\hline{\bf Range} & &  {\bf Default:} 1.0e-8 \\\multicolumn{1}{|p{\maxVarWidth}|}{\centering (0.0:*} & \multicolumn{2}{p{\paraWidth}|}{anything larger than 0 goes} \\\hline
\end{tabular*}

\vspace{0.5cm}\noindent \begin{tabular*}{\tableWidth}{|c|l@{\extracolsep{\fill}}r|}
\hline
\multicolumn{1}{|p{\maxVarWidth}}{grhydro\_eos\_table} & {\bf Scope:} restricted & STRING \\\hline
\multicolumn{3}{|p{\descWidth}|}{{\bf Description:}   {\em Name for the Equation of State}} \\
\hline{\bf Range} & &  {\bf Default:} Ideal\_Fluid \\\multicolumn{1}{|p{\maxVarWidth}|}{\centering .*} & \multicolumn{2}{p{\paraWidth}|}{Can be anything} \\\hline
\end{tabular*}

\vspace{0.5cm}\noindent \begin{tabular*}{\tableWidth}{|c|l@{\extracolsep{\fill}}r|}
\hline
\multicolumn{1}{|p{\maxVarWidth}}{grhydro\_eos\_type} & {\bf Scope:} restricted & KEYWORD \\\hline
\multicolumn{3}{|p{\descWidth}|}{{\bf Description:}   {\em Type of Equation of State}} \\
\hline{\bf Range} & &  {\bf Default:} General \\\multicolumn{1}{|p{\maxVarWidth}|}{\centering Polytype} & \multicolumn{2}{p{\paraWidth}|}{P = P(rho) or P = P(eps)} \\\multicolumn{1}{|p{\maxVarWidth}|}{\centering General} & \multicolumn{2}{p{\paraWidth}|}{P = P(rho, eps)} \\\hline
\end{tabular*}

\vspace{0.5cm}\noindent \begin{tabular*}{\tableWidth}{|c|l@{\extracolsep{\fill}}r|}
\hline
\multicolumn{1}{|p{\maxVarWidth}}{grhydro\_eps\_min} & {\bf Scope:} restricted & REAL \\\hline
\multicolumn{3}{|p{\descWidth}|}{{\bf Description:}   {\em Minimum value of specific internal energy - this is now only used in GRHydro\_InitData's GRHydro\_Only\_Atmo routine}} \\
\hline{\bf Range} & &  {\bf Default:} 1.e-10 \\\multicolumn{1}{|p{\maxVarWidth}|}{\centering 0:} & \multicolumn{2}{p{\paraWidth}|}{Positive} \\\hline
\end{tabular*}

\vspace{0.5cm}\noindent \begin{tabular*}{\tableWidth}{|c|l@{\extracolsep{\fill}}r|}
\hline
\multicolumn{1}{|p{\maxVarWidth}}{grhydro\_hot\_atmo\_temp} & {\bf Scope:} restricted & REAL \\\hline
\multicolumn{3}{|p{\descWidth}|}{{\bf Description:}   {\em Temperature of the hot atmosphere in MeV}} \\
\hline{\bf Range} & &  {\bf Default:} 0.1e0 \\\multicolumn{1}{|p{\maxVarWidth}|}{\centering (0.0:*} & \multicolumn{2}{p{\paraWidth}|}{Larger than 0 MeV} \\\hline
\end{tabular*}

\vspace{0.5cm}\noindent \begin{tabular*}{\tableWidth}{|c|l@{\extracolsep{\fill}}r|}
\hline
\multicolumn{1}{|p{\maxVarWidth}}{grhydro\_hot\_atmo\_y\_e} & {\bf Scope:} restricted & REAL \\\hline
\multicolumn{3}{|p{\descWidth}|}{{\bf Description:}   {\em Y\_e of the hot atmosphere}} \\
\hline{\bf Range} & &  {\bf Default:} 0.5e0 \\\multicolumn{1}{|p{\maxVarWidth}|}{\centering 0.0:*} & \multicolumn{2}{p{\paraWidth}|}{Larger than 0} \\\hline
\end{tabular*}

\vspace{0.5cm}\noindent \begin{tabular*}{\tableWidth}{|c|l@{\extracolsep{\fill}}r|}
\hline
\multicolumn{1}{|p{\maxVarWidth}}{grhydro\_hydro\_excision} & {\bf Scope:} restricted & INT \\\hline
\multicolumn{3}{|p{\descWidth}|}{{\bf Description:}   {\em Turns excision automatically on in HydroBase}} \\
\hline{\bf Range} & &  {\bf Default:} 1 \\\multicolumn{1}{|p{\maxVarWidth}|}{\centering 1:1} & \multicolumn{2}{p{\paraWidth}|}{Only '1' allowed} \\\hline
\end{tabular*}

\vspace{0.5cm}\noindent \begin{tabular*}{\tableWidth}{|c|l@{\extracolsep{\fill}}r|}
\hline
\multicolumn{1}{|p{\maxVarWidth}}{grhydro\_lorentz\_overshoot\_cutoff} & {\bf Scope:} restricted & REAL \\\hline
\multicolumn{3}{|p{\descWidth}|}{{\bf Description:}   {\em Set the Lorentz factor to this value in case it overshoots (1/0)}} \\
\hline{\bf Range} & &  {\bf Default:} 1.e100 \\\multicolumn{1}{|p{\maxVarWidth}|}{\centering 0:*} & \multicolumn{2}{p{\paraWidth}|}{Some big value} \\\hline
\end{tabular*}

\vspace{0.5cm}\noindent \begin{tabular*}{\tableWidth}{|c|l@{\extracolsep{\fill}}r|}
\hline
\multicolumn{1}{|p{\maxVarWidth}}{grhydro\_max\_temp} & {\bf Scope:} restricted & REAL \\\hline
\multicolumn{3}{|p{\descWidth}|}{{\bf Description:}   {\em maximum temperature we allow}} \\
\hline{\bf Range} & &  {\bf Default:} 90.0e0 \\\multicolumn{1}{|p{\maxVarWidth}|}{\centering (0.0:*} & \multicolumn{2}{p{\paraWidth}|}{Larger than 0 MeV} \\\hline
\end{tabular*}

\vspace{0.5cm}\noindent \begin{tabular*}{\tableWidth}{|c|l@{\extracolsep{\fill}}r|}
\hline
\multicolumn{1}{|p{\maxVarWidth}}{grhydro\_maxnumconstrainedvars} & {\bf Scope:} restricted & INT \\\hline
\multicolumn{3}{|p{\descWidth}|}{{\bf Description:}   {\em The maximum number of constrained variables used by GRHydro}} \\
\hline{\bf Range} & &  {\bf Default:} 37 \\\multicolumn{1}{|p{\maxVarWidth}|}{\centering 7:48} & \multicolumn{2}{p{\paraWidth}|}{A small range, depending on testing or not} \\\hline
\end{tabular*}

\vspace{0.5cm}\noindent \begin{tabular*}{\tableWidth}{|c|l@{\extracolsep{\fill}}r|}
\hline
\multicolumn{1}{|p{\maxVarWidth}}{grhydro\_maxnumevolvedvars} & {\bf Scope:} restricted & INT \\\hline
\multicolumn{3}{|p{\descWidth}|}{{\bf Description:}   {\em The maximum number of evolved variables used by GRHydro}} \\
\hline{\bf Range} & &  {\bf Default:} 5 \\\multicolumn{1}{|p{\maxVarWidth}|}{\centering } & \multicolumn{2}{p{\paraWidth}|}{when using multirate} \\\multicolumn{1}{|p{\maxVarWidth}|}{\centering 5:12} & \multicolumn{2}{p{\paraWidth}|}{dens scon[3] tau (B/A)vec[3] psidc ye entropy Aphi} \\\hline
\end{tabular*}

\vspace{0.5cm}\noindent \begin{tabular*}{\tableWidth}{|c|l@{\extracolsep{\fill}}r|}
\hline
\multicolumn{1}{|p{\maxVarWidth}}{grhydro\_maxnumevolvedvarsslow} & {\bf Scope:} restricted & INT \\\hline
\multicolumn{3}{|p{\descWidth}|}{{\bf Description:}   {\em The maximum number of evolved variables used by GRHydro}} \\
\hline{\bf Range} & &  {\bf Default:} (none) \\\multicolumn{1}{|p{\maxVarWidth}|}{\centering } & \multicolumn{2}{p{\paraWidth}|}{do not use multirate} \\\multicolumn{1}{|p{\maxVarWidth}|}{\centering 5:12} & \multicolumn{2}{p{\paraWidth}|}{dens scon[3] tau (B/A)vec[3] psidc ye entropy Aphi} \\\hline
\end{tabular*}

\vspace{0.5cm}\noindent \begin{tabular*}{\tableWidth}{|c|l@{\extracolsep{\fill}}r|}
\hline
\multicolumn{1}{|p{\maxVarWidth}}{grhydro\_maxnumsandrvars} & {\bf Scope:} restricted & INT \\\hline
\multicolumn{3}{|p{\descWidth}|}{{\bf Description:}   {\em The maximum number of save and restore variables used by GRHydro}} \\
\hline{\bf Range} & &  {\bf Default:} 16 \\\multicolumn{1}{|p{\maxVarWidth}|}{\centering 0:16} & \multicolumn{2}{p{\paraWidth}|}{A small range, depending on testing or not} \\\hline
\end{tabular*}

\vspace{0.5cm}\noindent \begin{tabular*}{\tableWidth}{|c|l@{\extracolsep{\fill}}r|}
\hline
\multicolumn{1}{|p{\maxVarWidth}}{grhydro\_nan\_verbose} & {\bf Scope:} restricted & INT \\\hline
\multicolumn{3}{|p{\descWidth}|}{{\bf Description:}   {\em The warning level for NaNs occuring within GRHydro}} \\
\hline{\bf Range} & &  {\bf Default:} 2 \\\multicolumn{1}{|p{\maxVarWidth}|}{\centering 0:*} & \multicolumn{2}{p{\paraWidth}|}{The warning level} \\\hline
\end{tabular*}

\vspace{0.5cm}\noindent \begin{tabular*}{\tableWidth}{|c|l@{\extracolsep{\fill}}r|}
\hline
\multicolumn{1}{|p{\maxVarWidth}}{grhydro\_oppm\_reflevel} & {\bf Scope:} restricted & INT \\\hline
\multicolumn{3}{|p{\descWidth}|}{{\bf Description:}   {\em Ref level where oPPM is used instead of ePPM (used with use\_enhaced\_ppm=yes).}} \\
\hline{\bf Range} & &  {\bf Default:} -1 \\\multicolumn{1}{|p{\maxVarWidth}|}{\centering -1:10} & \multicolumn{2}{p{\paraWidth}|}{0-10 (the reflevel number) or -1 (off)} \\\hline
\end{tabular*}

\vspace{0.5cm}\noindent \begin{tabular*}{\tableWidth}{|c|l@{\extracolsep{\fill}}r|}
\hline
\multicolumn{1}{|p{\maxVarWidth}}{grhydro\_perc\_ptol} & {\bf Scope:} restricted & REAL \\\hline
\multicolumn{3}{|p{\descWidth}|}{{\bf Description:}   {\em Tolerance for primitive variable solve (percent)}} \\
\hline{\bf Range} & &  {\bf Default:} 1.e-8 \\\multicolumn{1}{|p{\maxVarWidth}|}{\centering 0:} & \multicolumn{2}{p{\paraWidth}|}{Do we really want both tolerances?} \\\hline
\end{tabular*}

\vspace{0.5cm}\noindent \begin{tabular*}{\tableWidth}{|c|l@{\extracolsep{\fill}}r|}
\hline
\multicolumn{1}{|p{\maxVarWidth}}{grhydro\_polish} & {\bf Scope:} restricted & INT \\\hline
\multicolumn{3}{|p{\descWidth}|}{{\bf Description:}   {\em Number of extra iterations after root found}} \\
\hline{\bf Range} & &  {\bf Default:} (none) \\\multicolumn{1}{|p{\maxVarWidth}|}{\centering 0:*} & \multicolumn{2}{p{\paraWidth}|}{Must be non negative} \\\hline
\end{tabular*}

\vspace{0.5cm}\noindent \begin{tabular*}{\tableWidth}{|c|l@{\extracolsep{\fill}}r|}
\hline
\multicolumn{1}{|p{\maxVarWidth}}{grhydro\_rho\_central} & {\bf Scope:} restricted & REAL \\\hline
\multicolumn{3}{|p{\descWidth}|}{{\bf Description:}   {\em Central Density for Star}} \\
\hline{\bf Range} & &  {\bf Default:} 1.e-5 \\\multicolumn{1}{|p{\maxVarWidth}|}{\centering :} & \multicolumn{2}{p{\paraWidth}|}{} \\\hline
\end{tabular*}

\vspace{0.5cm}\noindent \begin{tabular*}{\tableWidth}{|c|l@{\extracolsep{\fill}}r|}
\hline
\multicolumn{1}{|p{\maxVarWidth}}{grhydro\_stencil} & {\bf Scope:} restricted & INT \\\hline
\multicolumn{3}{|p{\descWidth}|}{{\bf Description:}   {\em Width of the stencil}} \\
\hline{\bf Range} & &  {\bf Default:} 2 \\\multicolumn{1}{|p{\maxVarWidth}|}{\centering 0:} & \multicolumn{2}{p{\paraWidth}|}{Must be positive} \\\hline
\end{tabular*}

\vspace{0.5cm}\noindent \begin{tabular*}{\tableWidth}{|c|l@{\extracolsep{\fill}}r|}
\hline
\multicolumn{1}{|p{\maxVarWidth}}{grhydro\_y\_e\_max} & {\bf Scope:} restricted & REAL \\\hline
\multicolumn{3}{|p{\descWidth}|}{{\bf Description:}   {\em maximum allowed Y\_e}} \\
\hline{\bf Range} & &  {\bf Default:} 1.0 \\\multicolumn{1}{|p{\maxVarWidth}|}{\centering 0.0:*} & \multicolumn{2}{p{\paraWidth}|}{Larger than or equal to zero; 1 is default} \\\hline
\end{tabular*}

\vspace{0.5cm}\noindent \begin{tabular*}{\tableWidth}{|c|l@{\extracolsep{\fill}}r|}
\hline
\multicolumn{1}{|p{\maxVarWidth}}{grhydro\_y\_e\_min} & {\bf Scope:} restricted & REAL \\\hline
\multicolumn{3}{|p{\descWidth}|}{{\bf Description:}   {\em minimum allowed Y\_e}} \\
\hline{\bf Range} & &  {\bf Default:} 0.0 \\\multicolumn{1}{|p{\maxVarWidth}|}{\centering 0.0:*} & \multicolumn{2}{p{\paraWidth}|}{Larger than or equal to zero} \\\hline
\end{tabular*}

\vspace{0.5cm}\noindent \begin{tabular*}{\tableWidth}{|c|l@{\extracolsep{\fill}}r|}
\hline
\multicolumn{1}{|p{\maxVarWidth}}{hlle\_type} & {\bf Scope:} restricted & KEYWORD \\\hline
\multicolumn{3}{|p{\descWidth}|}{{\bf Description:}   {\em Which HLLE type to use}} \\
\hline{\bf Range} & &  {\bf Default:} Standard \\\multicolumn{1}{|p{\maxVarWidth}|}{\centering Standard} & \multicolumn{2}{p{\paraWidth}|}{Standard HLLE solver} \\\multicolumn{1}{|p{\maxVarWidth}|}{\centering Tadmor} & \multicolumn{2}{p{\paraWidth}|}{Tadmor's simplification of HLLE} \\\hline
\end{tabular*}

\vspace{0.5cm}\noindent \begin{tabular*}{\tableWidth}{|c|l@{\extracolsep{\fill}}r|}
\hline
\multicolumn{1}{|p{\maxVarWidth}}{initial\_atmosphere\_factor} & {\bf Scope:} restricted & REAL \\\hline
\multicolumn{3}{|p{\descWidth}|}{{\bf Description:}   {\em A relative (to the initial atmosphere) value for rho in the atmosphere. This is used at initial time only. Unused if negative.}} \\
\hline{\bf Range} & &  {\bf Default:} -1.0 \\\multicolumn{1}{|p{\maxVarWidth}|}{\centering -1.0:} & \multicolumn{2}{p{\paraWidth}|}{} \\\hline
\end{tabular*}

\vspace{0.5cm}\noindent \begin{tabular*}{\tableWidth}{|c|l@{\extracolsep{\fill}}r|}
\hline
\multicolumn{1}{|p{\maxVarWidth}}{initial\_gamma} & {\bf Scope:} restricted & REAL \\\hline
\multicolumn{3}{|p{\descWidth}|}{{\bf Description:}   {\em If changing Gamma, what was the value used in the initial data routine?}} \\
\hline{\bf Range} & &  {\bf Default:} 1.3333 \\\multicolumn{1}{|p{\maxVarWidth}|}{\centering (0.0:} & \multicolumn{2}{p{\paraWidth}|}{Positive} \\\hline
\end{tabular*}

\vspace{0.5cm}\noindent \begin{tabular*}{\tableWidth}{|c|l@{\extracolsep{\fill}}r|}
\hline
\multicolumn{1}{|p{\maxVarWidth}}{initial\_k} & {\bf Scope:} restricted & REAL \\\hline
\multicolumn{3}{|p{\descWidth}|}{{\bf Description:}   {\em If changing K, what was the value used in the initial data routine?}} \\
\hline{\bf Range} & &  {\bf Default:} 100.0 \\\multicolumn{1}{|p{\maxVarWidth}|}{\centering (0.0:} & \multicolumn{2}{p{\paraWidth}|}{Positive} \\\hline
\end{tabular*}

\vspace{0.5cm}\noindent \begin{tabular*}{\tableWidth}{|c|l@{\extracolsep{\fill}}r|}
\hline
\multicolumn{1}{|p{\maxVarWidth}}{initial\_rho\_abs\_min} & {\bf Scope:} restricted & REAL \\\hline
\multicolumn{3}{|p{\descWidth}|}{{\bf Description:}   {\em An absolute value for rho in the atmosphere. To be used by initial data routines only. Unused if negative.}} \\
\hline{\bf Range} & &  {\bf Default:} -1.0 \\\multicolumn{1}{|p{\maxVarWidth}|}{\centering -1.0:} & \multicolumn{2}{p{\paraWidth}|}{} \\\hline
\end{tabular*}

\vspace{0.5cm}\noindent \begin{tabular*}{\tableWidth}{|c|l@{\extracolsep{\fill}}r|}
\hline
\multicolumn{1}{|p{\maxVarWidth}}{initial\_rho\_rel\_min} & {\bf Scope:} restricted & REAL \\\hline
\multicolumn{3}{|p{\descWidth}|}{{\bf Description:}   {\em A relative (to the central density) value for rho in the atmosphere. To be used by initial data routines only. Unused if negative.}} \\
\hline{\bf Range} & &  {\bf Default:} -1.0 \\\multicolumn{1}{|p{\maxVarWidth}|}{\centering -1.0:} & \multicolumn{2}{p{\paraWidth}|}{} \\\hline
\end{tabular*}

\vspace{0.5cm}\noindent \begin{tabular*}{\tableWidth}{|c|l@{\extracolsep{\fill}}r|}
\hline
\multicolumn{1}{|p{\maxVarWidth}}{kap\_dc} & {\bf Scope:} restricted & REAL \\\hline
\multicolumn{3}{|p{\descWidth}|}{{\bf Description:}   {\em The kap parameter for divergence cleaning}} \\
\hline{\bf Range} & &  {\bf Default:} 10.0 \\\multicolumn{1}{|p{\maxVarWidth}|}{\centering 0:*} & \multicolumn{2}{p{\paraWidth}|}{Any non-negative value, but 1.0 to 10.0 seems preferred} \\\hline
\end{tabular*}

\vspace{0.5cm}\noindent \begin{tabular*}{\tableWidth}{|c|l@{\extracolsep{\fill}}r|}
\hline
\multicolumn{1}{|p{\maxVarWidth}}{left\_eigenvectors} & {\bf Scope:} restricted & KEYWORD \\\hline
\multicolumn{3}{|p{\descWidth}|}{{\bf Description:}   {\em How to compute the left eigenvectors}} \\
\hline{\bf Range} & &  {\bf Default:} analytical \\\multicolumn{1}{|p{\maxVarWidth}|}{\centering analytical} & \multicolumn{2}{p{\paraWidth}|}{Analytical left eigenvectors} \\\multicolumn{1}{|p{\maxVarWidth}|}{\centering numerical} & \multicolumn{2}{p{\paraWidth}|}{Numerical left eigenvectors} \\\hline
\end{tabular*}

\vspace{0.5cm}\noindent \begin{tabular*}{\tableWidth}{|c|l@{\extracolsep{\fill}}r|}
\hline
\multicolumn{1}{|p{\maxVarWidth}}{max\_magnetic\_to\_gas\_pressure\_ratio} & {\bf Scope:} restricted & REAL \\\hline
\multicolumn{3}{|p{\descWidth}|}{{\bf Description:}   {\em consider pressure to be magnetically dominated if magnetic pressure to gas pressure ratio is higher than this}} \\
\hline{\bf Range} & &  {\bf Default:} -1.0 \\\multicolumn{1}{|p{\maxVarWidth}|}{\centering (0:*} & \multicolumn{2}{p{\paraWidth}|}{any positive value, eg. 100.} \\\multicolumn{1}{|p{\maxVarWidth}|}{\centering -1.0} & \multicolumn{2}{p{\paraWidth}|}{disable} \\\hline
\end{tabular*}

\vspace{0.5cm}\noindent \begin{tabular*}{\tableWidth}{|c|l@{\extracolsep{\fill}}r|}
\hline
\multicolumn{1}{|p{\maxVarWidth}}{method\_type} & {\bf Scope:} restricted & KEYWORD \\\hline
\multicolumn{3}{|p{\descWidth}|}{{\bf Description:}   {\em Which type of method to use}} \\
\hline{\bf Range} & &  {\bf Default:} RSA FV \\\multicolumn{1}{|p{\maxVarWidth}|}{\centering RSA FV} & \multicolumn{2}{p{\paraWidth}|}{"Reconstruct-Solve-A 
verage finite volume method"} \\\multicolumn{1}{|p{\maxVarWidth}|}{\centering Flux Split FD} & \multicolumn{2}{p{\paraWidth}|}{Finite difference using Lax-Friedrichs flux splitting} \\\hline
\end{tabular*}

\vspace{0.5cm}\noindent \begin{tabular*}{\tableWidth}{|c|l@{\extracolsep{\fill}}r|}
\hline
\multicolumn{1}{|p{\maxVarWidth}}{min\_tracer} & {\bf Scope:} restricted & REAL \\\hline
\multicolumn{3}{|p{\descWidth}|}{{\bf Description:}   {\em The floor placed on the tracer}} \\
\hline{\bf Range} & &  {\bf Default:} 0.0 \\\multicolumn{1}{|p{\maxVarWidth}|}{\centering *:*} & \multicolumn{2}{p{\paraWidth}|}{Anything} \\\hline
\end{tabular*}

\vspace{0.5cm}\noindent \begin{tabular*}{\tableWidth}{|c|l@{\extracolsep{\fill}}r|}
\hline
\multicolumn{1}{|p{\maxVarWidth}}{mp5\_adaptive\_eps} & {\bf Scope:} restricted & BOOLEAN \\\hline
\multicolumn{3}{|p{\descWidth}|}{{\bf Description:}   {\em Same as WENO adaptive epsilon: adaptively reduce mp5\_eps according to norm of stencil. Original algorithm does not use this.}} \\
\hline & & {\bf Default:} no \\\hline
\end{tabular*}

\vspace{0.5cm}\noindent \begin{tabular*}{\tableWidth}{|c|l@{\extracolsep{\fill}}r|}
\hline
\multicolumn{1}{|p{\maxVarWidth}}{mp5\_alpha} & {\bf Scope:} restricted & REAL \\\hline
\multicolumn{3}{|p{\descWidth}|}{{\bf Description:}   {\em alpha parameter for MP5 reconstruction}} \\
\hline{\bf Range} & &  {\bf Default:} 4.0 \\\multicolumn{1}{|p{\maxVarWidth}|}{\centering 0:*} & \multicolumn{2}{p{\paraWidth}|}{positive} \\\hline
\end{tabular*}

\vspace{0.5cm}\noindent \begin{tabular*}{\tableWidth}{|c|l@{\extracolsep{\fill}}r|}
\hline
\multicolumn{1}{|p{\maxVarWidth}}{mp5\_eps} & {\bf Scope:} restricted & REAL \\\hline
\multicolumn{3}{|p{\descWidth}|}{{\bf Description:}   {\em epsilon parameter for MP5 reconstruction}} \\
\hline{\bf Range} & &  {\bf Default:} 0.0 \\\multicolumn{1}{|p{\maxVarWidth}|}{\centering 0:*} & \multicolumn{2}{p{\paraWidth}|}{0.0 or very small and positive. 1e-10 is suggested by Suresh\&Huynh. TOV star requires 0.0} \\\hline
\end{tabular*}

\vspace{0.5cm}\noindent \begin{tabular*}{\tableWidth}{|c|l@{\extracolsep{\fill}}r|}
\hline
\multicolumn{1}{|p{\maxVarWidth}}{myfloor} & {\bf Scope:} restricted & REAL \\\hline
\multicolumn{3}{|p{\descWidth}|}{{\bf Description:}   {\em A minimum number for the TVD reconstruction routine}} \\
\hline{\bf Range} & &  {\bf Default:} 1.e-10 \\\multicolumn{1}{|p{\maxVarWidth}|}{\centering 0.0:} & \multicolumn{2}{p{\paraWidth}|}{Must be positive} \\\hline
\end{tabular*}

\vspace{0.5cm}\noindent \begin{tabular*}{\tableWidth}{|c|l@{\extracolsep{\fill}}r|}
\hline
\multicolumn{1}{|p{\maxVarWidth}}{number\_of\_arrays} & {\bf Scope:} restricted & INT \\\hline
\multicolumn{3}{|p{\descWidth}|}{{\bf Description:}   {\em Number of arrays to evolve}} \\
\hline{\bf Range} & &  {\bf Default:} (none) \\\multicolumn{1}{|p{\maxVarWidth}|}{\centering 0:3} & \multicolumn{2}{p{\paraWidth}|}{Either zero or three, depending on the particles} \\\hline
\end{tabular*}

\vspace{0.5cm}\noindent \begin{tabular*}{\tableWidth}{|c|l@{\extracolsep{\fill}}r|}
\hline
\multicolumn{1}{|p{\maxVarWidth}}{number\_of\_particles} & {\bf Scope:} restricted & INT \\\hline
\multicolumn{3}{|p{\descWidth}|}{{\bf Description:}   {\em Number of particles to track}} \\
\hline{\bf Range} & &  {\bf Default:} (none) \\\multicolumn{1}{|p{\maxVarWidth}|}{\centering 0:*} & \multicolumn{2}{p{\paraWidth}|}{0 switches off particle tracking} \\\hline
\end{tabular*}

\vspace{0.5cm}\noindent \begin{tabular*}{\tableWidth}{|c|l@{\extracolsep{\fill}}r|}
\hline
\multicolumn{1}{|p{\maxVarWidth}}{number\_of\_tracers} & {\bf Scope:} restricted & INT \\\hline
\multicolumn{3}{|p{\descWidth}|}{{\bf Description:}   {\em Number of tracer variables to be advected}} \\
\hline{\bf Range} & &  {\bf Default:} (none) \\\multicolumn{1}{|p{\maxVarWidth}|}{\centering 0:*} & \multicolumn{2}{p{\paraWidth}|}{positive or zero} \\\hline
\end{tabular*}

\vspace{0.5cm}\noindent \begin{tabular*}{\tableWidth}{|c|l@{\extracolsep{\fill}}r|}
\hline
\multicolumn{1}{|p{\maxVarWidth}}{numerical\_viscosity} & {\bf Scope:} restricted & KEYWORD \\\hline
\multicolumn{3}{|p{\descWidth}|}{{\bf Description:}   {\em How to compute the numerical viscosity}} \\
\hline{\bf Range} & &  {\bf Default:} fast \\\multicolumn{1}{|p{\maxVarWidth}|}{\centering fast} & \multicolumn{2}{p{\paraWidth}|}{Fast numerical viscosity} \\\multicolumn{1}{|p{\maxVarWidth}|}{\centering normal} & \multicolumn{2}{p{\paraWidth}|}{Normal numerical viscosity} \\\hline
\end{tabular*}

\vspace{0.5cm}\noindent \begin{tabular*}{\tableWidth}{|c|l@{\extracolsep{\fill}}r|}
\hline
\multicolumn{1}{|p{\maxVarWidth}}{particle\_interpolation\_order} & {\bf Scope:} restricted & INT \\\hline
\multicolumn{3}{|p{\descWidth}|}{{\bf Description:}   {\em What order should be used for the particle interpolation}} \\
\hline{\bf Range} & &  {\bf Default:} 2 \\\multicolumn{1}{|p{\maxVarWidth}|}{\centering 1:*} & \multicolumn{2}{p{\paraWidth}|}{A valid positive interpolation order} \\\hline
\end{tabular*}

\vspace{0.5cm}\noindent \begin{tabular*}{\tableWidth}{|c|l@{\extracolsep{\fill}}r|}
\hline
\multicolumn{1}{|p{\maxVarWidth}}{particle\_interpolator} & {\bf Scope:} restricted & STRING \\\hline
\multicolumn{3}{|p{\descWidth}|}{{\bf Description:}   {\em What interpolator should be used for the particles}} \\
\hline{\bf Range} & &  {\bf Default:} Lagrange polynomial interpolation \\\multicolumn{1}{|p{\maxVarWidth}|}{\centering .+} & \multicolumn{2}{p{\paraWidth}|}{A valid interpolator name} \\\hline
\end{tabular*}

\vspace{0.5cm}\noindent \begin{tabular*}{\tableWidth}{|c|l@{\extracolsep{\fill}}r|}
\hline
\multicolumn{1}{|p{\maxVarWidth}}{ppm\_detect} & {\bf Scope:} restricted & BOOLEAN \\\hline
\multicolumn{3}{|p{\descWidth}|}{{\bf Description:}   {\em Should the PPM solver use shock detection?}} \\
\hline & & {\bf Default:} no \\\hline
\end{tabular*}

\vspace{0.5cm}\noindent \begin{tabular*}{\tableWidth}{|c|l@{\extracolsep{\fill}}r|}
\hline
\multicolumn{1}{|p{\maxVarWidth}}{ppm\_epsilon} & {\bf Scope:} restricted & REAL \\\hline
\multicolumn{3}{|p{\descWidth}|}{{\bf Description:}   {\em Epsilon for PPM zone flattening}} \\
\hline{\bf Range} & &  {\bf Default:} 0.33 \\\multicolumn{1}{|p{\maxVarWidth}|}{\centering 0.0:} & \multicolumn{2}{p{\paraWidth}|}{Must be positive. Default is from Colella \& Woodward} \\\hline
\end{tabular*}

\vspace{0.5cm}\noindent \begin{tabular*}{\tableWidth}{|c|l@{\extracolsep{\fill}}r|}
\hline
\multicolumn{1}{|p{\maxVarWidth}}{ppm\_epsilon\_shock} & {\bf Scope:} restricted & REAL \\\hline
\multicolumn{3}{|p{\descWidth}|}{{\bf Description:}   {\em Epsilon for PPM shock detection}} \\
\hline{\bf Range} & &  {\bf Default:} 0.01 \\\multicolumn{1}{|p{\maxVarWidth}|}{\centering :} & \multicolumn{2}{p{\paraWidth}|}{Anything goes. Default is from Colella \& Woodward} \\\hline
\end{tabular*}

\vspace{0.5cm}\noindent \begin{tabular*}{\tableWidth}{|c|l@{\extracolsep{\fill}}r|}
\hline
\multicolumn{1}{|p{\maxVarWidth}}{ppm\_eta1} & {\bf Scope:} restricted & REAL \\\hline
\multicolumn{3}{|p{\descWidth}|}{{\bf Description:}   {\em Eta1 for PPM shock detection}} \\
\hline{\bf Range} & &  {\bf Default:} 20.0 \\\multicolumn{1}{|p{\maxVarWidth}|}{\centering :} & \multicolumn{2}{p{\paraWidth}|}{Anything goes. Default is from Colella \& Woodward} \\\hline
\end{tabular*}

\vspace{0.5cm}\noindent \begin{tabular*}{\tableWidth}{|c|l@{\extracolsep{\fill}}r|}
\hline
\multicolumn{1}{|p{\maxVarWidth}}{ppm\_eta2} & {\bf Scope:} restricted & REAL \\\hline
\multicolumn{3}{|p{\descWidth}|}{{\bf Description:}   {\em Eta2 for PPM shock detection}} \\
\hline{\bf Range} & &  {\bf Default:} 0.05 \\\multicolumn{1}{|p{\maxVarWidth}|}{\centering :} & \multicolumn{2}{p{\paraWidth}|}{Anything goes. Default is from Colella \& Woodward} \\\hline
\end{tabular*}

\vspace{0.5cm}\noindent \begin{tabular*}{\tableWidth}{|c|l@{\extracolsep{\fill}}r|}
\hline
\multicolumn{1}{|p{\maxVarWidth}}{ppm\_flatten} & {\bf Scope:} restricted & KEYWORD \\\hline
\multicolumn{3}{|p{\descWidth}|}{{\bf Description:}   {\em Which flattening procedure should the PPM solver use?}} \\
\hline{\bf Range} & &  {\bf Default:} stencil\_3 \\\multicolumn{1}{|p{\maxVarWidth}|}{\centering stencil\_3} & \multicolumn{2}{p{\paraWidth}|}{our flattening procedure, which requires only stencil 3 and which may work} \\\multicolumn{1}{|p{\maxVarWidth}|}{\centering stencil\_4} & \multicolumn{2}{p{\paraWidth}|}{original C\&W flattening procedure, which requires stencil 4} \\\hline
\end{tabular*}

\vspace{0.5cm}\noindent \begin{tabular*}{\tableWidth}{|c|l@{\extracolsep{\fill}}r|}
\hline
\multicolumn{1}{|p{\maxVarWidth}}{ppm\_k0} & {\bf Scope:} restricted & REAL \\\hline
\multicolumn{3}{|p{\descWidth}|}{{\bf Description:}   {\em K0 for PPM shock detection}} \\
\hline{\bf Range} & &  {\bf Default:} 0.2 \\\multicolumn{1}{|p{\maxVarWidth}|}{\centering :} & \multicolumn{2}{p{\paraWidth}|}{Anything goes. Default suggested by Colella \& Woodward is: (polytropic constant)/10.0} \\\hline
\end{tabular*}

\vspace{0.5cm}\noindent \begin{tabular*}{\tableWidth}{|c|l@{\extracolsep{\fill}}r|}
\hline
\multicolumn{1}{|p{\maxVarWidth}}{ppm\_mppm} & {\bf Scope:} restricted & INT \\\hline
\multicolumn{3}{|p{\descWidth}|}{{\bf Description:}   {\em Use modified (enhanced) ppm scheme}} \\
\hline{\bf Range} & &  {\bf Default:} (none) \\\multicolumn{1}{|p{\maxVarWidth}|}{\centering 0:1} & \multicolumn{2}{p{\paraWidth}|}{0 (off, default) or 1 (on)} \\\hline
\end{tabular*}

\vspace{0.5cm}\noindent \begin{tabular*}{\tableWidth}{|c|l@{\extracolsep{\fill}}r|}
\hline
\multicolumn{1}{|p{\maxVarWidth}}{ppm\_mppm\_debug\_eigenvalues} & {\bf Scope:} restricted & INT \\\hline
\multicolumn{3}{|p{\descWidth}|}{{\bf Description:}   {\em write eigenvalues into debug grid variables}} \\
\hline{\bf Range} & &  {\bf Default:} (none) \\\multicolumn{1}{|p{\maxVarWidth}|}{\centering 0:1} & \multicolumn{2}{p{\paraWidth}|}{0 (off, default) or 1 (on)} \\\hline
\end{tabular*}

\vspace{0.5cm}\noindent \begin{tabular*}{\tableWidth}{|c|l@{\extracolsep{\fill}}r|}
\hline
\multicolumn{1}{|p{\maxVarWidth}}{ppm\_omega1} & {\bf Scope:} restricted & REAL \\\hline
\multicolumn{3}{|p{\descWidth}|}{{\bf Description:}   {\em Omega1 for PPM zone flattening}} \\
\hline{\bf Range} & &  {\bf Default:} 0.75 \\\multicolumn{1}{|p{\maxVarWidth}|}{\centering :} & \multicolumn{2}{p{\paraWidth}|}{Anything goes. Default is from Colella \& Woodward} \\\hline
\end{tabular*}

\vspace{0.5cm}\noindent \begin{tabular*}{\tableWidth}{|c|l@{\extracolsep{\fill}}r|}
\hline
\multicolumn{1}{|p{\maxVarWidth}}{ppm\_omega2} & {\bf Scope:} restricted & REAL \\\hline
\multicolumn{3}{|p{\descWidth}|}{{\bf Description:}   {\em Omega2 for PPM zone flattening}} \\
\hline{\bf Range} & &  {\bf Default:} 10.0 \\\multicolumn{1}{|p{\maxVarWidth}|}{\centering :} & \multicolumn{2}{p{\paraWidth}|}{Anything goes. Default is from Colella \& Woodward} \\\hline
\end{tabular*}

\vspace{0.5cm}\noindent \begin{tabular*}{\tableWidth}{|c|l@{\extracolsep{\fill}}r|}
\hline
\multicolumn{1}{|p{\maxVarWidth}}{ppm\_omega\_tracer} & {\bf Scope:} restricted & REAL \\\hline
\multicolumn{3}{|p{\descWidth}|}{{\bf Description:}   {\em Omega for tracer PPM zone flattening}} \\
\hline{\bf Range} & &  {\bf Default:} 0.50 \\\multicolumn{1}{|p{\maxVarWidth}|}{\centering :} & \multicolumn{2}{p{\paraWidth}|}{Anything goes. Default is from Plewa \& Mueller} \\\hline
\end{tabular*}

\vspace{0.5cm}\noindent \begin{tabular*}{\tableWidth}{|c|l@{\extracolsep{\fill}}r|}
\hline
\multicolumn{1}{|p{\maxVarWidth}}{ppm\_small} & {\bf Scope:} restricted & REAL \\\hline
\multicolumn{3}{|p{\descWidth}|}{{\bf Description:}   {\em A floor used by PPM shock detection}} \\
\hline{\bf Range} & &  {\bf Default:} 1.e-7 \\\multicolumn{1}{|p{\maxVarWidth}|}{\centering 0.0:1.0} & \multicolumn{2}{p{\paraWidth}|}{In [0,1]} \\\hline
\end{tabular*}

\vspace{0.5cm}\noindent \begin{tabular*}{\tableWidth}{|c|l@{\extracolsep{\fill}}r|}
\hline
\multicolumn{1}{|p{\maxVarWidth}}{psidcspeed} & {\bf Scope:} restricted & KEYWORD \\\hline
\multicolumn{3}{|p{\descWidth}|}{{\bf Description:}   {\em Which speed to set for psidc}} \\
\hline{\bf Range} & &  {\bf Default:} light speed \\\multicolumn{1}{|p{\maxVarWidth}|}{\centering char speed} & \multicolumn{2}{p{\paraWidth}|}{Based on the characteristic speeds} \\\multicolumn{1}{|p{\maxVarWidth}|}{\centering light speed} & \multicolumn{2}{p{\paraWidth}|}{Set the characteristic speeds to speed of light} \\\multicolumn{1}{|p{\maxVarWidth}|}{\centering set speed} & \multicolumn{2}{p{\paraWidth}|}{"Manually set the characteristic speeds: [setcharmin,setcharm 
ax]"} \\\hline
\end{tabular*}

\vspace{0.5cm}\noindent \begin{tabular*}{\tableWidth}{|c|l@{\extracolsep{\fill}}r|}
\hline
\multicolumn{1}{|p{\maxVarWidth}}{recon\_method} & {\bf Scope:} restricted & KEYWORD \\\hline
\multicolumn{3}{|p{\descWidth}|}{{\bf Description:}   {\em Which reconstruction method to use}} \\
\hline{\bf Range} & &  {\bf Default:} tvd \\\multicolumn{1}{|p{\maxVarWidth}|}{\centering tvd} & \multicolumn{2}{p{\paraWidth}|}{Slope limited TVD} \\\multicolumn{1}{|p{\maxVarWidth}|}{\centering ppm} & \multicolumn{2}{p{\paraWidth}|}{PPM reconstruction} \\\multicolumn{1}{|p{\maxVarWidth}|}{\centering eno} & \multicolumn{2}{p{\paraWidth}|}{ENO reconstruction} \\\multicolumn{1}{|p{\maxVarWidth}|}{\centering weno} & \multicolumn{2}{p{\paraWidth}|}{WENO reconstruction} \\\multicolumn{1}{|p{\maxVarWidth}|}{\centering weno-z} & \multicolumn{2}{p{\paraWidth}|}{WENO-Z reconstruction} \\\multicolumn{1}{|p{\maxVarWidth}|}{\centering mp5} & \multicolumn{2}{p{\paraWidth}|}{MP5 reconstruction} \\\hline
\end{tabular*}

\vspace{0.5cm}\noindent \begin{tabular*}{\tableWidth}{|c|l@{\extracolsep{\fill}}r|}
\hline
\multicolumn{1}{|p{\maxVarWidth}}{recon\_vars} & {\bf Scope:} restricted & KEYWORD \\\hline
\multicolumn{3}{|p{\descWidth}|}{{\bf Description:}   {\em Which type of variables to reconstruct}} \\
\hline{\bf Range} & &  {\bf Default:} primitive \\\multicolumn{1}{|p{\maxVarWidth}|}{\centering primitive} & \multicolumn{2}{p{\paraWidth}|}{Reconstruct the primitive variables} \\\multicolumn{1}{|p{\maxVarWidth}|}{\centering conservative} & \multicolumn{2}{p{\paraWidth}|}{Reconstruct the conserved variables} \\\hline
\end{tabular*}

\vspace{0.5cm}\noindent \begin{tabular*}{\tableWidth}{|c|l@{\extracolsep{\fill}}r|}
\hline
\multicolumn{1}{|p{\maxVarWidth}}{reconstruct\_temper} & {\bf Scope:} restricted & BOOLEAN \\\hline
\multicolumn{3}{|p{\descWidth}|}{{\bf Description:}   {\em if set to true, temperature will be reconstructed}} \\
\hline & & {\bf Default:} no \\\hline
\end{tabular*}

\vspace{0.5cm}\noindent \begin{tabular*}{\tableWidth}{|c|l@{\extracolsep{\fill}}r|}
\hline
\multicolumn{1}{|p{\maxVarWidth}}{reconstruct\_wv} & {\bf Scope:} restricted & BOOLEAN \\\hline
\multicolumn{3}{|p{\descWidth}|}{{\bf Description:}   {\em Reconstruct the primitive velocity W\_Lorentz*vel, rather than just vel.}} \\
\hline & & {\bf Default:} no \\\hline
\end{tabular*}

\vspace{0.5cm}\noindent \begin{tabular*}{\tableWidth}{|c|l@{\extracolsep{\fill}}r|}
\hline
\multicolumn{1}{|p{\maxVarWidth}}{rho\_abs\_min} & {\bf Scope:} restricted & REAL \\\hline
\multicolumn{3}{|p{\descWidth}|}{{\bf Description:}   {\em A minimum rho below which evolution is turned off (atmosphere). If negative, the relative minimum will be used instead.}} \\
\hline{\bf Range} & &  {\bf Default:} -1.0 \\\multicolumn{1}{|p{\maxVarWidth}|}{\centering -1.0:} & \multicolumn{2}{p{\paraWidth}|}{} \\\hline
\end{tabular*}

\vspace{0.5cm}\noindent \begin{tabular*}{\tableWidth}{|c|l@{\extracolsep{\fill}}r|}
\hline
\multicolumn{1}{|p{\maxVarWidth}}{rho\_abs\_min\_after\_recovery} & {\bf Scope:} restricted & REAL \\\hline
\multicolumn{3}{|p{\descWidth}|}{{\bf Description:}   {\em A new absolute value for rho in the atmosphere. To be used after recovering. Unused if negative.}} \\
\hline{\bf Range} & &  {\bf Default:} -1.0 \\\multicolumn{1}{|p{\maxVarWidth}|}{\centering -1.0:} & \multicolumn{2}{p{\paraWidth}|}{} \\\hline
\end{tabular*}

\vspace{0.5cm}\noindent \begin{tabular*}{\tableWidth}{|c|l@{\extracolsep{\fill}}r|}
\hline
\multicolumn{1}{|p{\maxVarWidth}}{rho\_rel\_min} & {\bf Scope:} restricted & REAL \\\hline
\multicolumn{3}{|p{\descWidth}|}{{\bf Description:}   {\em A minimum relative rho (rhomin = centden * rho\_rel\_min) below which evolution is turned off (atmosphere). Only used if rho\_abs\_min {\textless} 0.0}} \\
\hline{\bf Range} & &  {\bf Default:} 1.e-9 \\\multicolumn{1}{|p{\maxVarWidth}|}{\centering 0:} & \multicolumn{2}{p{\paraWidth}|}{} \\\hline
\end{tabular*}

\vspace{0.5cm}\noindent \begin{tabular*}{\tableWidth}{|c|l@{\extracolsep{\fill}}r|}
\hline
\multicolumn{1}{|p{\maxVarWidth}}{riemann\_solver} & {\bf Scope:} restricted & KEYWORD \\\hline
\multicolumn{3}{|p{\descWidth}|}{{\bf Description:}   {\em Which Riemann solver to use}} \\
\hline{\bf Range} & &  {\bf Default:} HLLE \\\multicolumn{1}{|p{\maxVarWidth}|}{\centering Roe} & \multicolumn{2}{p{\paraWidth}|}{Standard Roe solver} \\\multicolumn{1}{|p{\maxVarWidth}|}{\centering Marquina} & \multicolumn{2}{p{\paraWidth}|}{Marquina flux} \\\multicolumn{1}{|p{\maxVarWidth}|}{\centering HLLE} & \multicolumn{2}{p{\paraWidth}|}{HLLE} \\\multicolumn{1}{|p{\maxVarWidth}|}{\centering HLLC} & \multicolumn{2}{p{\paraWidth}|}{HLLC} \\\multicolumn{1}{|p{\maxVarWidth}|}{\centering LLF} & \multicolumn{2}{p{\paraWidth}|}{Local Lax-Friedrichs (MHD only at the moment)} \\\hline
\end{tabular*}

\vspace{0.5cm}\noindent \begin{tabular*}{\tableWidth}{|c|l@{\extracolsep{\fill}}r|}
\hline
\multicolumn{1}{|p{\maxVarWidth}}{set\_trivial\_rp\_grid\_function} & {\bf Scope:} restricted & INT \\\hline
\multicolumn{3}{|p{\descWidth}|}{{\bf Description:}   {\em set gf for triv. rp (only for debugging)}} \\
\hline{\bf Range} & &  {\bf Default:} (none) \\\multicolumn{1}{|p{\maxVarWidth}|}{\centering 0:1} & \multicolumn{2}{p{\paraWidth}|}{0 for no (default), 1 for yes} \\\hline
\end{tabular*}

\vspace{0.5cm}\noindent \begin{tabular*}{\tableWidth}{|c|l@{\extracolsep{\fill}}r|}
\hline
\multicolumn{1}{|p{\maxVarWidth}}{setcharmax} & {\bf Scope:} restricted & REAL \\\hline
\multicolumn{3}{|p{\descWidth}|}{{\bf Description:}   {\em Maximum characteristic speed for psidc if psidcspeed is set}} \\
\hline{\bf Range} & &  {\bf Default:} 1.0 \\\multicolumn{1}{|p{\maxVarWidth}|}{\centering 0:1} & \multicolumn{2}{p{\paraWidth}|}{Any value smaller than speed of light} \\\hline
\end{tabular*}

\vspace{0.5cm}\noindent \begin{tabular*}{\tableWidth}{|c|l@{\extracolsep{\fill}}r|}
\hline
\multicolumn{1}{|p{\maxVarWidth}}{setcharmin} & {\bf Scope:} restricted & REAL \\\hline
\multicolumn{3}{|p{\descWidth}|}{{\bf Description:}   {\em Minimum characteristic speed for psidc if psidcspeed is set}} \\
\hline{\bf Range} & &  {\bf Default:} -1.0 \\\multicolumn{1}{|p{\maxVarWidth}|}{\centering -1:0} & \multicolumn{2}{p{\paraWidth}|}{Any value smaller than speed of light - sign should be negative} \\\hline
\end{tabular*}

\vspace{0.5cm}\noindent \begin{tabular*}{\tableWidth}{|c|l@{\extracolsep{\fill}}r|}
\hline
\multicolumn{1}{|p{\maxVarWidth}}{sources\_spatial\_order} & {\bf Scope:} restricted & INT \\\hline
\multicolumn{3}{|p{\descWidth}|}{{\bf Description:}   {\em Order of spatial differencing of the source terms}} \\
\hline{\bf Range} & &  {\bf Default:} 2 \\\multicolumn{1}{|p{\maxVarWidth}|}{\centering 2} & \multicolumn{2}{p{\paraWidth}|}{2nd order finite differencing} \\\multicolumn{1}{|p{\maxVarWidth}|}{\centering 4} & \multicolumn{2}{p{\paraWidth}|}{4th order finite differencing} \\\hline
\end{tabular*}

\vspace{0.5cm}\noindent \begin{tabular*}{\tableWidth}{|c|l@{\extracolsep{\fill}}r|}
\hline
\multicolumn{1}{|p{\maxVarWidth}}{sqrtdet\_thr} & {\bf Scope:} restricted & REAL \\\hline
\multicolumn{3}{|p{\descWidth}|}{{\bf Description:}   {\em Threshold to apply cons rescalings deep inside the horizon}} \\
\hline{\bf Range} & &  {\bf Default:} -1.0 \\\multicolumn{1}{|p{\maxVarWidth}|}{\centering 1.0:} & \multicolumn{2}{p{\paraWidth}|}{Larger values guarantees this sort of rescaling only deep inside the horizon} \\\multicolumn{1}{|p{\maxVarWidth}|}{\centering -1.0} & \multicolumn{2}{p{\paraWidth}|}{Do not apply limit} \\\hline
\end{tabular*}

\vspace{0.5cm}\noindent \begin{tabular*}{\tableWidth}{|c|l@{\extracolsep{\fill}}r|}
\hline
\multicolumn{1}{|p{\maxVarWidth}}{sync\_conserved\_only} & {\bf Scope:} restricted & BOOLEAN \\\hline
\multicolumn{3}{|p{\descWidth}|}{{\bf Description:}   {\em Only sync evolved conserved quantities during evolution.}} \\
\hline & & {\bf Default:} no \\\hline
\end{tabular*}

\vspace{0.5cm}\noindent \begin{tabular*}{\tableWidth}{|c|l@{\extracolsep{\fill}}r|}
\hline
\multicolumn{1}{|p{\maxVarWidth}}{tau\_rel\_min} & {\bf Scope:} restricted & REAL \\\hline
\multicolumn{3}{|p{\descWidth}|}{{\bf Description:}   {\em A minimum relative tau (taumin = maxtau(t=0) * tau\_rel\_min) below which tau is reschaled}} \\
\hline{\bf Range} & &  {\bf Default:} 1.e-10 \\\multicolumn{1}{|p{\maxVarWidth}|}{\centering 0:} & \multicolumn{2}{p{\paraWidth}|}{} \\\hline
\end{tabular*}

\vspace{0.5cm}\noindent \begin{tabular*}{\tableWidth}{|c|l@{\extracolsep{\fill}}r|}
\hline
\multicolumn{1}{|p{\maxVarWidth}}{tmunu\_damping\_radius\_max} & {\bf Scope:} restricted & REAL \\\hline
\multicolumn{3}{|p{\descWidth}|}{{\bf Description:}   {\em damping radius at which Tmunu becomes 0}} \\
\hline{\bf Range} & &  {\bf Default:} -1 \\\multicolumn{1}{|p{\maxVarWidth}|}{\centering -1} & \multicolumn{2}{p{\paraWidth}|}{damping switched off} \\\multicolumn{1}{|p{\maxVarWidth}|}{\centering 0:*} & \multicolumn{2}{p{\paraWidth}|}{greater than minimum radius above} \\\hline
\end{tabular*}

\vspace{0.5cm}\noindent \begin{tabular*}{\tableWidth}{|c|l@{\extracolsep{\fill}}r|}
\hline
\multicolumn{1}{|p{\maxVarWidth}}{tmunu\_damping\_radius\_min} & {\bf Scope:} restricted & REAL \\\hline
\multicolumn{3}{|p{\descWidth}|}{{\bf Description:}   {\em damping radius at which we start to damp with a tanh function}} \\
\hline{\bf Range} & &  {\bf Default:} -1 \\\multicolumn{1}{|p{\maxVarWidth}|}{\centering -1} & \multicolumn{2}{p{\paraWidth}|}{damping switched off} \\\multicolumn{1}{|p{\maxVarWidth}|}{\centering 0:*} & \multicolumn{2}{p{\paraWidth}|}{damping radius at which we start to damp} \\\hline
\end{tabular*}

\vspace{0.5cm}\noindent \begin{tabular*}{\tableWidth}{|c|l@{\extracolsep{\fill}}r|}
\hline
\multicolumn{1}{|p{\maxVarWidth}}{track\_divb} & {\bf Scope:} restricted & BOOLEAN \\\hline
\multicolumn{3}{|p{\descWidth}|}{{\bf Description:}   {\em Track the magnetic field constraint violations}} \\
\hline & & {\bf Default:} no \\\hline
\end{tabular*}

\vspace{0.5cm}\noindent \begin{tabular*}{\tableWidth}{|c|l@{\extracolsep{\fill}}r|}
\hline
\multicolumn{1}{|p{\maxVarWidth}}{transport\_constraints} & {\bf Scope:} restricted & BOOLEAN \\\hline
\multicolumn{3}{|p{\descWidth}|}{{\bf Description:}   {\em Use constraint transport for magnetic field}} \\
\hline & & {\bf Default:} no \\\hline
\end{tabular*}

\vspace{0.5cm}\noindent \begin{tabular*}{\tableWidth}{|c|l@{\extracolsep{\fill}}r|}
\hline
\multicolumn{1}{|p{\maxVarWidth}}{tvd\_limiter} & {\bf Scope:} restricted & KEYWORD \\\hline
\multicolumn{3}{|p{\descWidth}|}{{\bf Description:}   {\em Which slope limiter to use}} \\
\hline{\bf Range} & &  {\bf Default:} minmod \\\multicolumn{1}{|p{\maxVarWidth}|}{\centering minmod} & \multicolumn{2}{p{\paraWidth}|}{Minmod} \\\multicolumn{1}{|p{\maxVarWidth}|}{\centering vanleerMC2} & \multicolumn{2}{p{\paraWidth}|}{Van Leer MC - Luca} \\\multicolumn{1}{|p{\maxVarWidth}|}{\centering Superbee} & \multicolumn{2}{p{\paraWidth}|}{Superbee} \\\hline
\end{tabular*}

\vspace{0.5cm}\noindent \begin{tabular*}{\tableWidth}{|c|l@{\extracolsep{\fill}}r|}
\hline
\multicolumn{1}{|p{\maxVarWidth}}{use\_enhanced\_ppm} & {\bf Scope:} restricted & BOOLEAN \\\hline
\multicolumn{3}{|p{\descWidth}|}{{\bf Description:}   {\em Use the enhanced ppm reconstruction method proposed by Colella \& Sekora 2008 and McCorquodale \& Colella 2011}} \\
\hline & & {\bf Default:} no \\\hline
\end{tabular*}

\vspace{0.5cm}\noindent \begin{tabular*}{\tableWidth}{|c|l@{\extracolsep{\fill}}r|}
\hline
\multicolumn{1}{|p{\maxVarWidth}}{use\_evolution\_mask} & {\bf Scope:} restricted & KEYWORD \\\hline
\multicolumn{3}{|p{\descWidth}|}{{\bf Description:}   {\em Set this to 'always' to skip validity tests in regions where CarpetEvolutionMask::evolution\_mask vanishes.}} \\
\hline{\bf Range} & &  {\bf Default:} never \\\multicolumn{1}{|p{\maxVarWidth}|}{\centering always} & \multicolumn{2}{p{\paraWidth}|}{use the mask} \\\multicolumn{1}{|p{\maxVarWidth}|}{\centering auto} & \multicolumn{2}{p{\paraWidth}|}{check if CarpetEvolutionMask is active, then use the mask} \\\multicolumn{1}{|p{\maxVarWidth}|}{\centering never} & \multicolumn{2}{p{\paraWidth}|}{do not use the mask} \\\hline
\end{tabular*}

\vspace{0.5cm}\noindent \begin{tabular*}{\tableWidth}{|c|l@{\extracolsep{\fill}}r|}
\hline
\multicolumn{1}{|p{\maxVarWidth}}{use\_min\_tracer} & {\bf Scope:} restricted & BOOLEAN \\\hline
\multicolumn{3}{|p{\descWidth}|}{{\bf Description:}   {\em Should there be a floor on the tracer?}} \\
\hline & & {\bf Default:} no \\\hline
\end{tabular*}

\vspace{0.5cm}\noindent \begin{tabular*}{\tableWidth}{|c|l@{\extracolsep{\fill}}r|}
\hline
\multicolumn{1}{|p{\maxVarWidth}}{use\_mol\_slow\_multirate\_sector} & {\bf Scope:} restricted & BOOLEAN \\\hline
\multicolumn{3}{|p{\descWidth}|}{{\bf Description:}   {\em Whether to make use of MoL's slow multirate sector}} \\
\hline & & {\bf Default:} no \\\hline
\end{tabular*}

\vspace{0.5cm}\noindent \begin{tabular*}{\tableWidth}{|c|l@{\extracolsep{\fill}}r|}
\hline
\multicolumn{1}{|p{\maxVarWidth}}{use\_optimized\_ppm} & {\bf Scope:} restricted & BOOLEAN \\\hline
\multicolumn{3}{|p{\descWidth}|}{{\bf Description:}   {\em use C++ templated version of PPM. Experimental}} \\
\hline & & {\bf Default:} no \\\hline
\end{tabular*}

\vspace{0.5cm}\noindent \begin{tabular*}{\tableWidth}{|c|l@{\extracolsep{\fill}}r|}
\hline
\multicolumn{1}{|p{\maxVarWidth}}{use\_weighted\_fluxes} & {\bf Scope:} restricted & BOOLEAN \\\hline
\multicolumn{3}{|p{\descWidth}|}{{\bf Description:}   {\em Weight the flux terms by the cell surface areas}} \\
\hline & & {\bf Default:} no \\\hline
\end{tabular*}

\vspace{0.5cm}\noindent \begin{tabular*}{\tableWidth}{|c|l@{\extracolsep{\fill}}r|}
\hline
\multicolumn{1}{|p{\maxVarWidth}}{weno\_adaptive\_epsilon} & {\bf Scope:} restricted & BOOLEAN \\\hline
\multicolumn{3}{|p{\descWidth}|}{{\bf Description:}   {\em use modified smoothness indicators that take into account scale of solution (adaptive epsilon)}} \\
\hline & & {\bf Default:} yes \\\hline
\end{tabular*}

\vspace{0.5cm}\noindent \begin{tabular*}{\tableWidth}{|c|l@{\extracolsep{\fill}}r|}
\hline
\multicolumn{1}{|p{\maxVarWidth}}{weno\_eps} & {\bf Scope:} restricted & REAL \\\hline
\multicolumn{3}{|p{\descWidth}|}{{\bf Description:}   {\em WENO epsilon parameter. For WENO-z, 1e-40 is recommended}} \\
\hline{\bf Range} & &  {\bf Default:} 1e-26 \\\multicolumn{1}{|p{\maxVarWidth}|}{\centering 0:*} & \multicolumn{2}{p{\paraWidth}|}{small and positive} \\\hline
\end{tabular*}

\vspace{0.5cm}\noindent \begin{tabular*}{\tableWidth}{|c|l@{\extracolsep{\fill}}r|}
\hline
\multicolumn{1}{|p{\maxVarWidth}}{weno\_order} & {\bf Scope:} restricted & INT \\\hline
\multicolumn{3}{|p{\descWidth}|}{{\bf Description:}   {\em The order of accuracy of the WENO reconstruction}} \\
\hline{\bf Range} & &  {\bf Default:} 5 \\\multicolumn{1}{|p{\maxVarWidth}|}{\centering 5} & \multicolumn{2}{p{\paraWidth}|}{Fifth-order} \\\hline
\end{tabular*}

\vspace{0.5cm}\noindent \begin{tabular*}{\tableWidth}{|c|l@{\extracolsep{\fill}}r|}
\hline
\multicolumn{1}{|p{\maxVarWidth}}{wk\_atmosphere} & {\bf Scope:} restricted & BOOLEAN \\\hline
\multicolumn{3}{|p{\descWidth}|}{{\bf Description:}   {\em Use some of Wolfgang Kastauns atmosphere tricks}} \\
\hline & & {\bf Default:} no \\\hline
\end{tabular*}

\vspace{0.5cm}\noindent \begin{tabular*}{\tableWidth}{|c|l@{\extracolsep{\fill}}r|}
\hline
\multicolumn{1}{|p{\maxVarWidth}}{use\_mask} & {\bf Scope:} shared from SPACEMASK & BOOLEAN \\\hline
\end{tabular*}

\vspace{0.5cm}\parskip = 10pt 

\section{Interfaces} 


\parskip = 0pt

\vspace{3mm} \subsection*{General}

\noindent {\bf Implements}: 

grhydro
\vspace{2mm}

\noindent {\bf Inherits}: 

admbase

boundary

spacemask

tmunubase

hydrobase
\vspace{2mm}
\subsection*{Grid Variables}
\vspace{5mm}\subsubsection{PRIVATE GROUPS}

\vspace{5mm}

\begin{tabular*}{150mm}{|c|c@{\extracolsep{\fill}}|rl|} \hline 
~ {\bf Group Names} ~ & ~ {\bf Variable Names} ~  &{\bf Details} ~ & ~\\ 
\hline 
inlastmolpoststep & InLastMoLPostStep & compact & 0 \\ 
 &  & description & Flag to indicate if we are currently in the last MoL\_PostStep \\ 
 &  & dimensions & 0 \\ 
 &  & distribution & CONSTANT \\ 
 &  & group type & SCALAR \\ 
 &  & tags & checkpoint="no" \\ 
 &  & timelevels & 1 \\ 
 &  & variable type & INT \\ 
\hline 
execute\_mol\_step & execute\_MoL\_Step & compact & 0 \\ 
 &  & description & Flag indicating whether we use the slow sector of multirate RK time integration \\ 
 &  & dimensions & 0 \\ 
 &  & distribution & CONSTANT \\ 
 &  & group type & SCALAR \\ 
 &  & tags & checkpoint="no" \\ 
 &  & timelevels & 1 \\ 
 &  & variable type & INT \\ 
\hline 
execute\_mol\_poststep & execute\_MoL\_PostStep & compact & 0 \\ 
 &  & description & Flag indicating whether we use the slow sector of multirate RK time integration \\ 
 &  & dimensions & 0 \\ 
 &  & distribution & CONSTANT \\ 
 &  & group type & SCALAR \\ 
 &  & tags & checkpoint="no" \\ 
 &  & timelevels & 1 \\ 
 &  & variable type & INT \\ 
\hline 
grhydro\_con\_bext &  & compact & 0 \\ 
 & densplus & description & Conservative variables extended to the cell boundaries \\ 
 & sxplus & dimensions & 3 \\ 
 & syplus & distribution & DEFAULT \\ 
 & szplus & group type & GF \\ 
 & tauplus & tags & Prolongation="None" checkpoint="no" \\ 
 & densminus & timelevels & 1 \\ 
 & sxminus & variable type & REAL \\ 
\hline 
grhydro\_mhd\_con\_bext &  & compact & 0 \\ 
 & Bconsxplus & description & Conservative variables extended to the cell boundaries \\ 
 & Bconsyplus & dimensions & 3 \\ 
 & Bconszplus & distribution & DEFAULT \\ 
 & Bconsxminus & group type & GF \\ 
 & Bconsyminus & tags & Prolongation="None" checkpoint="no" \\ 
 & Bconszminus & timelevels & 1 \\ 
 &  & variable type & REAL \\ 
\hline 
grhydro\_mhd\_prim\_bext &  & compact & 0 \\ 
 & Bvecxplus & description & Primitive mhd variables extended to the cell boundaries \\ 
 & Bvecyplus & dimensions & 3 \\ 
 & Bveczplus & distribution & DEFAULT \\ 
 & Bvecxminus & group type & GF \\ 
 & Bvecyminus & tags & Prolongation="None" checkpoint="no" \\ 
 & Bveczminus & timelevels & 1 \\ 
 &  & variable type & REAL \\ 
\hline 
\end{tabular*} 



\vspace{5mm}
\vspace{5mm}

\begin{tabular*}{150mm}{|c|c@{\extracolsep{\fill}}|rl|} \hline 
~ {\bf Group Names} ~ & ~ {\bf Variable Names} ~  &{\bf Details} ~ & ~ \\ 
\hline 
grhydro\_avec\_bext &  & compact & 0 \\ 
 & Avecxplus & description & Vector potential extended to the cell boundaries \\ 
 & Avecyplus & dimensions & 3 \\ 
 & Aveczplus & distribution & DEFAULT \\ 
 & Avecxminus & group type & GF \\ 
 & Avecyminus & tags & Prolongation="None" checkpoint="no" \\ 
 & Aveczminus & timelevels & 1 \\ 
 &  & variable type & REAL \\ 
\hline 
grhydro\_aphi\_bext &  & compact & 0 \\ 
 & Aphiplus & description & Vector potential phi extended to the cell boundaries \\ 
 & Aphiminus & dimensions & 3 \\ 
 &  & distribution & DEFAULT \\ 
 &  & group type & GF \\ 
 &  & tags & Prolongation="None" checkpoint="no" \\ 
 &  & timelevels & 1 \\ 
 &  & variable type & REAL \\ 
\hline 
grhydro\_mhd\_psidc\_bext &  & compact & 0 \\ 
 & psidcplus & description & Divergence cleaning variable extended to the cell boundaries for diverence cleaning \\ 
 & psidcminus & dimensions & 3 \\ 
 &  & distribution & DEFAULT \\ 
 &  & group type & GF \\ 
 &  & tags & Prolongation="None" checkpoint="no" \\ 
 &  & timelevels & 1 \\ 
 &  & variable type & REAL \\ 
\hline 
grhydro\_entropy\_prim\_bext &  & compact & 0 \\ 
 & entropyplus & description & Primitive entropy extended to the cell boundaries \\ 
 & entropyminus & dimensions & 3 \\ 
 &  & distribution & DEFAULT \\ 
 &  & group type & GF \\ 
 &  & tags & Prolongation="None" checkpoint="no" \\ 
 &  & timelevels & 1 \\ 
 &  & variable type & REAL \\ 
\hline 
grhydro\_entropy\_con\_bext &  & compact & 0 \\ 
 & entropyconsplus & description & Conservative entropy extended to the cell boundaries \\ 
 & entropyconsminus & dimensions & 3 \\ 
 &  & distribution & DEFAULT \\ 
 &  & group type & GF \\ 
 &  & tags & Prolongation="None" checkpoint="no" \\ 
 &  & timelevels & 1 \\ 
 &  & variable type & REAL \\ 
\hline 
whichpsidcspeed & whichpsidcspeed & compact & 0 \\ 
 &  & description & Which speed to set for psidc? Set in ParamCheck \\ 
 &  & dimensions & 0 \\ 
 &  & distribution & CONSTANT \\ 
 &  & group type & SCALAR \\ 
 &  & tags & checkpoint="no" \\ 
 &  & timelevels & 1 \\ 
 &  & variable type & INT \\ 
\hline 
\end{tabular*} 



\vspace{5mm}
\vspace{5mm}

\begin{tabular*}{150mm}{|c|c@{\extracolsep{\fill}}|rl|} \hline 
~ {\bf Group Names} ~ & ~ {\bf Variable Names} ~  &{\bf Details} ~ & ~ \\ 
\hline 
grhydro\_coords &  & compact & 0 \\ 
 & GRHydro\_x & description & Coordinates to use with the comoving shift \\ 
 & GRHydro\_y & dimensions & 3 \\ 
 & GRHydro\_z & distribution & DEFAULT \\ 
 &  & group type & GF \\ 
 &  & timelevels & 3 \\ 
 &  & variable type & REAL \\ 
\hline 
grhydro\_coords\_rhs &  & compact & 0 \\ 
 & GRHydro\_x\_rhs & description & RHS for coordinates to use with the comoving shift \\ 
 & GRHydro\_y\_rhs & dimensions & 3 \\ 
 & GRHydro\_z\_rhs & distribution & DEFAULT \\ 
 &  & group type & GF \\ 
 &  & tags & Prolongation="None" \\ 
 &  & timelevels & 1 \\ 
 &  & variable type & REAL \\ 
\hline 
grhydro\_trivial\_rp\_gf\_group &  & compact & 0 \\ 
 & GRHydro\_trivial\_rp\_gf\_x & description & set gf for triv. rp (only for debugging) \\ 
 & GRHydro\_trivial\_rp\_gf\_y & dimensions & 3 \\ 
 & GRHydro\_trivial\_rp\_gf\_z & distribution & DEFAULT \\ 
 &  & group type & GF \\ 
 &  & tags & Prolongation="None" \\ 
 &  & timelevels & 1 \\ 
 &  & variable type & INT \\ 
\hline 
flux\_splitting &  & compact & 0 \\ 
 & densfplus & description & Fluxes for use in the flux splitting \\ 
 & densfminus & dimensions & 3 \\ 
 & sxfplus & distribution & DEFAULT \\ 
 & sxfminus & group type & GF \\ 
 & syfplus & tags & Prolongation="None" checkpoint="no" \\ 
 & syfminus & timelevels & 1 \\ 
 & szfplus & variable type & REAL \\ 
\hline 
fs\_alpha &  & compact & 0 \\ 
 & fs\_alpha1 & description & Maximum characteristic speeds for the flux splitting \\ 
 & fs\_alpha2 & dimensions & 0 \\ 
 & fs\_alpha3 & distribution & CONSTANT \\ 
 & fs\_alpha4 & group type & SCALAR \\ 
 & fs\_alpha5 & timelevels & 1 \\ 
 &  & variable type & REAL \\ 
\hline 
y\_e\_plus & Y\_e\_plus & compact & 0 \\ 
 &  & description & Plus state  for the electron fraction \\ 
 &  & dimensions & 3 \\ 
 &  & distribution & DEFAULT \\ 
 &  & group type & GF \\ 
 &  & tags & Prolongation="None" checkpoint="no" \\ 
 &  & timelevels & 1 \\ 
 &  & variable type & REAL \\ 
\hline 
\end{tabular*} 



\vspace{5mm}
\vspace{5mm}

\begin{tabular*}{150mm}{|c|c@{\extracolsep{\fill}}|rl|} \hline 
~ {\bf Group Names} ~ & ~ {\bf Variable Names} ~  &{\bf Details} ~ & ~ \\ 
\hline 
y\_e\_minus & Y\_e\_minus & compact & 0 \\ 
 &  & description & Minus state for the electron fraction \\ 
 &  & dimensions & 3 \\ 
 &  & distribution & DEFAULT \\ 
 &  & group type & GF \\ 
 &  & tags & Prolongation="None" checkpoint="no" \\ 
 &  & timelevels & 1 \\ 
 &  & variable type & REAL \\ 
\hline 
tempplus & tempplus & compact & 0 \\ 
 &  & description & Plus state  for the temperature \\ 
 &  & dimensions & 3 \\ 
 &  & distribution & DEFAULT \\ 
 &  & group type & GF \\ 
 &  & tags & Prolongation="None" checkpoint="no" \\ 
 &  & timelevels & 1 \\ 
 &  & variable type & REAL \\ 
\hline 
tempminus & tempminus & compact & 0 \\ 
 &  & description & Minus state for the temperature \\ 
 &  & dimensions & 3 \\ 
 &  & distribution & DEFAULT \\ 
 &  & group type & GF \\ 
 &  & tags & Prolongation="None" checkpoint="no" \\ 
 &  & timelevels & 1 \\ 
 &  & variable type & REAL \\ 
\hline 
grhydro\_tracer\_rhs &  & compact & 0 \\ 
 & cons\_tracerrhs & description & RHS for the tracer \\ 
 &  & dimensions & 3 \\ 
 &  & distribution & DEFAULT \\ 
 &  & group type & GF \\ 
 &  & tags & Prolongation="None" checkpoint="no" \\ 
 &  & timelevels & 1 \\ 
 &  & vararray\_size & number\_of\_tracers \\ 
 &  & variable type & REAL \\ 
\hline 
grhydro\_tracer\_flux &  & compact & 0 \\ 
 & cons\_tracerflux & description & Flux for the tracer \\ 
 &  & dimensions & 3 \\ 
 &  & distribution & DEFAULT \\ 
 &  & group type & GF \\ 
 &  & tags & Prolongation="None" checkpoint="no" \\ 
 &  & timelevels & 1 \\ 
 &  & vararray\_size & number\_of\_tracers \\ 
 &  & variable type & REAL \\ 
\hline 
grhydro\_tracer\_cons\_bext &  & compact & 0 \\ 
 & cons\_tracerplus & description & Cell boundary values for the tracer \\ 
 & cons\_tracerminus & dimensions & 3 \\ 
 &  & distribution & DEFAULT \\ 
 &  & group type & GF \\ 
 &  & tags & Prolongation="None" checkpoint="no" \\ 
 &  & timelevels & 1 \\ 
 &  & vararray\_size & number\_of\_tracers \\ 
 &  & variable type & REAL \\ 
\hline 
\end{tabular*} 



\vspace{5mm}
\vspace{5mm}

\begin{tabular*}{150mm}{|c|c@{\extracolsep{\fill}}|rl|} \hline 
~ {\bf Group Names} ~ & ~ {\bf Variable Names} ~  &{\bf Details} ~ & ~ \\ 
\hline 
grhydro\_tracer\_prim\_bext &  & compact & 0 \\ 
 & tracerplus & description & Primitive cell boundary values for the tracer \\ 
 & tracerminus & dimensions & 3 \\ 
 &  & distribution & DEFAULT \\ 
 &  & group type & GF \\ 
 &  & tags & Prolongation="None" checkpoint="no" \\ 
 &  & timelevels & 1 \\ 
 &  & vararray\_size & number\_of\_tracers \\ 
 &  & variable type & REAL \\ 
\hline 
grhydro\_tracer\_flux\_splitting &  & compact & 0 \\ 
 & tracerfplus & description & Flux splitting for the tracer \\ 
 & tracerfminus & dimensions & 3 \\ 
 &  & distribution & DEFAULT \\ 
 &  & group type & GF \\ 
 &  & tags & Prolongation="None" checkpoint="no" \\ 
 &  & timelevels & 1 \\ 
 &  & vararray\_size & number\_of\_tracers \\ 
 &  & variable type & REAL \\ 
\hline 
grhydro\_mppm\_eigenvalues &  & compact & 0 \\ 
 & GRHydro\_mppm\_eigenvalue\_x\_left & description & debug variable for flux eigenvalues in mppm \\ 
 & GRHydro\_mppm\_eigenvalue\_x\_right & dimensions & 3 \\ 
 & GRHydro\_mppm\_eigenvalue\_y\_left & distribution & DEFAULT \\ 
 & GRHydro\_mppm\_eigenvalue\_y\_right & group type & GF \\ 
 & GRHydro\_mppm\_eigenvalue\_z\_left & tags & Prolongation="None" checkpoint="no" \\ 
 & GRHydro\_mppm\_eigenvalue\_z\_right & timelevels & 1 \\ 
 & GRHydro\_mppm\_xwind & variable type & REAL \\ 
\hline 
particles &  & compact & 0 \\ 
 & particle\_x & description & Coordinates of particles to be tracked \\ 
 & particle\_y & dimensions & 1 \\ 
 & particle\_z & distribution & DEFAULT \\ 
 &  & ghostsize & 0 \\ 
 &  & group type & ARRAY \\ 
 &  & size & NUMBER\_OF\_PARTICLES \\ 
 &  & timelevels & 3 \\ 
 &  & variable type & REAL \\ 
\hline 
particle\_rhs &  & compact & 0 \\ 
 & particle\_x\_rhs & description & RHS functions for particles to be tracked \\ 
 & particle\_y\_rhs & dimensions & 1 \\ 
 & particle\_z\_rhs & distribution & DEFAULT \\ 
 &  & ghostsize & 0 \\ 
 &  & group type & ARRAY \\ 
 &  & size & NUMBER\_OF\_PARTICLES \\ 
 &  & timelevels & 1 \\ 
 &  & variable type & REAL \\ 
\hline 
particle\_arrays &  & compact & 0 \\ 
 & particle\_vx & description & Temporaries to hold interpolated values for particle tracking \\ 
 & particle\_vy & dimensions & 1 \\ 
 & particle\_vz & distribution & DEFAULT \\ 
 & particle\_alp & ghostsize & 0 \\ 
 & particle\_betax & group type & ARRAY \\ 
 & particle\_betay & size & NUMBER\_OF\_PARTICLES \\ 
 & particle\_betaz & tags & checkpoint="no" \\ 
 &  & timelevels & 1 \\ 
 &  & variable type & REAL \\ 
\hline 
\end{tabular*} 



\vspace{5mm}
\vspace{5mm}

\begin{tabular*}{150mm}{|c|c@{\extracolsep{\fill}}|rl|} \hline 
~ {\bf Group Names} ~ & ~ {\bf Variable Names} ~  &{\bf Details} ~ & ~ \\ 
\hline 
grhydro\_maxima\_location &  & compact & 0 \\ 
 & maxima\_i & description & The location (point index) of the maximum value of rho \\ 
 & maxima\_j & dimensions & 0 \\ 
 & maxima\_k & distribution & CONSTANT \\ 
 &  & group type & SCALAR \\ 
 &  & tags & checkpoint="no" \\ 
 &  & timelevels & 1 \\ 
 &  & variable type & REAL \\ 
\hline 
grhydro\_maxima\_iteration & GRHydro\_maxima\_iteration & compact & 0 \\ 
 &  & description & Iteration on which maximum was last set \\ 
 &  & dimensions & 0 \\ 
 &  & distribution & CONSTANT \\ 
 &  & group type & SCALAR \\ 
 &  & timelevels & 1 \\ 
 &  & variable type & INT \\ 
\hline 
grhydro\_maxima\_separation &  & compact & 0 \\ 
 & GRHydro\_separation & description & The distance between the centres (locations of maximum density) of a binary NS \\ 
 & GRHydro\_proper\_separation & dimensions & 0 \\ 
 &  & distribution & CONSTANT \\ 
 &  & group type & SCALAR \\ 
 &  & tags & checkpoint="no" \\ 
 &  & timelevels & 1 \\ 
 &  & variable type & REAL \\ 
\hline 
diffrho & DiffRho & compact & 0 \\ 
 &  & description & The first difference in rho \\ 
 &  & dimensions & 3 \\ 
 &  & distribution & DEFAULT \\ 
 &  & group type & GF \\ 
 &  & tags & Prolongation="None" checkpoint="no" \\ 
 &  & timelevels & 1 \\ 
 &  & variable type & REAL \\ 
\hline 
eos\_temps &  & compact & 0 \\ 
 & eos\_cs2\_p & description & Temporaries for the EOS calls \\ 
 & eos\_cs2\_m & dimensions & 3 \\ 
 & eos\_dpdeps\_p & distribution & DEFAULT \\ 
 & eos\_dpdeps\_m & group type & GF \\ 
 &  & tags & Prolongation="None" checkpoint="no" \\ 
 &  & timelevels & 1 \\ 
 &  & variable type & REAL \\ 
\hline 
roeaverage\_temps &  & compact & 0 \\ 
 & rho\_ave & description & Temporaries for the Roe solver \\ 
 & velx\_ave & dimensions & 3 \\ 
 & vely\_ave & distribution & DEFAULT \\ 
 & velz\_ave & group type & GF \\ 
 & eps\_ave & tags & Prolongation="None" checkpoint="no" \\ 
 & press\_ave & timelevels & 1 \\ 
 & eos\_cs2\_ave & variable type & REAL \\ 
\hline 
\end{tabular*} 



\vspace{5mm}
\vspace{5mm}

\begin{tabular*}{150mm}{|c|c@{\extracolsep{\fill}}|rl|} \hline 
~ {\bf Group Names} ~ & ~ {\bf Variable Names} ~  &{\bf Details} ~ & ~ \\ 
\hline 
con2prim\_temps &  & compact & 0 \\ 
 & press\_old & description & Temporaries for the conservative to primitive conversion \\ 
 & press\_new & dimensions & 3 \\ 
 & eos\_dpdeps\_temp & distribution & DEFAULT \\ 
 & eos\_dpdrho\_temp & group type & GF \\ 
 &  & tags & Prolongation="None" checkpoint="no" \\ 
 &  & timelevels & 1 \\ 
 &  & variable type & REAL \\ 
\hline 
h\_viscosity\_temps &  & compact & 0 \\ 
 & eos\_c & description & Temporaries for H viscosity \\ 
 &  & dimensions & 3 \\ 
 &  & distribution & DEFAULT \\ 
 &  & group type & GF \\ 
 &  & tags & Prolongation="None" checkpoint="no" \\ 
 &  & timelevels & 1 \\ 
 &  & variable type & REAL \\ 
\hline 
\end{tabular*} 


\vspace{5mm}\subsubsection{PUBLIC GROUPS}

\vspace{5mm}

\begin{tabular*}{150mm}{|c|c@{\extracolsep{\fill}}|rl|} \hline 
~ {\bf Group Names} ~ & ~ {\bf Variable Names} ~  &{\bf Details} ~ & ~\\ 
\hline 
grhydro\_eos\_scalars &  & compact & 0 \\ 
 & GRHydro\_eos\_handle & description & Handle number for EOS \\ 
 & GRHydro\_polytrope\_handle & dimensions & 0 \\ 
 &  & distribution & CONSTANT \\ 
 &  & group type & SCALAR \\ 
 &  & timelevels & 1 \\ 
 &  & variable type & INT \\ 
\hline 
grhydro\_minima &  & compact & 0 \\ 
 & GRHydro\_rho\_min & description & Atmosphere values \\ 
 & GRHydro\_tau\_min & dimensions & 0 \\ 
 &  & distribution & CONSTANT \\ 
 &  & group type & SCALAR \\ 
 &  & timelevels & 1 \\ 
 &  & variable type & REAL \\ 
\hline 
dens & dens & compact & 0 \\ 
 &  & description & generalized particle number \\ 
 &  & dimensions & 3 \\ 
 &  & distribution & DEFAULT \\ 
 &  & group type & GF \\ 
 &  & tags & ProlongationParameter="HydroBase::prolongation\_type" tensortypealias="Scalar" tensorweight=+1.0 jacobian="inverse\_jacobian" interpolator="matter" \\ 
 &  & timelevels & 3 \\ 
 &  & variable type & REAL \\ 
\hline 
tau & tau & compact & 0 \\ 
 &  & description & internal energy \\ 
 &  & dimensions & 3 \\ 
 &  & distribution & DEFAULT \\ 
 &  & group type & GF \\ 
 &  & tags & ProlongationParameter="HydroBase::prolongation\_type" tensortypealias="Scalar" tensorweight=+1.0 jacobian="inverse\_jacobian" interpolator="matter" \\ 
 &  & timelevels & 3 \\ 
 &  & variable type & REAL \\ 
\hline 
scon & scon & compact & 0 \\ 
 &  & description & generalized momenta \\ 
 &  & dimensions & 3 \\ 
 &  & distribution & DEFAULT \\ 
 &  & group type & GF \\ 
 &  & tags & ProlongationParameter="HydroBase::prolongation\_type" tensortypealias="D" tensorweight=+1.0 jacobian="inverse\_jacobian" interpolator="matter" \\ 
 &  & timelevels & 3 \\ 
 &  & vararray\_size & 3 \\ 
 &  & variable type & REAL \\ 
\hline 
bcons & Bcons & compact & 0 \\ 
 &  & description & B-field conservative variable \\ 
 &  & dimensions & 3 \\ 
 &  & distribution & DEFAULT \\ 
 &  & group type & GF \\ 
 &  & tags & ProlongationParameter="HydroBase::prolongation\_type" tensortypealias="U" tensorparity=-1 tensorweight=+1.0 jacobian="jacobian" interpolator="matter" \\ 
 &  & timelevels & 3 \\ 
 &  & vararray\_size & 3 \\ 
 &  & variable type & REAL \\ 
\hline 
\end{tabular*} 



\vspace{5mm}
\vspace{5mm}

\begin{tabular*}{150mm}{|c|c@{\extracolsep{\fill}}|rl|} \hline 
~ {\bf Group Names} ~ & ~ {\bf Variable Names} ~  &{\bf Details} ~ & ~ \\ 
\hline 
evec & Evec & compact & 0 \\ 
 &  & description & Electric field at edges \\ 
 &  & dimensions & 3 \\ 
 &  & distribution & DEFAULT \\ 
 &  & group type & GF \\ 
 &  & tags & ProlongationParameter="HydroBase::prolongation\_type" tensortypealias="U" tensorweight=+1.0 jacobian="jacobian" interpolator="matter" \\ 
 &  & timelevels & 1 \\ 
 &  & vararray\_size & 3 \\ 
 &  & variable type & REAL \\ 
\hline 
y\_e\_con & Y\_e\_con & compact & 0 \\ 
 &  & description & Conserved electron fraction \\ 
 &  & dimensions & 3 \\ 
 &  & distribution & DEFAULT \\ 
 &  & group type & GF \\ 
 &  & tags & ProlongationParameter="HydroBase::prolongation\_type" tensortypealias="Scalar" tensorweight=+1.0 jacobian="inverse\_jacobian" interpolator="matter" \\ 
 &  & timelevels & 3 \\ 
 &  & variable type & REAL \\ 
\hline 
entropycons & entropycons & compact & 0 \\ 
 &  & description & Conserved entropy density \\ 
 &  & dimensions & 3 \\ 
 &  & distribution & DEFAULT \\ 
 &  & group type & GF \\ 
 &  & tags & ProlongationParameter="HydroBase::prolongation\_type" tensortypealias="Scalar" tensorweight=+1.0 jacobian="inverse\_jacobian" interpolator="matter" \\ 
 &  & timelevels & 3 \\ 
 &  & variable type & REAL \\ 
\hline 
grhydro\_tracers &  & compact & 0 \\ 
 & tracer & description & Tracers \\ 
 &  & dimensions & 3 \\ 
 &  & distribution & DEFAULT \\ 
 &  & group type & GF \\ 
 &  & tags & ProlongationParameter="HydroBase::prolongation\_type" tensortypealias="Scalar" \\ 
 &  & timelevels & 3 \\ 
 &  & vararray\_size & number\_of\_tracers \\ 
 &  & variable type & REAL \\ 
\hline 
sdetg & sdetg & compact & 0 \\ 
 &  & description & Sqrt of the determinant of the 3-metric \\ 
 &  & dimensions & 3 \\ 
 &  & distribution & DEFAULT \\ 
 &  & group type & GF \\ 
 &  & tags & Prolongation="None" tensortypealias="Scalar" tensorweight=+1.0 interpolator="matter" checkpoint="no" \\ 
 &  & timelevels & 1 \\ 
 &  & variable type & REAL \\ 
\hline 
psidc & psidc & compact & 0 \\ 
 &  & description & Psi parameter for divergence cleaning \\ 
 &  & dimensions & 3 \\ 
 &  & distribution & DEFAULT \\ 
 &  & group type & GF \\ 
 &  & tags & ProlongationParameter="HydroBase::prolongation\_type" tensortypealias="Scalar" tensorweight=+1.0 tensorparity=-1 jacobian="inverse\_jacobian" interpolator="matter" \\ 
 &  & timelevels & 3 \\ 
 &  & variable type & REAL \\ 
\hline 
\end{tabular*} 



\vspace{5mm}
\vspace{5mm}

\begin{tabular*}{150mm}{|c|c@{\extracolsep{\fill}}|rl|} \hline 
~ {\bf Group Names} ~ & ~ {\bf Variable Names} ~  &{\bf Details} ~ & ~ \\ 
\hline 
densrhs & densrhs & compact & 0 \\ 
 &  & description & Update term for dens \\ 
 &  & dimensions & 3 \\ 
 &  & distribution & DEFAULT \\ 
 &  & group type & GF \\ 
 &  & tags & Prolongation="None" checkpoint="no" \\ 
 &  & timelevels & 1 \\ 
 &  & variable type & REAL \\ 
\hline 
taurhs & taurhs & compact & 0 \\ 
 &  & description & Update term for tau \\ 
 &  & dimensions & 3 \\ 
 &  & distribution & DEFAULT \\ 
 &  & group type & GF \\ 
 &  & tags & Prolongation="None" checkpoint="no" \\ 
 &  & timelevels & 1 \\ 
 &  & variable type & REAL \\ 
\hline 
srhs & srhs & compact & 0 \\ 
 &  & description & Update term for s \\ 
 &  & dimensions & 3 \\ 
 &  & distribution & DEFAULT \\ 
 &  & group type & GF \\ 
 &  & tags & Prolongation="None" checkpoint="no" \\ 
 &  & timelevels & 1 \\ 
 &  & vararray\_size & 3 \\ 
 &  & variable type & REAL \\ 
\hline 
bconsrhs & Bconsrhs & compact & 0 \\ 
 &  & description & Update term for Bcons \\ 
 &  & dimensions & 3 \\ 
 &  & distribution & DEFAULT \\ 
 &  & group type & GF \\ 
 &  & tags & Prolongation="None" checkpoint="no" \\ 
 &  & timelevels & 1 \\ 
 &  & vararray\_size & 3 \\ 
 &  & variable type & REAL \\ 
\hline 
avecrhs & Avecrhs & compact & 0 \\ 
 &  & description & Update term for Avec \\ 
 &  & dimensions & 3 \\ 
 &  & distribution & DEFAULT \\ 
 &  & group type & GF \\ 
 &  & tags & Prolongation="None" checkpoint="no" \\ 
 &  & timelevels & 1 \\ 
 &  & vararray\_size & 3 \\ 
 &  & variable type & REAL \\ 
\hline 
aphirhs & Aphirhs & compact & 0 \\ 
 &  & description & Update term for Aphi \\ 
 &  & dimensions & 3 \\ 
 &  & distribution & DEFAULT \\ 
 &  & group type & GF \\ 
 &  & tags & Prolongation="None" checkpoint="no" \\ 
 &  & timelevels & 1 \\ 
 &  & variable type & REAL \\ 
\hline 
\end{tabular*} 



\vspace{5mm}
\vspace{5mm}

\begin{tabular*}{150mm}{|c|c@{\extracolsep{\fill}}|rl|} \hline 
~ {\bf Group Names} ~ & ~ {\bf Variable Names} ~  &{\bf Details} ~ & ~ \\ 
\hline 
psidcrhs & psidcrhs & compact & 0 \\ 
 &  & description & Update term for psidc \\ 
 &  & dimensions & 3 \\ 
 &  & distribution & DEFAULT \\ 
 &  & group type & GF \\ 
 &  & tags & Prolongation="None" checkpoint="no" \\ 
 &  & timelevels & 1 \\ 
 &  & variable type & REAL \\ 
\hline 
entropyrhs & entropyrhs & compact & 0 \\ 
 &  & description & Update term for entropycons \\ 
 &  & dimensions & 3 \\ 
 &  & distribution & DEFAULT \\ 
 &  & group type & GF \\ 
 &  & tags & Prolongation="None" checkpoint="no" \\ 
 &  & timelevels & 1 \\ 
 &  & variable type & REAL \\ 
\hline 
divb & divB & compact & 0 \\ 
 &  & description & Magnetic field constraint \\ 
 &  & dimensions & 3 \\ 
 &  & distribution & DEFAULT \\ 
 &  & group type & GF \\ 
 &  & tags & Prolongation="Restrict" checkpoint="no" tensorparity=-1 \\ 
 &  & timelevels & 1 \\ 
 &  & variable type & REAL \\ 
\hline 
bcom & bcom & compact & 0 \\ 
 &  & description & b\^i: comoving contravariant magnetic field 4-vector spatial components \\ 
 &  & dimensions & 3 \\ 
 &  & distribution & DEFAULT \\ 
 &  & group type & GF \\ 
 &  & tags & ProlongationParameter="HydroBase::prolongation\_type" tensortypealias="U" tensorparity=-1 interpolator="matter" \\ 
 &  & timelevels & 3 \\ 
 &  & vararray\_size & 3 \\ 
 &  & variable type & REAL \\ 
\hline 
bcom0 & bcom0 & compact & 0 \\ 
 &  & description & b\^0 component of the comoving contravariant magnetic field 4-vector \\ 
 &  & dimensions & 3 \\ 
 &  & distribution & DEFAULT \\ 
 &  & group type & GF \\ 
 &  & tags & ProlongationParameter="HydroBase::prolongation\_type" tensortypealias="Scalar" interpolator="matter" \\ 
 &  & timelevels & 3 \\ 
 &  & variable type & REAL \\ 
\hline 
bcom\_sq & bcom\_sq & compact & 0 \\ 
 &  & description & half of magnectic pressure: contraction of b\_a b\^a  \\ 
 &  & dimensions & 3 \\ 
 &  & distribution & DEFAULT \\ 
 &  & group type & GF \\ 
 &  & tags & ProlongationParameter="HydroBase::prolongation\_type" tensortypealias="Scalar" interpolator="matter" \\ 
 &  & timelevels & 3 \\ 
 &  & variable type & REAL \\ 
\hline 
\end{tabular*} 



\vspace{5mm}
\vspace{5mm}

\begin{tabular*}{150mm}{|c|c@{\extracolsep{\fill}}|rl|} \hline 
~ {\bf Group Names} ~ & ~ {\bf Variable Names} ~  &{\bf Details} ~ & ~ \\ 
\hline 
lvel & lvel & compact & 0 \\ 
 &  & description & local velocity v\^i \\ 
 &  & dimensions & 3 \\ 
 &  & distribution & DEFAULT \\ 
 &  & group type & GF \\ 
 &  & tags & ProlongationParameter="HydroBase::prolongation\_type" tensortypealias="U" jacobian="jacobian" interpolator="matter" \\ 
 &  & timelevels & 3 \\ 
 &  & vararray\_size & 3 \\ 
 &  & variable type & REAL \\ 
\hline 
lbvec & lBvec & compact & 0 \\ 
 &  & description & local Magnetic field components B\^i \\ 
 &  & dimensions & 3 \\ 
 &  & distribution & DEFAULT \\ 
 &  & group type & GF \\ 
 &  & tags & ProlongationParameter="HydroBase::prolongation\_type" tensortypealias="U" jacobian="jacobian" tensorparity=-1 interpolator="matter" \\ 
 &  & timelevels & 3 \\ 
 &  & vararray\_size & 3 \\ 
 &  & variable type & REAL \\ 
\hline 
local\_metric &  & compact & 0 \\ 
 & gaa & description & local ADM metric g\_ij \\ 
 & gab & dimensions & 3 \\ 
 & gac & distribution & DEFAULT \\ 
 & gbb & group type & GF \\ 
 & gbc & tags & Prolongation="None" checkpoint="no" \\ 
 & gcc & timelevels & 3 \\ 
 &  & variable type & REAL \\ 
\hline 
local\_extrinsic\_curvature &  & compact & 0 \\ 
 & kaa & description & local extrinsic curvature K\_ij \\ 
 & kab & dimensions & 3 \\ 
 & kac & distribution & DEFAULT \\ 
 & kbb & group type & GF \\ 
 & kbc & tags & Prolongation="None" checkpoint="no" \\ 
 & kcc & timelevels & 1 \\ 
 &  & variable type & REAL \\ 
\hline 
local\_shift &  & compact & 0 \\ 
 & betaa & description & local ADM shift {\textbackslash}beta\^i \\ 
 & betab & dimensions & 3 \\ 
 & betac & distribution & DEFAULT \\ 
 &  & group type & GF \\ 
 &  & tags & Prolongation="None" checkpoint="no" \\ 
 &  & timelevels & 1 \\ 
 &  & variable type & REAL \\ 
\hline 
grhydro\_prim\_bext &  & compact & 0 \\ 
 & rhoplus & description & Primitive variables extended to the cell boundaries \\ 
 & velxplus & dimensions & 3 \\ 
 & velyplus & distribution & DEFAULT \\ 
 & velzplus & group type & GF \\ 
 & pressplus & tags & Prolongation="None" checkpoint="no" \\ 
 & epsplus & timelevels & 1 \\ 
 & w\_lorentzplus & variable type & REAL \\ 
\hline 
\end{tabular*} 



\vspace{5mm}
\vspace{5mm}

\begin{tabular*}{150mm}{|c|c@{\extracolsep{\fill}}|rl|} \hline 
~ {\bf Group Names} ~ & ~ {\bf Variable Names} ~  &{\bf Details} ~ & ~ \\ 
\hline 
grhydro\_scalars &  & compact & 0 \\ 
 & flux\_direction & description & Which direction are we taking the fluxes in and the offsets \\ 
 & xoffset & dimensions & 0 \\ 
 & yoffset & distribution & CONSTANT \\ 
 & zoffset & group type & SCALAR \\ 
 &  & tags & checkpoint="no" \\ 
 &  & timelevels & 1 \\ 
 &  & variable type & INT \\ 
\hline 
grhydro\_atmosphere\_mask &  & compact & 0 \\ 
 & atmosphere\_mask & description & Flags to say whether a point needs to be reset to the atmosphere \\ 
 &  & dimensions & 3 \\ 
 &  & distribution & DEFAULT \\ 
 &  & group type & GF \\ 
 &  & tags & Prolongation="None" \\ 
 &  & timelevels & 1 \\ 
 &  & variable type & INT \\ 
\hline 
grhydro\_atmosphere\_mask\_real &  & compact & 0 \\ 
 & atmosphere\_mask\_real & description & Flags to say whether a point needs to be reset to the atmosphere. This is sync'ed (and possibly interpolated)! \\ 
 &  & dimensions & 3 \\ 
 &  & distribution & DEFAULT \\ 
 &  & group type & GF \\ 
 &  & tags & Prolongation="sync" checkpoint="no" \\ 
 &  & timelevels & 1 \\ 
 &  & variable type & REAL \\ 
\hline 
grhydro\_atmosphere\_descriptors &  & compact & 0 \\ 
 & atmosphere\_field\_descriptor & dimensions & 0 \\ 
 & atmosphere\_atmosp\_descriptor & distribution & CONSTANT \\ 
 & atmosphere\_normal\_descriptor & group type & SCALAR \\ 
 &  & timelevels & 1 \\ 
 &  & variable type & INT \\ 
\hline 
grhydro\_cons\_tracers &  & compact & 0 \\ 
 & cons\_tracer & description & The conserved tracer variable \\ 
 &  & dimensions & 3 \\ 
 &  & distribution & DEFAULT \\ 
 &  & group type & GF \\ 
 &  & tags & ProlongationParameter="HydroBase::prolongation\_type" tensortypealias="Scalar" \\ 
 &  & timelevels & 3 \\ 
 &  & vararray\_size & number\_of\_tracers \\ 
 &  & variable type & REAL \\ 
\hline 
grhydro\_maxima\_position &  & compact & 0 \\ 
 & maxima\_x & description & The position (coordinate values) of the maximum value of rho \\ 
 & maxima\_y & dimensions & 0 \\ 
 & maxima\_z & distribution & CONSTANT \\ 
 & maximum\_density & group type & SCALAR \\ 
 &  & tags & checkpoint="no" \\ 
 &  & timelevels & 1 \\ 
 &  & variable type & REAL \\ 
\hline 
\end{tabular*} 



\vspace{5mm}
\vspace{5mm}

\begin{tabular*}{150mm}{|c|c@{\extracolsep{\fill}}|rl|} \hline 
~ {\bf Group Names} ~ & ~ {\bf Variable Names} ~  &{\bf Details} ~ & ~ \\ 
\hline 
maxrho\_global & maxrho\_global & compact & 0 \\ 
 &  & description & store the global maximum of rho \\ 
& ~ & description &  for refinment-grid steering \\ 
 &  & dimensions & 0 \\ 
 &  & distribution & CONSTANT \\ 
 &  & group type & SCALAR \\ 
 &  & tags & checkpoint="no" \\ 
 &  & timelevels & 1 \\ 
 &  & variable type & REAL \\ 
\hline 
grhydro\_c2p\_failed & GRHydro\_C2P\_failed & compact & 0 \\ 
 &  & description & Mask that stores the points where C2P has failed \\ 
 &  & dimensions & 3 \\ 
 &  & distribution & DEFAULT \\ 
 &  & group type & GF \\ 
 &  & tags & Prolongation="restrict" tensortypealias="Scalar" checkpoint="no" \\ 
 &  & timelevels & 1 \\ 
 &  & variable type & REAL \\ 
\hline 
grhydro\_fluxes &  & compact & 0 \\ 
 & densflux & description & Fluxes for each conserved variable \\ 
 & sxflux & dimensions & 3 \\ 
 & syflux & distribution & DEFAULT \\ 
 & szflux & group type & GF \\ 
 & tauflux & tags & Prolongation="None" checkpoint="no" \\ 
 &  & timelevels & 1 \\ 
 &  & variable type & REAL \\ 
\hline 
grhydro\_bfluxes &  & compact & 0 \\ 
 & Bconsxflux & description & Fluxes for each B-field variable \\ 
 & Bconsyflux & dimensions & 3 \\ 
 & Bconszflux & distribution & DEFAULT \\ 
 &  & group type & GF \\ 
 &  & tags & Prolongation="None" checkpoint="no" \\ 
 &  & timelevels & 1 \\ 
 &  & variable type & REAL \\ 
\hline 
grhydro\_psifluxes &  & compact & 0 \\ 
 & psidcflux & description & Fluxes for the divergence cleaning parameter \\ 
 &  & dimensions & 3 \\ 
 &  & distribution & DEFAULT \\ 
 &  & group type & GF \\ 
 &  & tags & Prolongation="None" checkpoint="no" \\ 
 &  & timelevels & 1 \\ 
 &  & variable type & REAL \\ 
\hline 
grhydro\_entropyfluxes &  & compact & 0 \\ 
 & entropyflux & description & Fluxes for the conserved entropy density \\ 
 &  & dimensions & 3 \\ 
 &  & distribution & DEFAULT \\ 
 &  & group type & GF \\ 
 &  & tags & Prolongation="None" checkpoint="no" \\ 
 &  & timelevels & 1 \\ 
 &  & variable type & REAL \\ 
\hline 
\end{tabular*} 



\vspace{5mm}
\vspace{5mm}

\begin{tabular*}{150mm}{|c|c@{\extracolsep{\fill}}|rl|} \hline 
~ {\bf Group Names} ~ & ~ {\bf Variable Names} ~  &{\bf Details} ~ & ~ \\ 
\hline 
grhydro\_avecfluxes &  & compact & 0 \\ 
 & Avecxflux & description & Fluxes for each Avec variable \\ 
 & Avecyflux & dimensions & 3 \\ 
 & Aveczflux & distribution & DEFAULT \\ 
 &  & group type & GF \\ 
 &  & tags & Prolongation="None" checkpoint="no" \\ 
 &  & timelevels & 1 \\ 
 &  & variable type & REAL \\ 
\hline 
grhydro\_aphifluxes &  & compact & 0 \\ 
 & Aphiflux & description & Fluxes for Aphi \\ 
 &  & dimensions & 3 \\ 
 &  & distribution & DEFAULT \\ 
 &  & group type & GF \\ 
 &  & tags & Prolongation="None" checkpoint="no" \\ 
 &  & timelevels & 1 \\ 
 &  & variable type & REAL \\ 
\hline 
evolve\_y\_e & evolve\_Y\_e & compact & 0 \\ 
 &  & description & Are we evolving Y\_e? Set in Paramcheck \\ 
 &  & dimensions & 0 \\ 
 &  & distribution & CONSTANT \\ 
 &  & group type & SCALAR \\ 
 &  & tags & checkpoint="no" \\ 
 &  & timelevels & 1 \\ 
 &  & variable type & INT \\ 
\hline 
evolve\_temper & evolve\_temper & compact & 0 \\ 
 &  & description & Are we evolving temperature? Set in Paramcheck \\ 
 &  & dimensions & 0 \\ 
 &  & distribution & CONSTANT \\ 
 &  & group type & SCALAR \\ 
 &  & tags & checkpoint="no" \\ 
 &  & timelevels & 1 \\ 
 &  & variable type & INT \\ 
\hline 
evolve\_entropy & evolve\_entropy & compact & 0 \\ 
 &  & description & Are we evolving entropy? Set in Paramcheck \\ 
 &  & dimensions & 0 \\ 
 &  & distribution & CONSTANT \\ 
 &  & group type & SCALAR \\ 
 &  & tags & checkpoint="no" \\ 
 &  & timelevels & 1 \\ 
 &  & variable type & INT \\ 
\hline 
evolve\_mhd & evolve\_MHD & compact & 0 \\ 
 &  & description & Are we doing MHD? Set in ParamCheck \\ 
 &  & dimensions & 0 \\ 
 &  & distribution & CONSTANT \\ 
 &  & group type & SCALAR \\ 
 &  & tags & checkpoint="no" \\ 
 &  & timelevels & 1 \\ 
 &  & variable type & INT \\ 
\hline 
\end{tabular*} 



\vspace{5mm}
\vspace{5mm}

\begin{tabular*}{150mm}{|c|c@{\extracolsep{\fill}}|rl|} \hline 
~ {\bf Group Names} ~ & ~ {\bf Variable Names} ~  &{\bf Details} ~ & ~ \\ 
\hline 
evolve\_lorenz\_gge & evolve\_Lorenz\_gge & compact & 0 \\ 
 &  & description & Are we evolving the Lorenz gauge? \\ 
 &  & dimensions & 0 \\ 
 &  & distribution & CONSTANT \\ 
 &  & group type & SCALAR \\ 
 &  & tags & checkpoint="no" \\ 
 &  & timelevels & 1 \\ 
 &  & variable type & INT \\ 
\hline 
grhydro\_reflevel & GRHydro\_reflevel & compact & 0 \\ 
 &  & description & Refinement level GRHydro is working on right now \\ 
 &  & dimensions & 0 \\ 
 &  & distribution & CONSTANT \\ 
 &  & group type & SCALAR \\ 
 &  & tags & checkpoint="no" \\ 
 &  & timelevels & 1 \\ 
 &  & variable type & INT \\ 
\hline 
y\_e\_con\_rhs & Y\_e\_con\_rhs & compact & 0 \\ 
 &  & description & RHS for the electron fraction \\ 
 &  & dimensions & 3 \\ 
 &  & distribution & DEFAULT \\ 
 &  & group type & GF \\ 
 &  & tags & Prolongation="None" checkpoint="no" \\ 
 &  & timelevels & 1 \\ 
 &  & variable type & REAL \\ 
\hline 
y\_e\_con\_flux & Y\_e\_con\_flux & compact & 0 \\ 
 &  & description & Flux for the electron fraction \\ 
 &  & dimensions & 3 \\ 
 &  & distribution & DEFAULT \\ 
 &  & group type & GF \\ 
 &  & tags & Prolongation="None" checkpoint="no" \\ 
 &  & timelevels & 1 \\ 
 &  & variable type & REAL \\ 
\hline 
\end{tabular*} 



\vspace{5mm}

\noindent {\bf Uses header}: 

SpaceMask.h

carpet.hh
\vspace{2mm}

\noindent {\bf Provides}: 



SpatialDet to 

UpperMet to 

Con2PrimPoly to 

Con2PrimGenM to 

Con2PrimGenMee to 

Con2PrimGen to 

Con2PrimPolyM to 

Prim2ConGen to 

Prim2ConPoly to 

Prim2ConGenM to 

Prim2ConGenM\_hot to 

Prim2ConPolyM to 
\vspace{2mm}\parskip = 10pt 

\section{Schedule} 


\parskip = 0pt


\noindent This section lists all the variables which are assigned storage by thorn EinsteinEvolve/GRHydro.  Storage can either last for the duration of the run ({\bf Always} means that if this thorn is activated storage will be assigned, {\bf Conditional} means that if this thorn is activated storage will be assigned for the duration of the run if some condition is met), or can be turned on for the duration of a schedule function.


\subsection*{Storage}

\hspace{5mm}

 \begin{tabular*}{160mm}{ll} 

{\bf Always:}& {\bf Conditional:} \\ 
 execute\_MoL\_Step &  execute\_MoL\_PostStep\\ 
 evolve\_MHD &  HydroBase::temperature[timelevels]\\ 
 evolve\_Y\_e &  HydroBase::entropy[timelevels]\\ 
 evolve\_temper &  HydroBase::Bvec[timelevels]\\ 
 GRHydro\_reflevel &  GRHydro::Bcons[timelevels]\\ 
 InLastMoLPostStep &  Bconsrhs\\ 
 sdetg &  psidc[timelevels]\\ 
 densrhs &  HydroBase::Bvec[timelevels]\\ 
 taurhs &  HydroBase::Avec[timelevels]\\ 
 srhs &  HydroBase::Aphi[timelevels]\\ 
 GRHydro\_C2P\_failed[1] &  dens[timelevels]\\ 
~ &  Bconsrhs\\ 
~ &  psidcrhs\\ 
~ &  tau[timelevels]\\ 
~ &  whichpsidcspeed\\ 
~ &  divB\\ 
~ &  GRHydro::bcom[timelevels]\\ 
~ &  GRHydro::bcom0[timelevels]\\ 
~ &  GRHydro::bcom\_sq[timelevels]\\ 
~ &  Avecrhs\\ 
~ &  evolve\_Lorenz\_gge\\ 
~ &  Aphirhs\\ 
~ &  divB\\ 
~ &  GRHydro::bcom[timelevels]\\ 
~ &  scon[timelevels]\\ 
~ &  GRHydro::bcom0[timelevels]\\ 
~ &  GRHydro::bcom\_sq[timelevels]\\ 
~ &  HydroBase::entropy[timelevels]\\ 
~ &  GRHydro::entropycons[timelevels]\\ 
~ &  entropyrhs\\ 
~ &  evolve\_MHD\\ 
~ &  evolve\_Y\_e\\ 
~ &  evolve\_temper\\ 
~ &  evolve\_entropy\\ 
~ &  GRHydro\_reflevel\\ 
~ &  particles[timelevels]\\ 
~ &  InLastMoLPostStep\\ 
~ &  densrhs\\ 
~ &  taurhs\\ 
~ &  srhs\\ 
~ &  GRHydro\_eos\_scalars\\ 
~ &  GRHydro\_minima\\ 
~ &  GRHydro\_scalars\\ 
~ &  particle\_rhs\\ 
~ &  particle\_arrays\\ 
~ &  GRHydro\_tracers[timelevels]\\ 
~ &  Y\_e\_con[timelevels]\\ 
~ &  GRHydro\_cons\_tracers[timelevels]\\ 
~ &  GRHydro\_tracer\_rhs\\ 
~ &  Evec\\ 
~ &  ADMBase::metric[timelevels] ADMBase::curv[timelevels]\\ 
~ &  ADMBase::lapse[timelevels]\\ 
~ &  ADMBase::shift[timelevels]\\ 
~ &  GRHydro\_coords[timelevels]\\ 
~ &  GRHydro\_coords\_rhs\\ 
~ &  lvel[timelevels]\\ 
~ &  lBvec[timelevels]\\ 
~ &  Y\_e\_con\_rhs Y\_e\_con\_flux Y\_e\_plus Y\_e\_minus\\ 
~ &  local\_metric[timelevels]\\ 
~ &  local\_extrinsic\_curvature\\ 
~ &  local\_shift\\ 
~ &  H\_viscosity\_temps\\ 
~ &  GRHydro\_atmosphere\_mask\\ 
~ &  GRHydro\_atmosphere\_mask\_real\\ 
~ &  GRHydro\_atmosphere\_descriptors\\ 
~ &  fs\_alpha\\ 
~ &  GRHydro\_tracers[timelevels]\\ 
~ &  GRHydro\_cons\_tracers[timelevels]\\ 
~ &  HydroBase::Y\_e[timelevels]\\ 
~ &  GRHydro\_tracer\_rhs\\ 
~ &  GRHydro\_trivial\_rp\_gf\_group\\ 
~ &  GRHydro\_mppm\_eigenvalues\\ 
~ &  tempplus tempminus\\ 
~ & ~\\ 
\end{tabular*} 


\subsection*{Scheduled Functions}
\vspace{5mm}

\noindent {\bf CCTK\_BASEGRID} 

\hspace{5mm} grhydro\_reset\_execution\_flags 

\hspace{5mm}{\it initially set execution flags to 'yeah, execute'! } 


\hspace{5mm}

 \begin{tabular*}{160mm}{cll} 
~ & Language:  & c \\ 
~ & Options:  & global \\ 
~ & Type:  & function \\ 
\end{tabular*} 


\vspace{5mm}

\noindent {\bf MoL\_Step}   (conditional) 

\hspace{5mm} grhydro\_set\_execution\_flags 

\hspace{5mm}{\it check if we need to execute rhs / post-step calculation } 


\hspace{5mm}

 \begin{tabular*}{160mm}{cll} 
~ & After:  & mol\_decrementcounter \\ 
~ & Before:  & mol\_poststepmodify \\ 
~ & Language:  & c \\ 
~ & Options:  & level \\ 
~ & Type:  & function \\ 
\end{tabular*} 


\vspace{5mm}

\noindent {\bf GRHydroRHS}   (conditional) 

\hspace{5mm} grhydro\_evolvecoords 

\hspace{5mm}{\it evolve the coordinates for the comoving shift } 


\hspace{5mm}

 \begin{tabular*}{160mm}{cll} 
~ & Language:  & fortran \\ 
~ & Type:  & function \\ 
\end{tabular*} 


\vspace{5mm}

\noindent {\bf HydroBase\_Prim2ConInitial}   (conditional) 

\hspace{5mm} primitive2conservativecells 

\hspace{5mm}{\it convert initial data given in primive variables to conserved variables } 


\hspace{5mm}

 \begin{tabular*}{160mm}{cll} 
~ & Language:  & fortran \\ 
~ & Type:  & function \\ 
\end{tabular*} 


\vspace{5mm}

\noindent {\bf HydroBase\_Con2Prim}   (conditional) 

\hspace{5mm} conservative2primitivepolytypem 

\hspace{5mm}{\it convert back to primitive variables (polytype) - mhd version } 


\hspace{5mm}

 \begin{tabular*}{160mm}{cll} 
~ & If:  & grhydro::execute\_mol\_poststep \\ 
~ & Language:  & fortran \\ 
~ & Type:  & function \\ 
\end{tabular*} 


\vspace{5mm}

\noindent {\bf HydroBase\_Prim2ConInitial}   (conditional) 

\hspace{5mm} primitive2conservativepolycellsm 

\hspace{5mm}{\it convert initial data given in primive variables to conserved variables - mhd version } 


\hspace{5mm}

 \begin{tabular*}{160mm}{cll} 
~ & Language:  & fortran \\ 
~ & Type:  & function \\ 
\end{tabular*} 


\vspace{5mm}

\noindent {\bf HydroBase\_Con2Prim}   (conditional) 

\hspace{5mm} conservative2primitivepolytypeam 

\hspace{5mm}{\it convert back to primitive variables (polytype) - mhd with avec version } 


\hspace{5mm}

 \begin{tabular*}{160mm}{cll} 
~ & If:  & grhydro::execute\_mol\_poststep \\ 
~ & Language:  & fortran \\ 
~ & Type:  & function \\ 
\end{tabular*} 


\vspace{5mm}

\noindent {\bf HydroBase\_Prim2ConInitial}   (conditional) 

\hspace{5mm} primitive2conservativepolycellsam 

\hspace{5mm}{\it convert initial data given in primive variables to conserved variables - mhd with avec version } 


\hspace{5mm}

 \begin{tabular*}{160mm}{cll} 
~ & Language:  & fortran \\ 
~ & Type:  & function \\ 
\end{tabular*} 


\vspace{5mm}

\noindent {\bf HydroBase\_Con2Prim}   (conditional) 

\hspace{5mm} conservative2primitivepolytype 

\hspace{5mm}{\it convert back to primitive variables (polytype) } 


\hspace{5mm}

 \begin{tabular*}{160mm}{cll} 
~ & If:  & grhydro::execute\_mol\_poststep \\ 
~ & Language:  & fortran \\ 
~ & Type:  & function \\ 
\end{tabular*} 


\vspace{5mm}

\noindent {\bf HydroBase\_Prim2ConInitial}   (conditional) 

\hspace{5mm} primitive2conservativepolycells 

\hspace{5mm}{\it convert initial data given in primive variables to conserved variables } 


\hspace{5mm}

 \begin{tabular*}{160mm}{cll} 
~ & Language:  & fortran \\ 
~ & Type:  & function \\ 
\end{tabular*} 


\vspace{5mm}

\noindent {\bf HydroBase\_Boundaries}   (conditional) 

\hspace{5mm} do\_grhydro\_boundaries 

\hspace{5mm}{\it grhydro boundary conditions group } 


\hspace{5mm}

 \begin{tabular*}{160mm}{cll} 
~ & Type:  & group \\ 
\end{tabular*} 


\vspace{5mm}

\noindent {\bf HydroBase\_PostStep}   (conditional) 

\hspace{5mm} grhydro\_atmospheremaskboundaries 

\hspace{5mm}{\it apply boundary conditions to primitives } 


\hspace{5mm}

 \begin{tabular*}{160mm}{cll} 
~ & Before:  & hydrobase\_boundaries \\ 
~& ~ &grhydro\_primitiveinitialguessesboundaries\\ 
~ & Type:  & group \\ 
\end{tabular*} 


\vspace{5mm}

\noindent {\bf GRHydro\_AtmosphereMaskBoundaries}   (conditional) 

\hspace{5mm} grhydro\_selectatmospheremaskboundaries 

\hspace{5mm}{\it select atmosphere mask for boundary conditions } 


\hspace{5mm}

 \begin{tabular*}{160mm}{cll} 
~ & Language:  & fortran \\ 
~ & Options:  & level \\ 
~ & Sync:  & grhydro\_atmosphere\_mask\_real \\ 
~ & Type:  & function \\ 
\end{tabular*} 


\vspace{5mm}

\noindent {\bf HydroBase\_PostStep}   (conditional) 

\hspace{5mm} grhydro\_comovingshift 

\hspace{5mm}{\it comoving shift } 


\hspace{5mm}

 \begin{tabular*}{160mm}{cll} 
~ & After:  & hydrobase\_con2prim \\ 
~ & Language:  & fortran \\ 
~ & Type:  & function \\ 
\end{tabular*} 


\vspace{5mm}

\noindent {\bf GRHydro\_AtmosphereMaskBoundaries}   (conditional) 

\hspace{5mm} applybcs 

\hspace{5mm}{\it apply boundary conditions to real-valued atmosphere mask } 


\hspace{5mm}

 \begin{tabular*}{160mm}{cll} 
~ & After:  & grhydro\_selectatmospheremaskboundaries \\ 
~ & Type:  & group \\ 
\end{tabular*} 


\vspace{5mm}

\noindent {\bf HydroBase\_Select\_Boundaries}   (conditional) 

\hspace{5mm} grhydro\_boundaries 

\hspace{5mm}{\it select grhydro boundary conditions } 


\hspace{5mm}

 \begin{tabular*}{160mm}{cll} 
~ & If:  & grhydro::execute\_mol\_poststep \\ 
~ & Language:  & fortran \\ 
~ & Options:  & level \\ 
~ & Sync:  & dens \\ 
~& ~ &tau\\ 
~& ~ &scon\\ 
~& ~ &hydrobase::w\_lorentz\\ 
~& ~ &hydrobase::rho\\ 
~& ~ &hydrobase::press\\ 
~& ~ &hydrobase::eps\\ 
~& ~ &hydrobase::vel\\ 
~& ~ &bcons\\ 
~& ~ &entropycons\\ 
~& ~ &hydrobase::bvec\\ 
~& ~ &psidc\\ 
~& ~ &grhydro\_cons\_tracers\\ 
~& ~ &grhydro\_tracers\\ 
~& ~ &hydrobase::temperature\\ 
~& ~ &hydrobase::entropy\\ 
~& ~ &hydrobase::y\_e\\ 
~& ~ &y\_e\_con\\ 
~& ~ &lvel\\ 
~& ~ &lbvec\\ 
~ & Type:  & function \\ 
\end{tabular*} 


\vspace{5mm}

\noindent {\bf HydroBase\_Select\_Boundaries}   (conditional) 

\hspace{5mm} grhydro\_boundaries 

\hspace{5mm}{\it select grhydro boundary conditions } 


\hspace{5mm}

 \begin{tabular*}{160mm}{cll} 
~ & If:  & grhydro::execute\_mol\_poststep \\ 
~ & Language:  & fortran \\ 
~ & Options:  & level \\ 
~ & Sync:  & dens \\ 
~& ~ &tau\\ 
~& ~ &scon\\ 
~& ~ &bcons\\ 
~& ~ &entropycons\\ 
~& ~ &psidc\\ 
~& ~ &grhydro\_cons\_tracers\\ 
~& ~ &y\_e\_con\\ 
~ & Type:  & function \\ 
\end{tabular*} 


\vspace{5mm}

\noindent {\bf CCTK\_POSTREGRID}   (conditional) 

\hspace{5mm} grhydro\_primitiveboundaries 

\hspace{5mm}{\it apply boundary conditions to all primitives } 


\hspace{5mm}

 \begin{tabular*}{160mm}{cll} 
~ & Before:  & mol\_poststep \\ 
~ & Type:  & group \\ 
\end{tabular*} 


\vspace{5mm}

\noindent {\bf CCTK\_POSTREGRIDINITIAL}   (conditional) 

\hspace{5mm} grhydro\_primitiveboundaries 

\hspace{5mm}{\it apply boundary conditions to all primitives } 


\hspace{5mm}

 \begin{tabular*}{160mm}{cll} 
~ & Before:  & mol\_poststep \\ 
~ & Type:  & group \\ 
\end{tabular*} 


\vspace{5mm}

\noindent {\bf HydroBase\_PostStep}   (conditional) 

\hspace{5mm} grhydro\_primitiveinitialguessesboundaries 

\hspace{5mm}{\it apply boundary conditions to those primitives used as initial guesses } 


\hspace{5mm}

 \begin{tabular*}{160mm}{cll} 
~ & Before:  & hydrobase\_boundaries \\ 
~ & If:  & grhydro::inlastmolpoststep \\ 
~ & Type:  & group \\ 
\end{tabular*} 


\vspace{5mm}

\noindent {\bf GRHydro\_PrimitiveInitialGuessesBoundaries}   (conditional) 

\hspace{5mm} grhydro\_selectprimitiveinitialguessesboundaries 

\hspace{5mm}{\it select initial guess primitive variables for boudary conditions } 


\hspace{5mm}

 \begin{tabular*}{160mm}{cll} 
~ & Language:  & fortran \\ 
~ & Options:  & level \\ 
~ & Sync:  & hydrobase::press \\ 
~& ~ &hydrobase::rho\\ 
~& ~ &hydrobase::eps\\ 
~& ~ &hydrobase::temperature\\ 
~& ~ &lvel\\ 
~& ~ &lbvec\\ 
~ & Type:  & function \\ 
\end{tabular*} 


\vspace{5mm}

\noindent {\bf GRHydro\_PrimitiveBoundaries}   (conditional) 

\hspace{5mm} grhydro\_selectprimitiveboundaries 

\hspace{5mm}{\it select primitive variables for boundary conditions } 


\hspace{5mm}

 \begin{tabular*}{160mm}{cll} 
~ & Language:  & fortran \\ 
~ & Options:  & level \\ 
~ & Sync:  & hydrobase::press \\ 
~& ~ &hydrobase::entropy\\ 
~& ~ &hydrobase::y\_e\\ 
~& ~ &grhydro\_tracers\\ 
~& ~ &hydrobase::rho\\ 
~& ~ &hydrobase::eps\\ 
~& ~ &hydrobase::temperature\\ 
~& ~ &lvel\\ 
~& ~ &lbvec\\ 
~ & Type:  & function \\ 
\end{tabular*} 


\vspace{5mm}

\noindent {\bf GRHydro\_PrimitiveInitialGuessesBoundaries}   (conditional) 

\hspace{5mm} grhydro\_selectprimitiveinitialguessesboundaries 

\hspace{5mm}{\it select initial guess primitive variables for boudary conditions } 


\hspace{5mm}

 \begin{tabular*}{160mm}{cll} 
~ & Language:  & fortran \\ 
~ & Options:  & level \\ 
~ & Sync:  & hydrobase::press \\ 
~& ~ &hydrobase::rho\\ 
~& ~ &hydrobase::eps\\ 
~& ~ &hydrobase::vel\\ 
~& ~ &hydrobase::bvec\\ 
~& ~ &hydrobase::temperature\\ 
~ & Type:  & function \\ 
\end{tabular*} 


\vspace{5mm}

\noindent {\bf GRHydro\_PrimitiveBoundaries}   (conditional) 

\hspace{5mm} grhydro\_selectprimitiveboundaries 

\hspace{5mm}{\it select primitive variables for boundary conditions } 


\hspace{5mm}

 \begin{tabular*}{160mm}{cll} 
~ & Language:  & fortran \\ 
~ & Options:  & level \\ 
~ & Sync:  & hydrobase::press \\ 
~& ~ &hydrobase::entropy\\ 
~& ~ &hydrobase::y\_e\\ 
~& ~ &grhydro\_tracers\\ 
~& ~ &hydrobase::rho\\ 
~& ~ &hydrobase::eps\\ 
~& ~ &hydrobase::vel\\ 
~& ~ &hydrobase::bvec\\ 
~& ~ &hydrobase::temperature\\ 
~ & Type:  & function \\ 
\end{tabular*} 


\vspace{5mm}

\noindent {\bf CCTK\_STARTUP}   (conditional) 

\hspace{5mm} grhydro\_startup 

\hspace{5mm}{\it startup banner } 


\hspace{5mm}

 \begin{tabular*}{160mm}{cll} 
~ & Language:  & fortran \\ 
~ & Type:  & function \\ 
\end{tabular*} 


\vspace{5mm}

\noindent {\bf GRHydro\_PrimitiveInitialGuessesBoundaries}   (conditional) 

\hspace{5mm} applybcs 

\hspace{5mm}{\it apply boundary conditions to initial guess primitive variables } 


\hspace{5mm}

 \begin{tabular*}{160mm}{cll} 
~ & After:  & grhydro\_selectprimitiveinitialguessesboundaries \\ 
~ & Type:  & group \\ 
\end{tabular*} 


\vspace{5mm}

\noindent {\bf GRHydro\_PrimitiveBoundaries}   (conditional) 

\hspace{5mm} applybcs 

\hspace{5mm}{\it apply boundary conditions to all primitive variables } 


\hspace{5mm}

 \begin{tabular*}{160mm}{cll} 
~ & After:  & grhydro\_selectprimitiveboundaries \\ 
~ & Type:  & group \\ 
\end{tabular*} 


\vspace{5mm}

\noindent {\bf MoL\_PostStep}   (conditional) 

\hspace{5mm} grhydro\_setlastmolpoststep 

\hspace{5mm}{\it set grid scalar inlastmolpoststep if this is the last mol poststep call } 


\hspace{5mm}

 \begin{tabular*}{160mm}{cll} 
~ & Language:  & c \\ 
~ & Options:  & level \\ 
~ & Type:  & function \\ 
\end{tabular*} 


\vspace{5mm}

\noindent {\bf MoL\_Step}   (conditional) 

\hspace{5mm} grhydro\_clearlastmolpoststep 

\hspace{5mm}{\it reset inlastmolpoststep to zero } 


\hspace{5mm}

 \begin{tabular*}{160mm}{cll} 
~ & After:  & mol\_poststep \\ 
~ & Language:  & c \\ 
~ & Options:  & level \\ 
~ & Type:  & function \\ 
\end{tabular*} 


\vspace{5mm}

\noindent {\bf CCTK\_WRAGH}   (conditional) 

\hspace{5mm} grhydro\_clearlastmolpoststep 

\hspace{5mm}{\it initialize inlastmolpoststep to zero } 


\hspace{5mm}

 \begin{tabular*}{160mm}{cll} 
~ & Language:  & c \\ 
~ & Options:  & global-early \\ 
~ & Type:  & function \\ 
\end{tabular*} 


\vspace{5mm}

\noindent {\bf HydroBase\_PostStep}   (conditional) 

\hspace{5mm} grhydropostsyncatmospheremask 

\hspace{5mm}{\it set integer atmosphere mask from synchronized real atmosphere mask } 


\hspace{5mm}

 \begin{tabular*}{160mm}{cll} 
~ & After:  & grhydro\_atmospheremaskboundaries \\ 
~ & Language:  & fortran \\ 
~ & Type:  & function \\ 
\end{tabular*} 


\vspace{5mm}

\noindent {\bf HydroBase\_PostStep}   (conditional) 

\hspace{5mm} grhydro\_atmosphereresetm 

\hspace{5mm}{\it reset the atmosphere - mhd version } 


\hspace{5mm}

 \begin{tabular*}{160mm}{cll} 
~ & After:  & grhydropostsyncatmospheremask \\ 
~ & Before:  & hydrobase\_boundaries \\ 
~& ~ &grhydro\_primitiveinitialguessesboundaries\\ 
~ & If:  & grhydro::inlastmolpoststep \\ 
~ & Language:  & fortran \\ 
~ & Type:  & function \\ 
\end{tabular*} 


\vspace{5mm}

\noindent {\bf HydroBase\_PostStep}   (conditional) 

\hspace{5mm} grhydro\_atmosphereresetam 

\hspace{5mm}{\it reset the atmosphere - mhd with avec version } 


\hspace{5mm}

 \begin{tabular*}{160mm}{cll} 
~ & After:  & grhydropostsyncatmospheremask \\ 
~ & Before:  & hydrobase\_boundaries \\ 
~& ~ &grhydro\_primitiveinitialguessesboundaries\\ 
~ & If:  & grhydro::inlastmolpoststep \\ 
~ & Language:  & fortran \\ 
~ & Type:  & function \\ 
\end{tabular*} 


\vspace{5mm}

\noindent {\bf HydroBase\_PostStep}   (conditional) 

\hspace{5mm} grhydro\_atmospherereset 

\hspace{5mm}{\it reset the atmosphere } 


\hspace{5mm}

 \begin{tabular*}{160mm}{cll} 
~ & After:  & grhydropostsyncatmospheremask \\ 
~ & Before:  & hydrobase\_boundaries \\ 
~& ~ &grhydro\_primitiveinitialguessesboundaries\\ 
~ & If:  & grhydro::inlastmolpoststep \\ 
~ & Language:  & fortran \\ 
~ & Type:  & function \\ 
\end{tabular*} 


\vspace{5mm}

\noindent {\bf CCTK\_ANALYSIS}   (conditional) 

\hspace{5mm} grhydro\_check\_rho\_minimum 

\hspace{5mm}{\it check whether somewhere rho(i,j,k) {\textless} grhydro\_rho\_min and produce a warning } 


\hspace{5mm}

 \begin{tabular*}{160mm}{cll} 
~ & Language:  & fortran \\ 
~ & Triggers:  & hydrobase::rho \\ 
~& ~ &hydrobase::press\\ 
~& ~ &hydrobase::eps\\ 
~ & Type:  & function \\ 
\end{tabular*} 


\vspace{5mm}

\noindent {\bf CCTK\_STARTUP}   (conditional) 

\hspace{5mm} grhydro\_registermask 

\hspace{5mm}{\it register the hydro masks } 


\hspace{5mm}

 \begin{tabular*}{160mm}{cll} 
~ & Language:  & c \\ 
~ & Type:  & function \\ 
\end{tabular*} 


\vspace{5mm}

\noindent {\bf CCTK\_ANALYSIS}   (conditional) 

\hspace{5mm} grhydro\_refinementlevel 

\hspace{5mm}{\it calculate current refinement level } 


\hspace{5mm}

 \begin{tabular*}{160mm}{cll} 
~ & Before:  & grhydro\_check\_rho\_minimum \\ 
~ & Language:  & fortran \\ 
~ & Triggers:  & hydrobase::rho \\ 
~& ~ &hydrobase::press\\ 
~& ~ &hydrobase::eps\\ 
~ & Type:  & function \\ 
\end{tabular*} 


\vspace{5mm}

\noindent {\bf CCTK\_BASEGRID} 

\hspace{5mm} reset\_grhydro\_c2p\_failed 

\hspace{5mm}{\it initialise the mask function that contains the points where c2p has failed (at basegrid) } 


\hspace{5mm}

 \begin{tabular*}{160mm}{cll} 
~ & Language:  & fortran \\ 
~ & Type:  & function \\ 
\end{tabular*} 


\vspace{5mm}

\noindent {\bf CCTK\_PRESTEP} 

\hspace{5mm} reset\_grhydro\_c2p\_failed 

\hspace{5mm}{\it reset the mask function that contains the points where c2p has failed (at prestep) } 


\hspace{5mm}

 \begin{tabular*}{160mm}{cll} 
~ & Language:  & fortran \\ 
~ & Type:  & function \\ 
\end{tabular*} 


\vspace{5mm}

\noindent {\bf CCTK\_EVOL} 

\hspace{5mm} sync\_grhydro\_c2p\_failed 

\hspace{5mm}{\it syncronise the mask function that contains the points where c2p has failed } 


\hspace{5mm}

 \begin{tabular*}{160mm}{cll} 
~ & After:  & mol\_evolution \\ 
~ & Language:  & fortran \\ 
~ & Sync:  & grhydro\_c2p\_failed \\ 
~ & Type:  & function \\ 
\end{tabular*} 


\vspace{5mm}

\noindent {\bf CCTK\_POSTSTEP} 

\hspace{5mm} check\_grhydro\_c2p\_failed 

\hspace{5mm}{\it check the mask function that contains the points where c2p has failed and report an error in case a failure is found } 


\hspace{5mm}

 \begin{tabular*}{160mm}{cll} 
~ & After:  & grhydro\_refinementlevel \\ 
~ & Language:  & fortran \\ 
~ & Type:  & function \\ 
\end{tabular*} 


\vspace{5mm}

\noindent {\bf AddToTmunu}   (conditional) 

\hspace{5mm} grhydro\_tmunum 

\hspace{5mm}{\it compute the energy-momentum tensor - mhd version } 


\hspace{5mm}

 \begin{tabular*}{160mm}{cll} 
~ & Language:  & fortran \\ 
~ & Type:  & function \\ 
\end{tabular*} 


\vspace{5mm}

\noindent {\bf HydroBase\_PostStep}   (conditional) 

\hspace{5mm} grhydro\_calcbcom 

\hspace{5mm}{\it compute comoving magnetic field, pressure, etc... } 


\hspace{5mm}

 \begin{tabular*}{160mm}{cll} 
~ & After:  & grhydrotransformprimtoglobalbasis \\ 
~ & Language:  & fortran \\ 
~ & Type:  & function \\ 
\end{tabular*} 


\vspace{5mm}

\noindent {\bf AddToTmunu}   (conditional) 

\hspace{5mm} grhydro\_tmunu 

\hspace{5mm}{\it compute the energy-momentum tensor } 


\hspace{5mm}

 \begin{tabular*}{160mm}{cll} 
~ & Language:  & fortran \\ 
~ & Type:  & function \\ 
\end{tabular*} 


\vspace{5mm}

\noindent {\bf CCTK\_POSTPOSTINITIAL}   (conditional) 

\hspace{5mm} settmunu 

\hspace{5mm}{\it calculate the stress-energy tensor } 


\hspace{5mm}

 \begin{tabular*}{160mm}{cll} 
~ & After:  & con2prim \\ 
~ & Before:  & admconstraintsgroup \\ 
~ & Type:  & group \\ 
\end{tabular*} 


\vspace{5mm}

\noindent {\bf MoL\_PseudoEvolution}   (conditional) 

\hspace{5mm} grhydroanalysis 

\hspace{5mm}{\it calculate analysis quantities } 


\hspace{5mm}

 \begin{tabular*}{160mm}{cll} 
~ & Type:  & group \\ 
\end{tabular*} 


\vspace{5mm}

\noindent {\bf CCTK\_PARAMCHECK}   (conditional) 

\hspace{5mm} grhydro\_paramcheck 

\hspace{5mm}{\it check parameters } 


\hspace{5mm}

 \begin{tabular*}{160mm}{cll} 
~ & Language:  & fortran \\ 
~ & Type:  & function \\ 
\end{tabular*} 


\vspace{5mm}

\noindent {\bf GRHydroAnalysis}   (conditional) 

\hspace{5mm} grhydro\_calcdivb 

\hspace{5mm}{\it calculate divb } 


\hspace{5mm}

 \begin{tabular*}{160mm}{cll} 
~ & After:  & grhydro\_analysis\_init \\ 
~ & Language:  & fortran \\ 
~ & Type:  & function \\ 
\end{tabular*} 


\vspace{5mm}

\noindent {\bf GRHydroAnalysis}   (conditional) 

\hspace{5mm} applybcs 

\hspace{5mm}{\it apply boundary conditions to divb } 


\hspace{5mm}

 \begin{tabular*}{160mm}{cll} 
~ & After:  & grhydro\_calcdivb \\ 
~ & Type:  & group \\ 
\end{tabular*} 


\vspace{5mm}

\noindent {\bf MoL\_PostStepModify}   (conditional) 

\hspace{5mm} constrainsconto1d 

\hspace{5mm}{\it constrain conserved fluid velocity to radial direction } 


\hspace{5mm}

 \begin{tabular*}{160mm}{cll} 
~ & Language:  & fortran \\ 
~ & Options:  & local \\ 
~ & Type:  & function \\ 
\end{tabular*} 


\vspace{5mm}

\noindent {\bf GRHydroRHS}   (conditional) 

\hspace{5mm} h\_viscosity\_calc\_cs\_cc 

\hspace{5mm}{\it compute local temporaries for h viscosity - c++ version } 


\hspace{5mm}

 \begin{tabular*}{160mm}{cll} 
~ & Before:  & fluxterms \\ 
~ & Language:  & c \\ 
~ & Options:  & local \\ 
~ & Type:  & function \\ 
\end{tabular*} 


\vspace{5mm}

\noindent {\bf GRHydroRHS}   (conditional) 

\hspace{5mm} h\_viscosity 

\hspace{5mm}{\it compute local temporaries for h viscosity } 


\hspace{5mm}

 \begin{tabular*}{160mm}{cll} 
~ & Before:  & fluxterms \\ 
~ & Language:  & fortran \\ 
~ & Options:  & local \\ 
~ & Type:  & function \\ 
\end{tabular*} 


\vspace{5mm}

\noindent {\bf CCTK\_BASEGRID}   (conditional) 

\hspace{5mm} grhydro\_check\_jacobian\_state 

\hspace{5mm}{\it test state of jacobians } 


\hspace{5mm}

 \begin{tabular*}{160mm}{cll} 
~ & After:  & tmunubase\_setstressenergystate \\ 
~& ~ &coordinates\_setglobalcoords\_group\\ 
~ & Language:  & c \\ 
~ & Options:  & global \\ 
~ & Type:  & function \\ 
\end{tabular*} 


\vspace{5mm}

\noindent {\bf CCTK\_BASEGRID}   (conditional) 

\hspace{5mm} grhydro\_initsymbound 

\hspace{5mm}{\it schedule symmetries and check shift state } 


\hspace{5mm}

 \begin{tabular*}{160mm}{cll} 
~ & After:  & admbase\_setshiftstateon \\ 
~& ~ &admbase\_setshiftstateoff\\ 
~ & Language:  & fortran \\ 
~ & Type:  & function \\ 
\end{tabular*} 


\vspace{5mm}

\noindent {\bf CCTK\_INITIAL}   (conditional) 

\hspace{5mm} grhydro\_eoshandle 

\hspace{5mm}{\it set the eos number } 


\hspace{5mm}

 \begin{tabular*}{160mm}{cll} 
~ & Before:  & hydrobase\_initial \\ 
~ & Language:  & c \\ 
~ & Options:  & global \\ 
~ & Type:  & function \\ 
\end{tabular*} 


\vspace{5mm}

\noindent {\bf CCTK\_POST\_RECOVER\_VARIABLES}   (conditional) 

\hspace{5mm} grhydro\_eoshandle 

\hspace{5mm}{\it set the eos number } 


\hspace{5mm}

 \begin{tabular*}{160mm}{cll} 
~ & Language:  & c \\ 
~ & Options:  & global \\ 
~ & Type:  & function \\ 
\end{tabular*} 


\vspace{5mm}

\noindent {\bf CCTK\_INITIAL}   (conditional) 

\hspace{5mm} grhydro\_rho\_minima\_setup 

\hspace{5mm}{\it set up minimum for the rest-mass density in the atmosphere (before intial data) } 


\hspace{5mm}

 \begin{tabular*}{160mm}{cll} 
~ & Before:  & hydrobase\_initial \\ 
~ & Language:  & fortran \\ 
~ & Type:  & function \\ 
\end{tabular*} 


\vspace{5mm}

\noindent {\bf MoL\_StartStep}   (conditional) 

\hspace{5mm} grhydro\_reset\_execution\_flags 

\hspace{5mm}{\it reset execution flags to 'yeah, execute'! } 


\hspace{5mm}

 \begin{tabular*}{160mm}{cll} 
~ & After:  & mol\_setcounter \\ 
~ & Language:  & c \\ 
~ & Options:  & level \\ 
~ & Type:  & function \\ 
\end{tabular*} 


\vspace{5mm}

\noindent {\bf CCTK\_POST\_RECOVER\_VARIABLES}   (conditional) 

\hspace{5mm} grhydro\_change\_rho\_minimum\_at\_recovery 

\hspace{5mm}{\it set up minimum for the rest-mass density in the atmosphere (before intial data) } 


\hspace{5mm}

 \begin{tabular*}{160mm}{cll} 
~ & Language:  & fortran \\ 
~ & Type:  & function \\ 
\end{tabular*} 


\vspace{5mm}

\noindent {\bf CCTK\_POSTINITIAL}   (conditional) 

\hspace{5mm} grhydro\_rho\_minima\_setup\_final\_pugh 

\hspace{5mm}{\it set the value of the rest-mass density of the atmosphere which will be used during the evolution (pugh) } 


\hspace{5mm}

 \begin{tabular*}{160mm}{cll} 
~ & Before:  & mol\_poststepmodify \\ 
~& ~ &mol\_poststep\\ 
~ & Language:  & c \\ 
~ & Type:  & function \\ 
\end{tabular*} 


\vspace{5mm}

\noindent {\bf CCTK\_POSTPOSTINITIAL}   (conditional) 

\hspace{5mm} grhydro\_rho\_minima\_setup\_final 

\hspace{5mm}{\it set the value of the rest-mass density of the atmosphere which will be used during the evolution } 


\hspace{5mm}

 \begin{tabular*}{160mm}{cll} 
~ & Before:  & con2prim \\ 
~ & Language:  & c \\ 
~ & Type:  & function \\ 
\end{tabular*} 


\vspace{5mm}

\noindent {\bf CCTK\_INITIAL}   (conditional) 

\hspace{5mm} grhydro\_sqrtspatialdeterminant 

\hspace{5mm}{\it calculate sdetg } 


\hspace{5mm}

 \begin{tabular*}{160mm}{cll} 
~ & After:  & hydrobase\_initial \\ 
~& ~ &grhydrotransformadmtolocalbasis\\ 
~& ~ &admbase\_postinitial\\ 
~ & Before:  & hydrobase\_prim2coninitial \\ 
~ & Language:  & fortran \\ 
~ & Type:  & function \\ 
\end{tabular*} 


\vspace{5mm}

\noindent {\bf CCTK\_POSTINITIAL}   (conditional) 

\hspace{5mm} grhydro\_initialatmosphereresetm 

\hspace{5mm}{\it use mask to enforce atmosphere at initial time } 


\hspace{5mm}

 \begin{tabular*}{160mm}{cll} 
~ & After:  & grhydro\_rho\_minima\_setup\_final\_pugh \\ 
~ & Before:  & mol\_poststepmodify \\ 
~& ~ &mol\_poststep\\ 
~ & Language:  & fortran \\ 
~ & Type:  & function \\ 
\end{tabular*} 


\vspace{5mm}

\noindent {\bf CCTK\_POSTPOSTINITIAL}   (conditional) 

\hspace{5mm} grhydro\_initialatmosphereresetm 

\hspace{5mm}{\it use mask to enforce atmosphere at initial time } 


\hspace{5mm}

 \begin{tabular*}{160mm}{cll} 
~ & After:  & grhydro\_rho\_minima\_setup\_final \\ 
~ & Before:  & con2prim \\ 
~ & Language:  & fortran \\ 
~ & Type:  & function \\ 
\end{tabular*} 


\vspace{5mm}

\noindent {\bf CCTK\_POSTINITIAL}   (conditional) 

\hspace{5mm} grhydro\_initialatmosphereresetam 

\hspace{5mm}{\it use mask to enforce atmosphere at initial time } 


\hspace{5mm}

 \begin{tabular*}{160mm}{cll} 
~ & After:  & grhydro\_rho\_minima\_setup\_final\_pugh \\ 
~ & Before:  & mol\_poststepmodify \\ 
~& ~ &mol\_poststep\\ 
~ & Language:  & fortran \\ 
~ & Type:  & function \\ 
\end{tabular*} 


\vspace{5mm}

\noindent {\bf CCTK\_POSTPOSTINITIAL}   (conditional) 

\hspace{5mm} grhydro\_initialatmosphereresetam 

\hspace{5mm}{\it use mask to enforce atmosphere at initial time } 


\hspace{5mm}

 \begin{tabular*}{160mm}{cll} 
~ & After:  & grhydro\_rho\_minima\_setup\_final \\ 
~ & Before:  & con2prim \\ 
~ & Language:  & fortran \\ 
~ & Type:  & function \\ 
\end{tabular*} 


\vspace{5mm}

\noindent {\bf CCTK\_POSTINITIAL}   (conditional) 

\hspace{5mm} grhydro\_initialatmospherereset 

\hspace{5mm}{\it use mask to enforce atmosphere at initial time } 


\hspace{5mm}

 \begin{tabular*}{160mm}{cll} 
~ & After:  & grhydro\_rho\_minima\_setup\_final\_pugh \\ 
~ & Before:  & mol\_poststepmodify \\ 
~& ~ &mol\_poststep\\ 
~ & Language:  & fortran \\ 
~ & Type:  & function \\ 
\end{tabular*} 


\vspace{5mm}

\noindent {\bf CCTK\_POSTPOSTINITIAL}   (conditional) 

\hspace{5mm} grhydro\_initialatmospherereset 

\hspace{5mm}{\it use mask to enforce atmosphere at initial time } 


\hspace{5mm}

 \begin{tabular*}{160mm}{cll} 
~ & After:  & grhydro\_rho\_minima\_setup\_final \\ 
~ & Before:  & con2prim \\ 
~ & Language:  & fortran \\ 
~ & Type:  & function \\ 
\end{tabular*} 


\vspace{5mm}

\noindent {\bf MoL\_Evolution}   (conditional) 

\hspace{5mm} grhydro\_reset\_execution\_flags 

\hspace{5mm}{\it reset execution flags to 'yeah, execute'! } 


\hspace{5mm}

 \begin{tabular*}{160mm}{cll} 
~ & After:  & mol\_finishloop \\ 
~ & Language:  & c \\ 
~ & Options:  & level \\ 
~ & Type:  & function \\ 
\end{tabular*} 


\vspace{5mm}

\noindent {\bf CCTK\_POSTPOSTINITIAL} 

\hspace{5mm} admconstraintsgroup 

\hspace{5mm}{\it evaluate adm constraints, and perform symmetry boundary conditions } 


\hspace{5mm}

 \begin{tabular*}{160mm}{cll} 
~ & Type:  & group \\ 
\end{tabular*} 


\vspace{5mm}

\noindent {\bf CCTK\_BASEGRID}   (conditional) 

\hspace{5mm} grhydro\_enosetup 

\hspace{5mm}{\it coefficients for eno reconstruction } 


\hspace{5mm}

 \begin{tabular*}{160mm}{cll} 
~ & Language:  & fortran \\ 
~ & Options:  & global \\ 
~ & Type:  & function \\ 
\end{tabular*} 


\vspace{5mm}

\noindent {\bf CCTK\_TERMINATE}   (conditional) 

\hspace{5mm} grhydro\_enoshutdown 

\hspace{5mm}{\it deallocate eno coefficients } 


\hspace{5mm}

 \begin{tabular*}{160mm}{cll} 
~ & Before:  & driver\_terminate \\ 
~ & Language:  & fortran \\ 
~ & Options:  & global \\ 
~ & Type:  & function \\ 
\end{tabular*} 


\vspace{5mm}

\noindent {\bf CCTK\_BASEGRID}   (conditional) 

\hspace{5mm} grhydro\_wenosetup 

\hspace{5mm}{\it coefficients for weno reconstruction } 


\hspace{5mm}

 \begin{tabular*}{160mm}{cll} 
~ & Language:  & fortran \\ 
~ & Options:  & global \\ 
~ & Type:  & function \\ 
\end{tabular*} 


\vspace{5mm}

\noindent {\bf CCTK\_TERMINATE}   (conditional) 

\hspace{5mm} grhydro\_wenoshutdown 

\hspace{5mm}{\it deallocate weno coefficients } 


\hspace{5mm}

 \begin{tabular*}{160mm}{cll} 
~ & Before:  & driver\_terminate \\ 
~ & Language:  & fortran \\ 
~ & Options:  & global \\ 
~ & Type:  & function \\ 
\end{tabular*} 


\vspace{5mm}

\noindent {\bf CCTK\_INITIAL}   (conditional) 

\hspace{5mm} grhydro\_eoschangegamma 

\hspace{5mm}{\it reset the specific internal energy if the eos changes between id and evolution } 


\hspace{5mm}

 \begin{tabular*}{160mm}{cll} 
~ & After:  & hydrobase\_initial \\ 
~ & Before:  & grhydro\_ivp \\ 
~ & Language:  & fortran \\ 
~ & Type:  & function \\ 
\end{tabular*} 


\vspace{5mm}

\noindent {\bf CCTK\_INITIAL}   (conditional) 

\hspace{5mm} grhydro\_eoschangek 

\hspace{5mm}{\it reset the hydro variables if the eos (k) changes between id and evolution } 


\hspace{5mm}

 \begin{tabular*}{160mm}{cll} 
~ & After:  & hydrobase\_initial \\ 
~ & Before:  & grhydro\_ivp \\ 
~ & Language:  & fortran \\ 
~ & Type:  & function \\ 
\end{tabular*} 


\vspace{5mm}

\noindent {\bf CCTK\_INITIAL}   (conditional) 

\hspace{5mm} grhydro\_eoschangegammak\_shibata 

\hspace{5mm}{\it reset the hydro variables if the eos gamma and k change between id and evolution } 


\hspace{5mm}

 \begin{tabular*}{160mm}{cll} 
~ & After:  & hydrobase\_initial \\ 
~ & Before:  & grhydro\_ivp \\ 
~ & Language:  & fortran \\ 
~ & Sync:  & dens \\ 
~& ~ &tau\\ 
~& ~ &scon\\ 
~& ~ &hydrobase::w\_lorentz\\ 
~& ~ &hydrobase::rho\\ 
~& ~ &hydrobase::press\\ 
~& ~ &hydrobase::eps\\ 
~& ~ &hydrobase::vel\\ 
~& ~ &lvel\\ 
~& ~ &hydrobase::temperature\\ 
~& ~ &hydrobase::entropy\\ 
~& ~ &hydrobase::y\_e\\ 
~& ~ &y\_e\_con\\ 
~ & Type:  & function \\ 
\end{tabular*} 


\vspace{5mm}

\noindent {\bf MoL\_Register}   (conditional) 

\hspace{5mm} grhydro\_register 

\hspace{5mm}{\it register variables for mol } 


\hspace{5mm}

 \begin{tabular*}{160mm}{cll} 
~ & Language:  & c \\ 
~ & Type:  & function \\ 
\end{tabular*} 


\vspace{5mm}

\noindent {\bf MoL\_PreStep}   (conditional) 

\hspace{5mm} grhydro\_scalar\_setup 

\hspace{5mm}{\it set up and check scalars for efficiency } 


\hspace{5mm}

 \begin{tabular*}{160mm}{cll} 
~ & Language:  & fortran \\ 
~ & Type:  & function \\ 
\end{tabular*} 


\vspace{5mm}

\noindent {\bf HydroBase\_Initial}   (conditional) 

\hspace{5mm} grhydro\_initial 

\hspace{5mm}{\it grhydro initial data group } 


\hspace{5mm}

 \begin{tabular*}{160mm}{cll} 
~ & Type:  & group \\ 
\end{tabular*} 


\vspace{5mm}

\noindent {\bf CCTK\_POSTINITIAL}   (conditional) 

\hspace{5mm} grhydro\_scalar\_setup 

\hspace{5mm}{\it set up and check scalars for efficiency } 


\hspace{5mm}

 \begin{tabular*}{160mm}{cll} 
~ & Language:  & fortran \\ 
~ & Type:  & function \\ 
\end{tabular*} 


\vspace{5mm}

\noindent {\bf CCTK\_INITIAL}   (conditional) 

\hspace{5mm} grhydro\_setupmask 

\hspace{5mm}{\it initialize the atmosphere mask } 


\hspace{5mm}

 \begin{tabular*}{160mm}{cll} 
~ & Before:  & hydrobase\_initial \\ 
~ & Language:  & fortran \\ 
~ & Type:  & function \\ 
\end{tabular*} 


\vspace{5mm}

\noindent {\bf CCTK\_POST\_RECOVER\_VARIABLES}   (conditional) 

\hspace{5mm} grhydrocopyintegermask 

\hspace{5mm}{\it initialize the real valued atmosphere mask after checkpoint recovery } 


\hspace{5mm}

 \begin{tabular*}{160mm}{cll} 
~ & Before:  & mol\_poststep \\ 
~ & Language:  & fortran \\ 
~ & Type:  & function \\ 
\end{tabular*} 


\vspace{5mm}

\noindent {\bf CCTK\_POSTREGRID}   (conditional) 

\hspace{5mm} grhydro\_setupmask 

\hspace{5mm}{\it initialize the atmosphere mask } 


\hspace{5mm}

 \begin{tabular*}{160mm}{cll} 
~ & After:  & maskone \\ 
~& ~ &maskzero\\ 
~ & Before:  & mol\_poststep \\ 
~ & Language:  & fortran \\ 
~ & Type:  & function \\ 
\end{tabular*} 


\vspace{5mm}

\noindent {\bf CCTK\_POSTREGRIDINITIAL}   (conditional) 

\hspace{5mm} grhydro\_setupmask 

\hspace{5mm}{\it initialize the atmosphere mask } 


\hspace{5mm}

 \begin{tabular*}{160mm}{cll} 
~ & After:  & maskone \\ 
~& ~ &maskzero\\ 
~ & Before:  & mol\_poststep \\ 
~ & Language:  & fortran \\ 
~ & Type:  & function \\ 
\end{tabular*} 


\vspace{5mm}

\noindent {\bf CCTK\_INITIAL}   (conditional) 

\hspace{5mm} grhydro\_initialatmospherereset 

\hspace{5mm}{\it use mask to enforce atmosphere at initial time } 


\hspace{5mm}

 \begin{tabular*}{160mm}{cll} 
~ & After:  & hydrobase\_initial \\ 
~& ~ &grhydro\_sqrtspatialdeterminant\\ 
~ & Before:  & hydrobase\_prim2coninitial \\ 
~ & Language:  & fortran \\ 
~ & Type:  & function \\ 
\end{tabular*} 


\vspace{5mm}

\noindent {\bf CCTK\_POSTINITIAL}   (conditional) 

\hspace{5mm} grhydro\_initatmosmask 

\hspace{5mm}{\it set the atmosphere mask } 


\hspace{5mm}

 \begin{tabular*}{160mm}{cll} 
~ & Language:  & fortran \\ 
~ & Type:  & function \\ 
\end{tabular*} 


\vspace{5mm}

\noindent {\bf CCTK\_POSTREGRID}   (conditional) 

\hspace{5mm} grhydro\_initatmosmask 

\hspace{5mm}{\it set the atmosphere mask } 


\hspace{5mm}

 \begin{tabular*}{160mm}{cll} 
~ & After:  & grhydro\_setupmask \\ 
~ & Before:  & mol\_poststep \\ 
~ & Language:  & fortran \\ 
~ & Type:  & function \\ 
\end{tabular*} 


\vspace{5mm}

\noindent {\bf CCTK\_POSTREGRIDINITIAL}   (conditional) 

\hspace{5mm} grhydro\_initatmosmask 

\hspace{5mm}{\it set the atmosphere mask } 


\hspace{5mm}

 \begin{tabular*}{160mm}{cll} 
~ & After:  & grhydro\_setupmask \\ 
~ & Before:  & hydrobase\_poststep \\ 
~ & Language:  & fortran \\ 
~ & Type:  & function \\ 
\end{tabular*} 


\vspace{5mm}

\noindent {\bf CCTK\_INITIAL}   (conditional) 

\hspace{5mm} grhydrotransformprimtolocalbasis 

\hspace{5mm}{\it transform primitive vars to local tensor basis. } 


\hspace{5mm}

 \begin{tabular*}{160mm}{cll} 
~ & After:  & hydrobase\_initial \\ 
~& ~ &admbase\_postinitial\\ 
~ & Before:  & hydrobase\_prim2coninitial \\ 
~ & Language:  & c \\ 
~ & Type:  & function \\ 
\end{tabular*} 


\vspace{5mm}

\noindent {\bf HydroBase\_Initial}   (conditional) 

\hspace{5mm} grhydro\_initdivergenceclean 

\hspace{5mm}{\it set psi for divergence cleaning initially to zero } 


\hspace{5mm}

 \begin{tabular*}{160mm}{cll} 
~ & Language:  & fortran \\ 
~ & Type:  & function \\ 
\end{tabular*} 


\vspace{5mm}

\noindent {\bf CCTK\_INITIAL}   (conditional) 

\hspace{5mm} grhydrotransformadmtolocalbasis 

\hspace{5mm}{\it transform adm metric, extr. curv. and shift to local tensor basis. } 


\hspace{5mm}

 \begin{tabular*}{160mm}{cll} 
~ & After:  & hydrobase\_initial \\ 
~ & Before:  & grhydrotransformprimtolocalbasis \\ 
~ & Language:  & c \\ 
~ & Type:  & function \\ 
\end{tabular*} 


\vspace{5mm}

\noindent {\bf ADMBase\_SetADMVars}   (conditional) 

\hspace{5mm} grhydrotransformadmtolocalbasis 

\hspace{5mm}{\it transform metric and shift to local tensor basis. } 


\hspace{5mm}

 \begin{tabular*}{160mm}{cll} 
~ & If:  & grhydro::execute\_mol\_step \\ 
~ & Language:  & c \\ 
~ & Type:  & function \\ 
\end{tabular*} 


\vspace{5mm}

\noindent {\bf HydroBase\_PostStep}   (conditional) 

\hspace{5mm} grhydrotransformprimtoglobalbasis 

\hspace{5mm}{\it transform primitive vars to global tensor basis. } 


\hspace{5mm}

 \begin{tabular*}{160mm}{cll} 
~ & After:  & hydrobase\_con2prim \\ 
~ & If:  & grhydro::execute\_mol\_poststep \\ 
~ & Language:  & c \\ 
~ & Type:  & function \\ 
\end{tabular*} 


\vspace{5mm}

\noindent {\bf HydroBase\_RHS} 

\hspace{5mm} grhydrorhs 

\hspace{5mm}{\it calculate the update terms } 


\hspace{5mm}

 \begin{tabular*}{160mm}{cll} 
~ & If:  & grhydro::execute\_mol\_step \\ 
~ & Storage:  & grhydro\_scalars \\ 
~ & Type:  & group \\ 
\end{tabular*} 


\vspace{5mm}

\noindent {\bf CCTK\_INITIAL}   (conditional) 

\hspace{5mm} grhydroparticleinitial 

\hspace{5mm}{\it initial data for the particle arrays } 


\hspace{5mm}

 \begin{tabular*}{160mm}{cll} 
~ & Before:  & hydrobase\_initial \\ 
~ & Language:  & fortran \\ 
~ & Options:  & global \\ 
~ & Type:  & function \\ 
\end{tabular*} 


\vspace{5mm}

\noindent {\bf GRHydroRHS}   (conditional) 

\hspace{5mm} grhydroparticlerhs 

\hspace{5mm}{\it update terms for the particles } 


\hspace{5mm}

 \begin{tabular*}{160mm}{cll} 
~ & Language:  & fortran \\ 
~ & Options:  & global \\ 
~ & Type:  & function \\ 
\end{tabular*} 


\vspace{5mm}

\noindent {\bf GRHydroRHS}   (conditional) 

\hspace{5mm} sourceterms 

\hspace{5mm}{\it source term calculation } 


\hspace{5mm}

 \begin{tabular*}{160mm}{cll} 
~ & Before:  & fluxterms \\ 
~ & Language:  & c \\ 
~ & Type:  & function \\ 
\end{tabular*} 


\vspace{5mm}

\noindent {\bf GRHydroRHS}   (conditional) 

\hspace{5mm} sourcetermsm 

\hspace{5mm}{\it source term calculation - mhd version } 


\hspace{5mm}

 \begin{tabular*}{160mm}{cll} 
~ & Before:  & fluxterms \\ 
~ & Language:  & fortran \\ 
~ & Type:  & function \\ 
\end{tabular*} 


\vspace{5mm}

\noindent {\bf GRHydroRHS}   (conditional) 

\hspace{5mm} sourcetermsam 

\hspace{5mm}{\it source term calculation - vector potential mhd version } 


\hspace{5mm}

 \begin{tabular*}{160mm}{cll} 
~ & Before:  & fluxterms \\ 
~ & Language:  & fortran \\ 
~ & Type:  & function \\ 
\end{tabular*} 


\vspace{5mm}

\noindent {\bf GRHydroRHS}   (conditional) 

\hspace{5mm} sourceterms 

\hspace{5mm}{\it source term calculation } 


\hspace{5mm}

 \begin{tabular*}{160mm}{cll} 
~ & Before:  & fluxterms \\ 
~ & Language:  & fortran \\ 
~ & Type:  & function \\ 
\end{tabular*} 


\vspace{5mm}

\noindent {\bf HydroBase\_Initial}   (conditional) 

\hspace{5mm} grhydro\_divbinit 

\hspace{5mm}{\it set divb initially to zero } 


\hspace{5mm}

 \begin{tabular*}{160mm}{cll} 
~ & Language:  & fortran \\ 
~ & Type:  & function \\ 
\end{tabular*} 


\vspace{5mm}

\noindent {\bf GRHydroRHS}   (conditional) 

\hspace{5mm} grhydrostartloop 

\hspace{5mm}{\it set the flux\_direction variable } 


\hspace{5mm}

 \begin{tabular*}{160mm}{cll} 
~ & Before:  & fluxterms \\ 
~ & Language:  & fortran \\ 
~ & Options:  & level \\ 
~ & Type:  & function \\ 
\end{tabular*} 


\vspace{5mm}

\noindent {\bf GRHydroRHS}   (conditional) 

\hspace{5mm} fluxterms 

\hspace{5mm}{\it calculation of intercell fluxes } 


\hspace{5mm}

 \begin{tabular*}{160mm}{cll} 
~ & Storage:  & grhydro\_prim\_bext \\ 
~& ~ &grhydro\_con\_bext\\ 
~& ~ &grhydro\_fluxes\\ 
~& ~ &grhydro\_mhd\_con\_bext\\ 
~& ~ &grhydro\_mhd\_prim\_bext\\ 
~& ~ &grhydro\_mhd\_psidc\_bext\\ 
~& ~ &grhydro\_entropy\_prim\_bext\\ 
~& ~ &grhydro\_entropy\_con\_bext\\ 
~& ~ &grhydro\_bfluxes\\ 
~& ~ &grhydro\_psifluxes\\ 
~& ~ &grhydro\_entropyfluxes\\ 
~& ~ &grhydro\_tracer\_cons\_bext\\ 
~& ~ &grhydro\_tracer\_prim\_bext\\ 
~& ~ &grhydro\_tracer\_flux\\ 
~ & Type:  & group \\ 
~ & While:  & grhydro::flux\_direction \\ 
\end{tabular*} 


\vspace{5mm}

\noindent {\bf GRHydroRHS}   (conditional) 

\hspace{5mm} fluxterms 

\hspace{5mm}{\it calculation of intercell fluxes } 


\hspace{5mm}

 \begin{tabular*}{160mm}{cll} 
~ & Storage:  & grhydro\_prim\_bext \\ 
~& ~ &grhydro\_con\_bext\\ 
~& ~ &grhydro\_fluxes\\ 
~& ~ &grhydro\_mhd\_con\_bext\\ 
~& ~ &grhydro\_mhd\_prim\_bext\\ 
~& ~ &grhydro\_avec\_bext\\ 
~& ~ &grhydro\_avecfluxes\\ 
~& ~ &grhydro\_tracer\_cons\_bext\\ 
~& ~ &grhydro\_tracer\_prim\_bext\\ 
~& ~ &grhydro\_tracer\_flux\\ 
~ & Type:  & group \\ 
~ & While:  & grhydro::flux\_direction \\ 
\end{tabular*} 


\vspace{5mm}

\noindent {\bf GRHydroRHS}   (conditional) 

\hspace{5mm} fluxterms 

\hspace{5mm}{\it calculation of intercell fluxes } 


\hspace{5mm}

 \begin{tabular*}{160mm}{cll} 
~ & Storage:  & grhydro\_prim\_bext \\ 
~& ~ &grhydro\_con\_bext\\ 
~& ~ &grhydro\_fluxes\\ 
~& ~ &grhydro\_mhd\_con\_bext\\ 
~& ~ &grhydro\_mhd\_prim\_bext\\ 
~& ~ &grhydro\_avec\_bext\\ 
~& ~ &grhydro\_aphi\_bext\\ 
~& ~ &grhydro\_avecfluxes\\ 
~& ~ &grhydro\_aphifluxes\\ 
~& ~ &grhydro\_tracer\_cons\_bext\\ 
~& ~ &grhydro\_tracer\_prim\_bext\\ 
~& ~ &grhydro\_tracer\_flux\\ 
~ & Type:  & group \\ 
~ & While:  & grhydro::flux\_direction \\ 
\end{tabular*} 


\vspace{5mm}

\noindent {\bf GRHydroRHS}   (conditional) 

\hspace{5mm} fluxterms 

\hspace{5mm}{\it calculation of intercell fluxes } 


\hspace{5mm}

 \begin{tabular*}{160mm}{cll} 
~ & Storage:  & grhydro\_prim\_bext \\ 
~& ~ &grhydro\_con\_bext\\ 
~& ~ &grhydro\_fluxes\\ 
~& ~ &grhydro\_tracer\_cons\_bext\\ 
~& ~ &grhydro\_tracer\_prim\_bext\\ 
~& ~ &grhydro\_tracer\_flux\\ 
~ & Type:  & group \\ 
~ & While:  & grhydro::flux\_direction \\ 
\end{tabular*} 


\vspace{5mm}

\noindent {\bf GRHydroRHS}   (conditional) 

\hspace{5mm} fluxterms 

\hspace{5mm}{\it calculation of intercell fluxes } 


\hspace{5mm}

 \begin{tabular*}{160mm}{cll} 
~ & Storage:  & grhydro\_prim\_bext \\ 
~& ~ &grhydro\_con\_bext\\ 
~& ~ &grhydro\_fluxes\\ 
~& ~ &grhydro\_mhd\_con\_bext\\ 
~& ~ &grhydro\_mhd\_prim\_bext\\ 
~& ~ &grhydro\_mhd\_psidc\_bext\\ 
~& ~ &grhydro\_entropy\_con\_bext\\ 
~& ~ &grhydro\_entropy\_prim\_bext\\ 
~& ~ &grhydro\_bfluxes\\ 
~& ~ &grhydro\_psifluxes\\ 
~& ~ &grhydro\_entropyfluxes\\ 
~ & Type:  & group \\ 
~ & While:  & grhydro::flux\_direction \\ 
\end{tabular*} 


\vspace{5mm}

\noindent {\bf GRHydroRHS}   (conditional) 

\hspace{5mm} fluxterms 

\hspace{5mm}{\it calculation of intercell fluxes } 


\hspace{5mm}

 \begin{tabular*}{160mm}{cll} 
~ & Storage:  & grhydro\_prim\_bext \\ 
~& ~ &grhydro\_con\_bext\\ 
~& ~ &grhydro\_fluxes\\ 
~& ~ &grhydro\_mhd\_con\_bext\\ 
~& ~ &grhydro\_mhd\_prim\_bext\\ 
~& ~ &grhydro\_avec\_bext\\ 
~& ~ &grhydro\_aphi\_bext\\ 
~& ~ &grhydro\_avecfluxes\\ 
~& ~ &grhydro\_aphifluxes\\ 
~ & Type:  & group \\ 
~ & While:  & grhydro::flux\_direction \\ 
\end{tabular*} 


\vspace{5mm}

\noindent {\bf GRHydroRHS}   (conditional) 

\hspace{5mm} fluxterms 

\hspace{5mm}{\it calculation of intercell fluxes } 


\hspace{5mm}

 \begin{tabular*}{160mm}{cll} 
~ & Storage:  & grhydro\_prim\_bext \\ 
~& ~ &grhydro\_con\_bext\\ 
~& ~ &grhydro\_fluxes\\ 
~ & Type:  & group \\ 
~ & While:  & grhydro::flux\_direction \\ 
\end{tabular*} 


\vspace{5mm}

\noindent {\bf FluxTerms}   (conditional) 

\hspace{5mm} reconstruction\_cxx 

\hspace{5mm}{\it reconstruct the functions at the cell boundaries } 


\hspace{5mm}

 \begin{tabular*}{160mm}{cll} 
~ & Language:  & c \\ 
~ & Type:  & function \\ 
\end{tabular*} 


\vspace{5mm}

\noindent {\bf FluxTerms}   (conditional) 

\hspace{5mm} reconstruction 

\hspace{5mm}{\it reconstruct the functions at the cell boundaries } 


\hspace{5mm}

 \begin{tabular*}{160mm}{cll} 
~ & Language:  & fortran \\ 
~ & Type:  & function \\ 
\end{tabular*} 


\vspace{5mm}

\noindent {\bf HydroBase\_Initial}   (conditional) 

\hspace{5mm} grhydro\_bvecfromavec 

\hspace{5mm}{\it populate bvec from avec } 


\hspace{5mm}

 \begin{tabular*}{160mm}{cll} 
~ & Language:  & fortran \\ 
~ & Type:  & function \\ 
\end{tabular*} 


\vspace{5mm}

\noindent {\bf FluxTerms}   (conditional) 

\hspace{5mm} reconstructionpolytype 

\hspace{5mm}{\it reconstruct the functions at the cell boundaries } 


\hspace{5mm}

 \begin{tabular*}{160mm}{cll} 
~ & Language:  & fortran \\ 
~ & Type:  & function \\ 
\end{tabular*} 


\vspace{5mm}

\noindent {\bf FluxTerms}   (conditional) 

\hspace{5mm} set\_trivial\_riemann\_problem\_grid\_function 

\hspace{5mm}{\it set the gridfunction for the trp (for debugging only) } 


\hspace{5mm}

 \begin{tabular*}{160mm}{cll} 
~ & After:  & reconstruct \\ 
~ & Language:  & c \\ 
~ & Sync:  & grhydro\_trivial\_rp\_gf\_group \\ 
~ & Type:  & function \\ 
\end{tabular*} 


\vspace{5mm}

\noindent {\bf FluxTerms}   (conditional) 

\hspace{5mm} riemannsolvem 

\hspace{5mm}{\it solve the local riemann problems - mhd version } 


\hspace{5mm}

 \begin{tabular*}{160mm}{cll} 
~ & After:  & reconstruct \\ 
~ & Language:  & fortran \\ 
~ & Storage:  & eos\_temps \\ 
~ & Type:  & function \\ 
\end{tabular*} 


\vspace{5mm}

\noindent {\bf FluxTerms}   (conditional) 

\hspace{5mm} riemannsolveam 

\hspace{5mm}{\it solve the local riemann problems - vector potential mhd version } 


\hspace{5mm}

 \begin{tabular*}{160mm}{cll} 
~ & After:  & reconstruct \\ 
~ & Language:  & fortran \\ 
~ & Storage:  & eos\_temps \\ 
~ & Type:  & function \\ 
\end{tabular*} 


\vspace{5mm}

\noindent {\bf FluxTerms}   (conditional) 

\hspace{5mm} riemannsolve 

\hspace{5mm}{\it solve the local riemann problems } 


\hspace{5mm}

 \begin{tabular*}{160mm}{cll} 
~ & After:  & reconstruct \\ 
~ & Language:  & fortran \\ 
~ & Storage:  & eos\_temps \\ 
~ & Type:  & function \\ 
\end{tabular*} 


\vspace{5mm}

\noindent {\bf FluxTerms}   (conditional) 

\hspace{5mm} riemannsolvepolytype 

\hspace{5mm}{\it solve the local riemann problems } 


\hspace{5mm}

 \begin{tabular*}{160mm}{cll} 
~ & After:  & reconstruct \\ 
~ & Language:  & fortran \\ 
~ & Storage:  & eos\_temps \\ 
~ & Type:  & function \\ 
\end{tabular*} 


\vspace{5mm}

\noindent {\bf GRHydroRHS}   (conditional) 

\hspace{5mm} fluxterms 

\hspace{5mm}{\it calculation of intercell fluxes } 


\hspace{5mm}

 \begin{tabular*}{160mm}{cll} 
~ & Storage:  & grhydro\_fluxes \\ 
~& ~ &grhydro\_tracer\_flux\\ 
~ & Type:  & group \\ 
~ & While:  & grhydro::flux\_direction \\ 
\end{tabular*} 


\vspace{5mm}

\noindent {\bf FluxTerms}   (conditional) 

\hspace{5mm} grhydro\_fsalpha 

\hspace{5mm}{\it compute the maximum characteristic speeds } 


\hspace{5mm}

 \begin{tabular*}{160mm}{cll} 
~ & Before:  & grhydro\_splitflux \\ 
~ & Language:  & fortran \\ 
~ & Type:  & function \\ 
\end{tabular*} 


\vspace{5mm}

\noindent {\bf FluxTerms}   (conditional) 

\hspace{5mm} grhydro\_splitflux 

\hspace{5mm}{\it compute the fluxes using weno5 fd + lax-friedrichs splitting } 


\hspace{5mm}

 \begin{tabular*}{160mm}{cll} 
~ & Language:  & fortran \\ 
~ & Storage:  & flux\_splitting \\ 
~& ~ &grhydro\_tracer\_flux\_splitting\\ 
~ & Sync:  & grhydro\_fluxes \\ 
~ & Type:  & function \\ 
\end{tabular*} 


\vspace{5mm}

\noindent {\bf GRHydroRHS}   (conditional) 

\hspace{5mm} fluxterms 

\hspace{5mm}{\it calculation of intercell fluxes } 


\hspace{5mm}

 \begin{tabular*}{160mm}{cll} 
~ & Storage:  & grhydro\_fluxes \\ 
~ & Type:  & group \\ 
~ & While:  & grhydro::flux\_direction \\ 
\end{tabular*} 


\vspace{5mm}

\noindent {\bf HydroBase\_Initial}   (conditional) 

\hspace{5mm} hydrobase\_boundaries 

\hspace{5mm}{\it call boundary conditions after magnetic field initial data setup } 


\hspace{5mm}

 \begin{tabular*}{160mm}{cll} 
~ & After:  & grhydro\_poloidalmagfieldm \\ 
~& ~ &grhydro\_bvec\_from\_avec\\ 
~ & Type:  & group \\ 
\end{tabular*} 


\vspace{5mm}

\noindent {\bf FluxTerms}   (conditional) 

\hspace{5mm} grhydro\_fsalpha 

\hspace{5mm}{\it compute the maximum characteristic speeds } 


\hspace{5mm}

 \begin{tabular*}{160mm}{cll} 
~ & Before:  & grhydro\_splitflux \\ 
~ & Language:  & fortran \\ 
~ & Type:  & function \\ 
\end{tabular*} 


\vspace{5mm}

\noindent {\bf FluxTerms}   (conditional) 

\hspace{5mm} grhydro\_splitflux 

\hspace{5mm}{\it compute the fluxes using weno5 fd + lax-friedrichs splitting } 


\hspace{5mm}

 \begin{tabular*}{160mm}{cll} 
~ & Language:  & fortran \\ 
~ & Storage:  & flux\_splitting \\ 
~ & Sync:  & grhydro\_fluxes \\ 
~ & Type:  & function \\ 
\end{tabular*} 


\vspace{5mm}

\noindent {\bf FluxTerms}   (conditional) 

\hspace{5mm} updatecalculation 

\hspace{5mm}{\it calculate the update term from the fluxes } 


\hspace{5mm}

 \begin{tabular*}{160mm}{cll} 
~ & After:  & riemann \\ 
~ & Language:  & fortran \\ 
~ & Type:  & function \\ 
\end{tabular*} 


\vspace{5mm}

\noindent {\bf FluxTerms}   (conditional) 

\hspace{5mm} grhydroadvanceloop 

\hspace{5mm}{\it decrement the flux\_direction variable } 


\hspace{5mm}

 \begin{tabular*}{160mm}{cll} 
~ & After:  & updatecalcul \\ 
~ & Language:  & fortran \\ 
~ & Options:  & level \\ 
~ & Type:  & function \\ 
\end{tabular*} 


\vspace{5mm}

\noindent {\bf GRHydroRHS}   (conditional) 

\hspace{5mm} grhydroupdateatmospheremask 

\hspace{5mm}{\it alter the update terms if inside the atmosphere region } 


\hspace{5mm}

 \begin{tabular*}{160mm}{cll} 
~ & After:  & fluxterms \\ 
~ & Language:  & fortran \\ 
~ & Type:  & function \\ 
\end{tabular*} 


\vspace{5mm}

\noindent {\bf HydroBase\_PostStep}   (conditional) 

\hspace{5mm} grhydro\_poststep 

\hspace{5mm}{\it post step tasks for grhydro } 


\hspace{5mm}

 \begin{tabular*}{160mm}{cll} 
~ & Type:  & group \\ 
\end{tabular*} 


\vspace{5mm}

\noindent {\bf MoL\_PostStep}   (conditional) 

\hspace{5mm} grhydro\_refinementlevel 

\hspace{5mm}{\it calculate current refinement level } 


\hspace{5mm}

 \begin{tabular*}{160mm}{cll} 
~ & Before:  & hydrobase\_poststep \\ 
~ & Language:  & fortran \\ 
~ & Type:  & function \\ 
\end{tabular*} 


\vspace{5mm}

\noindent {\bf FluxTerms}   (conditional) 

\hspace{5mm} grhydro\_refinementlevel 

\hspace{5mm}{\it calculate current refinement level } 


\hspace{5mm}

 \begin{tabular*}{160mm}{cll} 
~ & Before:  & reconstruct \\ 
~ & Language:  & fortran \\ 
~ & Type:  & function \\ 
\end{tabular*} 


\vspace{5mm}

\noindent {\bf HydroBase\_Con2Prim}   (conditional) 

\hspace{5mm} grhydro\_sqrtspatialdeterminant 

\hspace{5mm}{\it calculate sdetg } 


\hspace{5mm}

 \begin{tabular*}{160mm}{cll} 
~ & Before:  & con2prim \\ 
~ & If:  & grhydro::execute\_mol\_step \\ 
~ & Language:  & fortran \\ 
~ & Type:  & function \\ 
\end{tabular*} 


\vspace{5mm}

\noindent {\bf CCTK\_POST\_RECOVER\_VARIABLES}   (conditional) 

\hspace{5mm} grhydro\_sqrtspatialdeterminant 

\hspace{5mm}{\it calculate sdetg } 


\hspace{5mm}

 \begin{tabular*}{160mm}{cll} 
~ & Language:  & fortran \\ 
~ & Type:  & function \\ 
\end{tabular*} 


\vspace{5mm}

\noindent {\bf HydroBase\_Initial}   (conditional) 

\hspace{5mm} grhydro\_setupcoords 

\hspace{5mm}{\it set up the coordinates for use with the comoving shift } 


\hspace{5mm}

 \begin{tabular*}{160mm}{cll} 
~ & Language:  & fortran \\ 
~ & Type:  & function \\ 
\end{tabular*} 


\vspace{5mm}

\noindent {\bf CCTK\_POSTSTEP}   (conditional) 

\hspace{5mm} grhydro\_refinementlevel 

\hspace{5mm}{\it calculate current refinement level (for the check of the c2p mask) } 


\hspace{5mm}

 \begin{tabular*}{160mm}{cll} 
~ & Language:  & fortran \\ 
~ & Type:  & function \\ 
\end{tabular*} 


\vspace{5mm}

\noindent {\bf CCTK\_POSTREGRIDINITIAL}   (conditional) 

\hspace{5mm} grhydro\_refinementlevel 

\hspace{5mm}{\it calculate current refinement level } 


\hspace{5mm}

 \begin{tabular*}{160mm}{cll} 
~ & Language:  & fortran \\ 
~ & Type:  & function \\ 
\end{tabular*} 


\vspace{5mm}

\noindent {\bf CCTK\_INITIAL}   (conditional) 

\hspace{5mm} grhydro\_refinementlevel 

\hspace{5mm}{\it calculate current refinement level } 


\hspace{5mm}

 \begin{tabular*}{160mm}{cll} 
~ & Before:  & hydrobase\_initial \\ 
~ & Language:  & fortran \\ 
~ & Type:  & function \\ 
\end{tabular*} 


\vspace{5mm}

\noindent {\bf HydroBase\_Con2Prim}   (conditional) 

\hspace{5mm} conservative2primitivem 

\hspace{5mm}{\it convert back to primitive variables (general) - mhd version } 


\hspace{5mm}

 \begin{tabular*}{160mm}{cll} 
~ & If:  & grhydro::execute\_mol\_poststep \\ 
~ & Language:  & fortran \\ 
~ & Type:  & function \\ 
\end{tabular*} 


\vspace{5mm}

\noindent {\bf HydroBase\_Prim2ConInitial}   (conditional) 

\hspace{5mm} primitive2conservativecellsm 

\hspace{5mm}{\it convert initial data given in primive variables to conserved variables - mhd version } 


\hspace{5mm}

 \begin{tabular*}{160mm}{cll} 
~ & Language:  & fortran \\ 
~ & Type:  & function \\ 
\end{tabular*} 


\vspace{5mm}

\noindent {\bf HydroBase\_Con2Prim}   (conditional) 

\hspace{5mm} grhydro\_bvecfromavec 

\hspace{5mm}{\it populate bvec from avec } 


\hspace{5mm}

 \begin{tabular*}{160mm}{cll} 
~ & Language:  & fortran \\ 
~ & Type:  & function \\ 
\end{tabular*} 


\vspace{5mm}

\noindent {\bf HydroBase\_Con2Prim}   (conditional) 

\hspace{5mm} hydrobase\_boundaries 

\hspace{5mm}{\it call boundary conditions after magnetic field initial data setup } 


\hspace{5mm}

 \begin{tabular*}{160mm}{cll} 
~ & After:  & grhydro\_bvecfromavec \\ 
~ & Type:  & group \\ 
\end{tabular*} 


\vspace{5mm}

\noindent {\bf HydroBase\_Con2Prim}   (conditional) 

\hspace{5mm} conservative2primitiveam 

\hspace{5mm}{\it convert back to primitive variables (general) - mhd with avec version } 


\hspace{5mm}

 \begin{tabular*}{160mm}{cll} 
~ & After:  & hydrobase\_boundaries \\ 
~ & If:  & grhydro::execute\_mol\_poststep \\ 
~ & Language:  & fortran \\ 
~ & Type:  & function \\ 
\end{tabular*} 


\vspace{5mm}

\noindent {\bf HydroBase\_Prim2ConInitial}   (conditional) 

\hspace{5mm} primitive2conservativecellsam 

\hspace{5mm}{\it convert initial data given in primive variables to conserved variables - mhd with avec version } 


\hspace{5mm}

 \begin{tabular*}{160mm}{cll} 
~ & Language:  & fortran \\ 
~ & Type:  & function \\ 
\end{tabular*} 


\vspace{5mm}

\noindent {\bf HydroBase\_Con2Prim}   (conditional) 

\hspace{5mm} conservative2primitive 

\hspace{5mm}{\it convert back to primitive variables (general) } 


\hspace{5mm}

 \begin{tabular*}{160mm}{cll} 
~ & If:  & grhydro::execute\_mol\_poststep \\ 
~ & Language:  & fortran \\ 
~ & Type:  & function \\ 
\end{tabular*} 


\subsection*{Aliased Functions}

\hspace{5mm}

 \begin{tabular*}{160mm}{ll} 

{\bf Alias Name:} ~~~~~~~ & {\bf Function Name:} \\ 
ApplyBCs & GRHydro\_ApplyDivBBCs \\ 
Conservative2Primitive & Con2Prim \\ 
Conservative2PrimitiveAM & Con2Prim \\ 
Conservative2PrimitiveM & Con2Prim \\ 
Conservative2PrimitivePolytype & Con2Prim \\ 
Conservative2PrimitivePolytypeAM & Con2Prim \\ 
Conservative2PrimitivePolytypeM & Con2Prim \\ 
GRHydro\_Boundaries & GRHydro\_Bound \\ 
GRHydro\_SplitFlux & Reconstruct \\ 
Reconstruction & Reconstruct \\ 
ReconstructionPolytype & Reconstruct \\ 
Reconstruction\_cxx & Reconstruct \\ 
RiemannSolve & Riemann \\ 
RiemannSolveAM & Riemann \\ 
RiemannSolveM & Riemann \\ 
RiemannSolvePolytype & Riemann \\ 
UpdateCalculation & UpdateCalcul \\ 
\end{tabular*} 



\vspace{5mm}\parskip = 10pt 
\end{document}
