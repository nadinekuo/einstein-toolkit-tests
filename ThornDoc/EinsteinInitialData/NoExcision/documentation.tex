% *======================================================================*
%  Cactus Thorn template for ThornGuide documentation
%  Author: Ian Kelley
%  Date: Sun Jun 02, 2002
%  $Header$
%
%  Thorn documentation in the latex file doc/documentation.tex
%  will be included in ThornGuides built with the Cactus make system.
%  The scripts employed by the make system automatically include
%  pages about variables, parameters and scheduling parsed from the
%  relevant thorn CCL files.
%
%  This template contains guidelines which help to assure that your
%  documentation will be correctly added to ThornGuides. More
%  information is available in the Cactus UsersGuide.
%
%  Guidelines:
%   - Do not change anything before the line
%       % START CACTUS THORNGUIDE",
%     except for filling in the title, author, date, etc. fields.
%        - Each of these fields should only be on ONE line.
%        - Author names should be separated with a \\ or a comma.
%   - You can define your own macros, but they must appear after
%     the START CACTUS THORNGUIDE line, and must not redefine standard
%     latex commands.
%   - To avoid name clashes with other thorns, 'labels', 'citations',
%     'references', and 'image' names should conform to the following
%     convention:
%       ARRANGEMENT_THORN_LABEL
%     For example, an image wave.eps in the arrangement CactusWave and
%     thorn WaveToyC should be renamed to CactusWave_WaveToyC_wave.eps
%   - Graphics should only be included using the graphicx package.
%     More specifically, with the "\includegraphics" command.  Do
%     not specify any graphic file extensions in your .tex file. This
%     will allow us to create a PDF version of the ThornGuide
%     via pdflatex.
%   - References should be included with the latex "\bibitem" command.
%   - Use \begin{abstract}...\end{abstract} instead of \abstract{...}
%   - Do not use \appendix, instead include any appendices you need as
%     standard sections.
%   - For the benefit of our Perl scripts, and for future extensions,
%     please use simple latex.
%
% *======================================================================*
%
% Example of including a graphic image:
%    \begin{figure}[ht]
% 	\begin{center}
%    	   \includegraphics[width=6cm]{/home/runner/work/einstein-toolkit-tests/einstein-toolkit-tests/arrangements/EinsteinInitialData/NoExcision/doc/MyArrangement_MyThorn_MyFigure}
% 	\end{center}
% 	\caption{Illustration of this and that}
% 	\label{MyArrangement_MyThorn_MyLabel}
%    \end{figure}
%
% Example of using a label:
%   \label{MyArrangement_MyThorn_MyLabel}
%
% Example of a citation:
%    \cite{MyArrangement_MyThorn_Author99}
%
% Example of including a reference
%   \bibitem{MyArrangement_MyThorn_Author99}
%   {J. Author, {\em The Title of the Book, Journal, or periodical}, 1 (1999),
%   1--16. {\tt http://www.nowhere.com/}}
%
% *======================================================================*

% If you are using CVS use this line to give version information
% $Header$

\documentclass{article}

% Use the Cactus ThornGuide style file
% (Automatically used from Cactus distribution, if you have a
%  thorn without the Cactus Flesh download this from the Cactus
%  homepage at www.cactuscode.org)
\usepackage{../../../../../doc/latex/cactus}

\newlength{\tableWidth} \newlength{\maxVarWidth} \newlength{\paraWidth} \newlength{\descWidth} \begin{document}

% The author of the documentation
\author{Erik Schnetter \textless schnetter@aei.mpg.de\textgreater}

% The title of the document (not necessarily the name of the Thorn)
\title{NoExcision}

% the date your document was last changed, if your document is in CVS,
% please use:
%    \date{$ $Date$ $}
\date{June 17 2004}

\maketitle

% Do not delete next line
% START CACTUS THORNGUIDE

% Add all definitions used in this documentation here
%   \def\mydef etc

% Add an abstract for this thorn's documentation
\begin{abstract}

\end{abstract}

% The following sections are suggestive only.
% Remove them or add your own.

\section{Introduction}

\section{Physical System}

\section{Numerical Implementation}

\section{Using This Thorn}

\subsection{Obtaining This Thorn}

\subsection{Basic Usage}

\subsection{Special Behaviour}

\subsection{Interaction With Other Thorns}

\subsection{Examples}

\subsection{Support and Feedback}

\section{History}

\subsection{Thorn Source Code}

\subsection{Thorn Documentation}

\subsection{Acknowledgements}


\begin{thebibliography}{9}

\end{thebibliography}

% Do not delete next line
% END CACTUS THORNGUIDE



\section{Parameters} 


\parskip = 0pt

\setlength{\tableWidth}{160mm}

\setlength{\paraWidth}{\tableWidth}
\setlength{\descWidth}{\tableWidth}
\settowidth{\maxVarWidth}{smoothing\_iterations}

\addtolength{\paraWidth}{-\maxVarWidth}
\addtolength{\paraWidth}{-\columnsep}
\addtolength{\paraWidth}{-\columnsep}
\addtolength{\paraWidth}{-\columnsep}

\addtolength{\descWidth}{-\columnsep}
\addtolength{\descWidth}{-\columnsep}
\addtolength{\descWidth}{-\columnsep}
\noindent \begin{tabular*}{\tableWidth}{|c|l@{\extracolsep{\fill}}r|}
\hline
\multicolumn{1}{|p{\maxVarWidth}}{centre\_x} & {\bf Scope:} private & REAL \\\hline
\multicolumn{3}{|p{\descWidth}|}{{\bf Description:}   {\em x-coordinate of the centre of the region}} \\
\hline{\bf Range} & &  {\bf Default:} 0.0 \\\multicolumn{1}{|p{\maxVarWidth}|}{\centering (*:*)} & \multicolumn{2}{p{\paraWidth}|}{} \\\hline
\end{tabular*}

\vspace{0.5cm}\noindent \begin{tabular*}{\tableWidth}{|c|l@{\extracolsep{\fill}}r|}
\hline
\multicolumn{1}{|p{\maxVarWidth}}{centre\_y} & {\bf Scope:} private & REAL \\\hline
\multicolumn{3}{|p{\descWidth}|}{{\bf Description:}   {\em y-coordinate of the centre of the region}} \\
\hline{\bf Range} & &  {\bf Default:} 0.0 \\\multicolumn{1}{|p{\maxVarWidth}|}{\centering (*:*)} & \multicolumn{2}{p{\paraWidth}|}{} \\\hline
\end{tabular*}

\vspace{0.5cm}\noindent \begin{tabular*}{\tableWidth}{|c|l@{\extracolsep{\fill}}r|}
\hline
\multicolumn{1}{|p{\maxVarWidth}}{centre\_z} & {\bf Scope:} private & REAL \\\hline
\multicolumn{3}{|p{\descWidth}|}{{\bf Description:}   {\em z-coordinate of the centre of the region}} \\
\hline{\bf Range} & &  {\bf Default:} 0.0 \\\multicolumn{1}{|p{\maxVarWidth}|}{\centering (*:*)} & \multicolumn{2}{p{\paraWidth}|}{} \\\hline
\end{tabular*}

\vspace{0.5cm}\noindent \begin{tabular*}{\tableWidth}{|c|l@{\extracolsep{\fill}}r|}
\hline
\multicolumn{1}{|p{\maxVarWidth}}{lapse\_scale} & {\bf Scope:} private & REAL \\\hline
\multicolumn{3}{|p{\descWidth}|}{{\bf Description:}   {\em Scaling factor for lapse}} \\
\hline{\bf Range} & &  {\bf Default:} 1.0 \\\multicolumn{1}{|p{\maxVarWidth}|}{\centering (*:*)} & \multicolumn{2}{p{\paraWidth}|}{Choose 1 for geodesic slicing, 0 to halt evolution} \\\hline
\end{tabular*}

\vspace{0.5cm}\noindent \begin{tabular*}{\tableWidth}{|c|l@{\extracolsep{\fill}}r|}
\hline
\multicolumn{1}{|p{\maxVarWidth}}{method} & {\bf Scope:} private & KEYWORD \\\hline
\multicolumn{3}{|p{\descWidth}|}{{\bf Description:}   {\em Method to use}} \\
\hline{\bf Range} & &  {\bf Default:} old \\\multicolumn{1}{|p{\maxVarWidth}|}{\centering old} & \multicolumn{2}{p{\paraWidth}|}{Use old method} \\\multicolumn{1}{|p{\maxVarWidth}|}{\centering new} & \multicolumn{2}{p{\paraWidth}|}{Use new method} \\\hline
\end{tabular*}

\vspace{0.5cm}\noindent \begin{tabular*}{\tableWidth}{|c|l@{\extracolsep{\fill}}r|}
\hline
\multicolumn{1}{|p{\maxVarWidth}}{minkowski\_scale} & {\bf Scope:} private & REAL \\\hline
\multicolumn{3}{|p{\descWidth}|}{{\bf Description:}   {\em Scaling factor for Minkowski}} \\
\hline{\bf Range} & &  {\bf Default:} 1.0 \\\multicolumn{1}{|p{\maxVarWidth}|}{\centering (*:*)} & \multicolumn{2}{p{\paraWidth}|}{Choose 1 for true Minkowski} \\\hline
\end{tabular*}

\vspace{0.5cm}\noindent \begin{tabular*}{\tableWidth}{|c|l@{\extracolsep{\fill}}r|}
\hline
\multicolumn{1}{|p{\maxVarWidth}}{num\_regions} & {\bf Scope:} private & INT \\\hline
\multicolumn{3}{|p{\descWidth}|}{{\bf Description:}   {\em Number of no-excision regions}} \\
\hline{\bf Range} & &  {\bf Default:} (none) \\\multicolumn{1}{|p{\maxVarWidth}|}{\centering 0:10} & \multicolumn{2}{p{\paraWidth}|}{} \\\hline
\end{tabular*}

\vspace{0.5cm}\noindent \begin{tabular*}{\tableWidth}{|c|l@{\extracolsep{\fill}}r|}
\hline
\multicolumn{1}{|p{\maxVarWidth}}{overwrite\_geometry} & {\bf Scope:} private & BOOLEAN \\\hline
\multicolumn{3}{|p{\descWidth}|}{{\bf Description:}   {\em Set the geometry to Minkowski}} \\
\hline & & {\bf Default:} yes \\\hline
\end{tabular*}

\vspace{0.5cm}\noindent \begin{tabular*}{\tableWidth}{|c|l@{\extracolsep{\fill}}r|}
\hline
\multicolumn{1}{|p{\maxVarWidth}}{overwrite\_lapse} & {\bf Scope:} private & BOOLEAN \\\hline
\multicolumn{3}{|p{\descWidth}|}{{\bf Description:}   {\em Set the lapse to one}} \\
\hline & & {\bf Default:} yes \\\hline
\end{tabular*}

\vspace{0.5cm}\noindent \begin{tabular*}{\tableWidth}{|c|l@{\extracolsep{\fill}}r|}
\hline
\multicolumn{1}{|p{\maxVarWidth}}{overwrite\_shift} & {\bf Scope:} private & BOOLEAN \\\hline
\multicolumn{3}{|p{\descWidth}|}{{\bf Description:}   {\em Set the shift to zero}} \\
\hline & & {\bf Default:} yes \\\hline
\end{tabular*}

\vspace{0.5cm}\noindent \begin{tabular*}{\tableWidth}{|c|l@{\extracolsep{\fill}}r|}
\hline
\multicolumn{1}{|p{\maxVarWidth}}{radius} & {\bf Scope:} private & REAL \\\hline
\multicolumn{3}{|p{\descWidth}|}{{\bf Description:}   {\em Radius of the region}} \\
\hline{\bf Range} & &  {\bf Default:} 1.0 \\\multicolumn{1}{|p{\maxVarWidth}|}{\centering 0.0:*)} & \multicolumn{2}{p{\paraWidth}|}{} \\\hline
\end{tabular*}

\vspace{0.5cm}\noindent \begin{tabular*}{\tableWidth}{|c|l@{\extracolsep{\fill}}r|}
\hline
\multicolumn{1}{|p{\maxVarWidth}}{radius\_x} & {\bf Scope:} private & REAL \\\hline
\multicolumn{3}{|p{\descWidth}|}{{\bf Description:}   {\em x-radius of the region}} \\
\hline{\bf Range} & &  {\bf Default:} 1.0 \\\multicolumn{1}{|p{\maxVarWidth}|}{\centering 0.0:*)} & \multicolumn{2}{p{\paraWidth}|}{} \\\hline
\end{tabular*}

\vspace{0.5cm}\noindent \begin{tabular*}{\tableWidth}{|c|l@{\extracolsep{\fill}}r|}
\hline
\multicolumn{1}{|p{\maxVarWidth}}{radius\_y} & {\bf Scope:} private & REAL \\\hline
\multicolumn{3}{|p{\descWidth}|}{{\bf Description:}   {\em y-radius of the region}} \\
\hline{\bf Range} & &  {\bf Default:} 1.0 \\\multicolumn{1}{|p{\maxVarWidth}|}{\centering 0.0:*)} & \multicolumn{2}{p{\paraWidth}|}{} \\\hline
\end{tabular*}

\vspace{0.5cm}\noindent \begin{tabular*}{\tableWidth}{|c|l@{\extracolsep{\fill}}r|}
\hline
\multicolumn{1}{|p{\maxVarWidth}}{radius\_z} & {\bf Scope:} private & REAL \\\hline
\multicolumn{3}{|p{\descWidth}|}{{\bf Description:}   {\em z-radius of the region}} \\
\hline{\bf Range} & &  {\bf Default:} 1.0 \\\multicolumn{1}{|p{\maxVarWidth}|}{\centering 0.0:*)} & \multicolumn{2}{p{\paraWidth}|}{} \\\hline
\end{tabular*}

\vspace{0.5cm}\noindent \begin{tabular*}{\tableWidth}{|c|l@{\extracolsep{\fill}}r|}
\hline
\multicolumn{1}{|p{\maxVarWidth}}{reduce\_rhs} & {\bf Scope:} private & BOOLEAN \\\hline
\multicolumn{3}{|p{\descWidth}|}{{\bf Description:}   {\em Reduce RHS}} \\
\hline & & {\bf Default:} no \\\hline
\end{tabular*}

\vspace{0.5cm}\noindent \begin{tabular*}{\tableWidth}{|c|l@{\extracolsep{\fill}}r|}
\hline
\multicolumn{1}{|p{\maxVarWidth}}{reduction\_factor} & {\bf Scope:} private & REAL \\\hline
\multicolumn{3}{|p{\descWidth}|}{{\bf Description:}   {\em Reduction factor for RHS (0=complete, 1=no reduction)}} \\
\hline{\bf Range} & &  {\bf Default:} 0.0 \\\multicolumn{1}{|p{\maxVarWidth}|}{\centering *:*} & \multicolumn{2}{p{\paraWidth}|}{} \\\hline
\end{tabular*}

\vspace{0.5cm}\noindent \begin{tabular*}{\tableWidth}{|c|l@{\extracolsep{\fill}}r|}
\hline
\multicolumn{1}{|p{\maxVarWidth}}{region\_shape} & {\bf Scope:} private & KEYWORD \\\hline
\multicolumn{3}{|p{\descWidth}|}{{\bf Description:}   {\em Shape of the region}} \\
\hline{\bf Range} & &  {\bf Default:} sphere \\\multicolumn{1}{|p{\maxVarWidth}|}{\centering sphere} & \multicolumn{2}{p{\paraWidth}|}{use radius} \\\multicolumn{1}{|p{\maxVarWidth}|}{\centering ellipsoid} & \multicolumn{2}{p{\paraWidth}|}{use radius\_x, radius\_y, and radius\_z} \\\multicolumn{1}{|p{\maxVarWidth}|}{\centering surface} & \multicolumn{2}{p{\paraWidth}|}{use a spherical surface shape} \\\hline
\end{tabular*}

\vspace{0.5cm}\noindent \begin{tabular*}{\tableWidth}{|c|l@{\extracolsep{\fill}}r|}
\hline
\multicolumn{1}{|p{\maxVarWidth}}{smooth\_regions} & {\bf Scope:} private & BOOLEAN \\\hline
\multicolumn{3}{|p{\descWidth}|}{{\bf Description:}   {\em Smooth overwritten regions?}} \\
\hline & & {\bf Default:} no \\\hline
\end{tabular*}

\vspace{0.5cm}\noindent \begin{tabular*}{\tableWidth}{|c|l@{\extracolsep{\fill}}r|}
\hline
\multicolumn{1}{|p{\maxVarWidth}}{smoothing\_eps} & {\bf Scope:} private & REAL \\\hline
\multicolumn{3}{|p{\descWidth}|}{{\bf Description:}   {\em CG smoothing stop criteria}} \\
\hline{\bf Range} & &  {\bf Default:} 1e-6 \\\multicolumn{1}{|p{\maxVarWidth}|}{\centering (0.0:*} & \multicolumn{2}{p{\paraWidth}|}{} \\\hline
\end{tabular*}

\vspace{0.5cm}\noindent \begin{tabular*}{\tableWidth}{|c|l@{\extracolsep{\fill}}r|}
\hline
\multicolumn{1}{|p{\maxVarWidth}}{smoothing\_factor} & {\bf Scope:} private & REAL \\\hline
\multicolumn{3}{|p{\descWidth}|}{{\bf Description:}   {\em Initial moothing factor}} \\
\hline{\bf Range} & &  {\bf Default:} 1.2 \\\multicolumn{1}{|p{\maxVarWidth}|}{\centering (0:2)} & \multicolumn{2}{p{\paraWidth}|}{} \\\hline
\end{tabular*}

\vspace{0.5cm}\noindent \begin{tabular*}{\tableWidth}{|c|l@{\extracolsep{\fill}}r|}
\hline
\multicolumn{1}{|p{\maxVarWidth}}{smoothing\_function} & {\bf Scope:} private & KEYWORD \\\hline
\multicolumn{3}{|p{\descWidth}|}{{\bf Description:}   {\em Smoothing function}} \\
\hline{\bf Range} & &  {\bf Default:} linear \\\multicolumn{1}{|p{\maxVarWidth}|}{\centering linear} & \multicolumn{2}{p{\paraWidth}|}{linear ramp} \\\multicolumn{1}{|p{\maxVarWidth}|}{\centering spline} & \multicolumn{2}{p{\paraWidth}|}{cubic spline ramp} \\\multicolumn{1}{|p{\maxVarWidth}|}{\centering cosine} & \multicolumn{2}{p{\paraWidth}|}{cosine ramp} \\\hline
\end{tabular*}

\vspace{0.5cm}\noindent \begin{tabular*}{\tableWidth}{|c|l@{\extracolsep{\fill}}r|}
\hline
\multicolumn{1}{|p{\maxVarWidth}}{smoothing\_iterations} & {\bf Scope:} private & INT \\\hline
\multicolumn{3}{|p{\descWidth}|}{{\bf Description:}   {\em Smoothing iterations}} \\
\hline{\bf Range} & &  {\bf Default:} 10 \\\multicolumn{1}{|p{\maxVarWidth}|}{\centering 0:*} & \multicolumn{2}{p{\paraWidth}|}{} \\\hline
\end{tabular*}

\vspace{0.5cm}\noindent \begin{tabular*}{\tableWidth}{|c|l@{\extracolsep{\fill}}r|}
\hline
\multicolumn{1}{|p{\maxVarWidth}}{smoothing\_order} & {\bf Scope:} private & INT \\\hline
\multicolumn{3}{|p{\descWidth}|}{{\bf Description:}   {\em Order of the derivatives used for CG smoothing}} \\
\hline{\bf Range} & &  {\bf Default:} 6 \\\multicolumn{1}{|p{\maxVarWidth}|}{\centering 2:6:2} & \multicolumn{2}{p{\paraWidth}|}{} \\\hline
\end{tabular*}

\vspace{0.5cm}\noindent \begin{tabular*}{\tableWidth}{|c|l@{\extracolsep{\fill}}r|}
\hline
\multicolumn{1}{|p{\maxVarWidth}}{smoothing\_zone\_width} & {\bf Scope:} private & REAL \\\hline
\multicolumn{3}{|p{\descWidth}|}{{\bf Description:}   {\em Relative width of smoothing zone inside the region}} \\
\hline{\bf Range} & &  {\bf Default:} 0.0 \\\multicolumn{1}{|p{\maxVarWidth}|}{\centering 0.0:1.0} & \multicolumn{2}{p{\paraWidth}|}{} \\\hline
\end{tabular*}

\vspace{0.5cm}\noindent \begin{tabular*}{\tableWidth}{|c|l@{\extracolsep{\fill}}r|}
\hline
\multicolumn{1}{|p{\maxVarWidth}}{surface\_index} & {\bf Scope:} private & INT \\\hline
\multicolumn{3}{|p{\descWidth}|}{{\bf Description:}   {\em Spherical surface index}} \\
\hline{\bf Range} & &  {\bf Default:} (none) \\\multicolumn{1}{|p{\maxVarWidth}|}{\centering 0:*} & \multicolumn{2}{p{\paraWidth}|}{must be an index of a spherical surface} \\\hline
\end{tabular*}

\vspace{0.5cm}\noindent \begin{tabular*}{\tableWidth}{|c|l@{\extracolsep{\fill}}r|}
\hline
\multicolumn{1}{|p{\maxVarWidth}}{use\_user\_regions} & {\bf Scope:} private & BOOLEAN \\\hline
\multicolumn{3}{|p{\descWidth}|}{{\bf Description:}   {\em Use user defined regions for the smoothing regions}} \\
\hline & & {\bf Default:} no \\\hline
\end{tabular*}

\vspace{0.5cm}\noindent \begin{tabular*}{\tableWidth}{|c|l@{\extracolsep{\fill}}r|}
\hline
\multicolumn{1}{|p{\maxVarWidth}}{verbose} & {\bf Scope:} private & BOOLEAN \\\hline
\multicolumn{3}{|p{\descWidth}|}{{\bf Description:}   {\em Produce some screen output}} \\
\hline & & {\bf Default:} no \\\hline
\end{tabular*}

\vspace{0.5cm}\parskip = 10pt 

\section{Interfaces} 


\parskip = 0pt

\vspace{3mm} \subsection*{General}

\noindent {\bf Implements}: 

noexcision
\vspace{2mm}

\noindent {\bf Inherits}: 

admbase

staticconformal

grid

sphericalsurface

boundary
\vspace{2mm}
\subsection*{Grid Variables}
\vspace{5mm}\subsubsection{PRIVATE GROUPS}

\vspace{5mm}

\begin{tabular*}{150mm}{|c|c@{\extracolsep{\fill}}|rl|} \hline 
~ {\bf Group Names} ~ & ~ {\bf Variable Names} ~  &{\bf Details} ~ & ~\\ 
\hline 
smask &  & compact & 0 \\ 
 & nes\_mask & description & mask for smoothing \\ 
 &  & dimensions & 3 \\ 
 &  & distribution & DEFAULT \\ 
 &  & group type & GF \\ 
 &  & tags & tensortypealias="scalar" Prolongation="None" \\ 
 &  & timelevels & 1 \\ 
 &  & variable type & INT \\ 
\hline 
reduction\_mask &  & compact & 0 \\ 
 & red\_mask & description & Copy of the weight grid function from CarpetReduce \\ 
 &  & dimensions & 3 \\ 
 &  & distribution & DEFAULT \\ 
 &  & group type & GF \\ 
 &  & tags & Prolongation="None" InterpNumTimelevels=1 checkpoint="no" \\ 
 &  & timelevels & 1 \\ 
 &  & variable type & REAL \\ 
\hline 
cg\_res\_metric &  & compact & 0 \\ 
 & resgxx & description & Conjugate Gradient residual for the metric \\ 
 & resgxy & dimensions & 3 \\ 
 & resgxz & distribution & DEFAULT \\ 
 & resgyy & group type & GF \\ 
 & resgyz & tags & tensortypealias="dd\_sym" Prolongation="None" \\ 
 & resgzz & timelevels & 1 \\ 
 &  & variable type & REAL \\ 
\hline 
cg\_res\_curv &  & compact & 0 \\ 
 & reskxx & description & Conjugate Gradient residual for the extrinsic curvature \\ 
 & reskxy & dimensions & 3 \\ 
 & reskxz & distribution & DEFAULT \\ 
 & reskyy & group type & GF \\ 
 & reskyz & tags & tensortypealias="dd\_sym" Prolongation="None" \\ 
 & reskzz & timelevels & 1 \\ 
 &  & variable type & REAL \\ 
\hline 
cg\_res\_shift &  & compact & 0 \\ 
 & resx & description & Conjugate Gradient residual for the shift \\ 
 & resy & dimensions & 3 \\ 
 & resz & distribution & DEFAULT \\ 
 &  & group type & GF \\ 
 &  & tags & tensortypealias="u" Prolongation="None" \\ 
 &  & timelevels & 1 \\ 
 &  & variable type & REAL \\ 
\hline 
cg\_res\_lapse &  & compact & 0 \\ 
 & res & description & Conjugate Gradient residual for the lapse \\ 
 &  & dimensions & 3 \\ 
 &  & distribution & DEFAULT \\ 
 &  & group type & GF \\ 
 &  & tags & tensortypealias="scalar" Prolongation="None" \\ 
 &  & timelevels & 1 \\ 
 &  & variable type & REAL \\ 
\hline 
\end{tabular*} 



\vspace{5mm}
\vspace{5mm}

\begin{tabular*}{150mm}{|c|c@{\extracolsep{\fill}}|rl|} \hline 
~ {\bf Group Names} ~ & ~ {\bf Variable Names} ~  &{\bf Details} ~ & ~ \\ 
\hline 
cg\_d\_metric &  & compact & 0 \\ 
 & dgxx & description & Conjugate Gradient d for the metric \\ 
 & dgxy & dimensions & 3 \\ 
 & dgxz & distribution & DEFAULT \\ 
 & dgyy & group type & GF \\ 
 & dgyz & tags & tensortypealias="dd\_sym" Prolongation="None" \\ 
 & dgzz & timelevels & 1 \\ 
 &  & variable type & REAL \\ 
\hline 
cg\_d\_curv &  & compact & 0 \\ 
 & dkxx & description & Conjugate Gradient d for the extrinsic curvature \\ 
 & dkxy & dimensions & 3 \\ 
 & dkxz & distribution & DEFAULT \\ 
 & dkyy & group type & GF \\ 
 & dkyz & tags & tensortypealias="dd\_sym" Prolongation="None" \\ 
 & dkzz & timelevels & 1 \\ 
 &  & variable type & REAL \\ 
\hline 
cg\_d\_shift &  & compact & 0 \\ 
 & dx & description & Conjugate Gradient d for the shift \\ 
 & dy & dimensions & 3 \\ 
 & dz & distribution & DEFAULT \\ 
 &  & group type & GF \\ 
 &  & tags & tensortypealias="u" Prolongation="None" \\ 
 &  & timelevels & 1 \\ 
 &  & variable type & REAL \\ 
\hline 
cg\_d\_lapse &  & compact & 0 \\ 
 & d & description & Conjugate Gradient d for the lapse \\ 
 &  & dimensions & 3 \\ 
 &  & distribution & DEFAULT \\ 
 &  & group type & GF \\ 
 &  & tags & tensortypealias="scalar" Prolongation="None" \\ 
 &  & timelevels & 1 \\ 
 &  & variable type & REAL \\ 
\hline 
cg\_q\_metric &  & compact & 0 \\ 
 & qgxx & description & Conjugate Gradient q for the metric \\ 
 & qgxy & dimensions & 3 \\ 
 & qgxz & distribution & DEFAULT \\ 
 & qgyy & group type & GF \\ 
 & qgyz & tags & tensortypealias="dd\_sym" Prolongation="None" \\ 
 & qgzz & timelevels & 1 \\ 
 &  & variable type & REAL \\ 
\hline 
cg\_q\_curv &  & compact & 0 \\ 
 & qkxx & description & Conjugate Gradient q for the extrinsic curvature \\ 
 & qkxy & dimensions & 3 \\ 
 & qkxz & distribution & DEFAULT \\ 
 & qkyy & group type & GF \\ 
 & qkyz & tags & tensortypealias="dd\_sym" Prolongation="None" \\ 
 & qkzz & timelevels & 1 \\ 
 &  & variable type & REAL \\ 
\hline 
\end{tabular*} 



\vspace{5mm}
\vspace{5mm}

\begin{tabular*}{150mm}{|c|c@{\extracolsep{\fill}}|rl|} \hline 
~ {\bf Group Names} ~ & ~ {\bf Variable Names} ~  &{\bf Details} ~ & ~ \\ 
\hline 
cg\_q\_shift &  & compact & 0 \\ 
 & qx & description & Conjugate Gradient q for the shift \\ 
 & qy & dimensions & 3 \\ 
 & qz & distribution & DEFAULT \\ 
 &  & group type & GF \\ 
 &  & tags & tensortypealias="u" Prolongation="None" \\ 
 &  & timelevels & 1 \\ 
 &  & variable type & REAL \\ 
\hline 
cg\_q\_lapse &  & compact & 0 \\ 
 & q & description & Conjugate Gradient q for the lapse \\ 
 &  & dimensions & 3 \\ 
 &  & distribution & DEFAULT \\ 
 &  & group type & GF \\ 
 &  & tags & tensortypealias="scalar" Prolongation="None" \\ 
 &  & timelevels & 1 \\ 
 &  & variable type & REAL \\ 
\hline 
cg\_red\_all &  & compact & 0 \\ 
 & redgxx & description & Conjugate Gradient red for all variables \\ 
 & redgxy & dimensions & 3 \\ 
 & redgxz & distribution & DEFAULT \\ 
 & redgyy & group type & GF \\ 
 & redgyz & tags & tensortypealias="scalar" Prolongation="None" \\ 
 & redgzz & timelevels & 1 \\ 
 & redkxx & variable type & REAL \\ 
\hline 
loop\_control & loop\_control & compact & 0 \\ 
 &  & dimensions & 0 \\ 
 &  & distribution & CONSTANT \\ 
 &  & group type & SCALAR \\ 
 &  & timelevels & 1 \\ 
 &  & variable type & INT \\ 
\hline 
\end{tabular*} 



\vspace{5mm}

\noindent {\bf Uses header}: 

Boundary.h

carpet.hh
\vspace{2mm}\parskip = 10pt 

\section{Schedule} 


\parskip = 0pt


\noindent This section lists all the variables which are assigned storage by thorn EinsteinInitialData/NoExcision.  Storage can either last for the duration of the run ({\bf Always} means that if this thorn is activated storage will be assigned, {\bf Conditional} means that if this thorn is activated storage will be assigned for the duration of the run if some condition is met), or can be turned on for the duration of a schedule function.


\subsection*{Storage}

\hspace{5mm}

 \begin{tabular*}{160mm}{ll} 

{\bf Always:}&  ~ \\ 
 reduction\_mask & ~\\ 
~ & ~\\ 
\end{tabular*} 


\subsection*{Scheduled Functions}
\vspace{5mm}

\noindent {\bf ADMBase\_PostInitial}   (conditional) 

\hspace{5mm} noexcision\_overwrite 

\hspace{5mm}{\it overwrite regions with minkowski } 


\hspace{5mm}

 \begin{tabular*}{160mm}{cll} 
~ & Language:  & fortran \\ 
~ & Type:  & function \\ 
~ & Writes:  & admbase::metric(everywhere) \\ 
~& ~ &admbase::curv(everywhere)\\ 
\end{tabular*} 


\vspace{5mm}

\noindent {\bf CCTK\_BASEGRID}   (conditional) 

\hspace{5mm} noexcision\_setsym 

\hspace{5mm}{\it register the symmetries for the conjugate gradient functions. } 


\hspace{5mm}

 \begin{tabular*}{160mm}{cll} 
~ & Language:  & fortran \\ 
~ & Type:  & function \\ 
~ & Writes:  & admbase::metric(everywhere) \\ 
~& ~ &admbase::curv(everywhere)\\ 
\end{tabular*} 


\vspace{5mm}

\noindent {\bf NoExcision\_CGSmoothing}   (conditional) 

\hspace{5mm} noexcision\_smoothing 

\hspace{5mm}{\it smooth regions } 


\hspace{5mm}

 \begin{tabular*}{160mm}{cll} 
~ & After:  & noexcision\_cginit\_2 \\ 
~ & Type:  & group \\ 
~ & While:  & noexcision::loop\_control \\ 
\end{tabular*} 


\vspace{5mm}

\noindent {\bf NoExcision\_Smoothing}   (conditional) 

\hspace{5mm} noexcision\_cg\_1 

\hspace{5mm}{\it conjugate gradients step 1 } 


\hspace{5mm}

 \begin{tabular*}{160mm}{cll} 
~ & Language:  & fortran \\ 
~ & Type:  & function \\ 
\end{tabular*} 


\vspace{5mm}

\noindent {\bf NoExcision\_Smoothing}   (conditional) 

\hspace{5mm} noexcision\_cgapplysym 

\hspace{5mm}{\it select variables for boundary conditions 2 } 


\hspace{5mm}

 \begin{tabular*}{160mm}{cll} 
~ & After:  & noexcision\_cg\_1 \\ 
~ & Language:  & fortran \\ 
~ & Sync:  & cg\_q\_metric \\ 
~& ~ &cg\_q\_curv\\ 
~& ~ &cg\_q\_shift\\ 
~& ~ &cg\_q\_lapse\\ 
~& ~ &cg\_red\_all\\ 
~ & Type:  & function \\ 
\end{tabular*} 


\vspace{5mm}

\noindent {\bf NoExcision\_Smoothing}   (conditional) 

\hspace{5mm} applybcs 

\hspace{5mm}{\it apply boundary conditions (symmetries) 2 } 


\hspace{5mm}

 \begin{tabular*}{160mm}{cll} 
~ & After:  & noexcision\_cgapplysym\_p2 \\ 
~ & Type:  & group \\ 
\end{tabular*} 


\vspace{5mm}

\noindent {\bf NoExcision\_Smoothing}   (conditional) 

\hspace{5mm} noexcision\_cg\_2 

\hspace{5mm}{\it conjugate gradients step 2 } 


\hspace{5mm}

 \begin{tabular*}{160mm}{cll} 
~ & After:  & noexcision\_cg\_1 \\ 
~ & Language:  & fortran \\ 
~ & Options:  & level \\ 
~ & Type:  & function \\ 
\end{tabular*} 


\vspace{5mm}

\noindent {\bf NoExcision\_Smoothing}   (conditional) 

\hspace{5mm} noexcision\_cg\_3 

\hspace{5mm}{\it conjugate gradients step 3 } 


\hspace{5mm}

 \begin{tabular*}{160mm}{cll} 
~ & Language:  & fortran \\ 
~ & Type:  & function \\ 
\end{tabular*} 


\vspace{5mm}

\noindent {\bf NoExcision\_Smoothing}   (conditional) 

\hspace{5mm} noexcision\_cgapplysym 

\hspace{5mm}{\it select variables for boundary conditions 3 } 


\hspace{5mm}

 \begin{tabular*}{160mm}{cll} 
~ & After:  & noexcision\_cg\_3 \\ 
~ & Language:  & fortran \\ 
~ & Sync:  & cg\_res\_metric \\ 
~& ~ &cg\_res\_curv\\ 
~& ~ &cg\_res\_shift\\ 
~& ~ &cg\_res\_lapse\\ 
~& ~ &metric\\ 
~& ~ &curv\\ 
~& ~ &shift\\ 
~& ~ &lapse\\ 
~& ~ &cg\_red\_all\\ 
~ & Type:  & function \\ 
\end{tabular*} 


\vspace{5mm}

\noindent {\bf NoExcision\_Smoothing}   (conditional) 

\hspace{5mm} applybcs 

\hspace{5mm}{\it apply boundary conditions (symmetries) 3 } 


\hspace{5mm}

 \begin{tabular*}{160mm}{cll} 
~ & After:  & noexcision\_cgapplysym\_p3 \\ 
~ & Type:  & group \\ 
\end{tabular*} 


\vspace{5mm}

\noindent {\bf NoExcision\_Smoothing}   (conditional) 

\hspace{5mm} noexcision\_cg\_4 

\hspace{5mm}{\it conjugate gradients step 4 } 


\hspace{5mm}

 \begin{tabular*}{160mm}{cll} 
~ & After:  & noexcision\_cg\_1 \\ 
~ & Language:  & fortran \\ 
~ & Options:  & level \\ 
~ & Type:  & function \\ 
\end{tabular*} 


\vspace{5mm}

\noindent {\bf NoExcision\_Smoothing}   (conditional) 

\hspace{5mm} noexcision\_cg\_5 

\hspace{5mm}{\it conjugate gradients step 5 } 


\hspace{5mm}

 \begin{tabular*}{160mm}{cll} 
~ & Language:  & fortran \\ 
~ & Type:  & function \\ 
\end{tabular*} 


\vspace{5mm}

\noindent {\bf ADMBase\_PostInitial}   (conditional) 

\hspace{5mm} noexcision\_smooth 

\hspace{5mm}{\it smooth regions } 


\hspace{5mm}

 \begin{tabular*}{160mm}{cll} 
~ & After:  & noexcision\_overwrite \\ 
~ & Language:  & fortran \\ 
~ & Type:  & function \\ 
\end{tabular*} 


\vspace{5mm}

\noindent {\bf NoExcision\_CGSmoothing}   (conditional) 

\hspace{5mm} noexcision\_cgapplysym 

\hspace{5mm}{\it select variables for boundary conditions 4 } 


\hspace{5mm}

 \begin{tabular*}{160mm}{cll} 
~ & After:  & noexcision\_cg\_5 \\ 
~ & Language:  & fortran \\ 
~ & Sync:  & cg\_d\_metric \\ 
~& ~ &cg\_d\_curv\\ 
~& ~ &cg\_d\_shift\\ 
~& ~ &cg\_d\_lapse\\ 
~ & Type:  & function \\ 
\end{tabular*} 


\vspace{5mm}

\noindent {\bf NoExcision\_CGSmoothing}   (conditional) 

\hspace{5mm} applybcs 

\hspace{5mm}{\it apply boundary conditions (symmetries) 4 } 


\hspace{5mm}

 \begin{tabular*}{160mm}{cll} 
~ & After:  & noexcision\_cgapplysym\_p4 \\ 
~ & Type:  & group \\ 
\end{tabular*} 


\vspace{5mm}

\noindent {\bf MoL\_PostRHS} 

\hspace{5mm} noexcision\_reduce 

\hspace{5mm}{\it reduce rhs } 


\hspace{5mm}

 \begin{tabular*}{160mm}{cll} 
~ & Language:  & c \\ 
~ & Type:  & function \\ 
\end{tabular*} 


\vspace{5mm}

\noindent {\bf ADMBase\_PostInitial}   (conditional) 

\hspace{5mm} noexcision\_set\_zero 

\hspace{5mm}{\it set variables to zero in user defined regions } 


\hspace{5mm}

 \begin{tabular*}{160mm}{cll} 
~ & Before:  & noexcision\_cgsmoothing \\ 
~ & Language:  & fortran \\ 
~ & Type:  & function \\ 
\end{tabular*} 


\vspace{5mm}

\noindent {\bf ADMBase\_PostInitial}   (conditional) 

\hspace{5mm} noexcision\_cgsmoothing 

\hspace{5mm}{\it conjugate gradient smoothing } 


\hspace{5mm}

 \begin{tabular*}{160mm}{cll} 
~ & Storage:  & cg\_res\_metric \\ 
~& ~ &cg\_res\_curv\\ 
~& ~ &cg\_res\_shift\\ 
~& ~ &cg\_res\_lapse\\ 
~& ~ &cg\_d\_metric\\ 
~& ~ &cg\_d\_curv\\ 
~& ~ &cg\_d\_shift\\ 
~& ~ &cg\_d\_lapse\\ 
~& ~ &cg\_q\_metric\\ 
~& ~ &cg\_q\_curv\\ 
~& ~ &cg\_q\_shift\\ 
~& ~ &cg\_q\_lapse\\ 
~& ~ &cg\_red\_all\\ 
~& ~ &smask\\ 
~& ~ &loop\_control\\ 
~ & Type:  & group \\ 
\end{tabular*} 


\vspace{5mm}

\noindent {\bf NoExcision\_CGSmoothing}   (conditional) 

\hspace{5mm} copymask 

\hspace{5mm}{\it copy the weight function from carpetreduce } 


\hspace{5mm}

 \begin{tabular*}{160mm}{cll} 
~ & Language:  & c \\ 
~ & Options:  & global \\ 
~& ~ &loop-local\\ 
~ & Type:  & function \\ 
\end{tabular*} 


\vspace{5mm}

\noindent {\bf NoExcision\_CGSmoothing}   (conditional) 

\hspace{5mm} noexcision\_cginit\_1 

\hspace{5mm}{\it initialise the conjugate gradient method 1 } 


\hspace{5mm}

 \begin{tabular*}{160mm}{cll} 
~ & After:  & copymask \\ 
~ & Language:  & fortran \\ 
~ & Reads:  & admbase::metric \\ 
~& ~ &admbase::curv\\ 
~& ~ &admbase::alp\\ 
~& ~ &admbase::shift\\ 
~& ~ &red\_mask\\ 
~ & Type:  & function \\ 
~ & Writes:  & cg\_res\_shift(everywhere) \\ 
~& ~ &cg\_res\_lapse(everywhere)\\ 
~& ~ &cg\_d\_metric(everywhere)\\ 
~& ~ &cg\_d\_curv(everywhere)\\ 
~& ~ &cg\_red\_all(everywhere)\\ 
~& ~ &cg\_d\_lapse(everywhere)\\ 
~& ~ &cg\_d\_shift(everywhere)\\ 
~& ~ &cg\_q\_metric(everywhere)\\ 
~& ~ &cg\_q\_shift(everywhere)\\ 
~& ~ &cg\_q\_curv(everywhere)\\ 
~& ~ &cg\_q\_lapse(everywhere)\\ 
~& ~ &loop\_control\\ 
~& ~ &nes\_mask(everywhere)\\ 
~& ~ &cg\_res\_metric(everywhere)\\ 
~& ~ &cg\_res\_curv(everywhere)\\ 
\end{tabular*} 


\vspace{5mm}

\noindent {\bf NoExcision\_CGSmoothing}   (conditional) 

\hspace{5mm} noexcision\_cgapplysym 

\hspace{5mm}{\it select variables for boundary conditions 1 } 


\hspace{5mm}

 \begin{tabular*}{160mm}{cll} 
~ & After:  & noexcision\_cginit\_1 \\ 
~ & Language:  & fortran \\ 
~ & Sync:  & cg\_d\_metric \\ 
~& ~ &cg\_d\_curv\\ 
~& ~ &cg\_d\_shift\\ 
~& ~ &cg\_d\_lapse\\ 
~& ~ &cg\_res\_metric\\ 
~& ~ &cg\_res\_curv\\ 
~& ~ &cg\_res\_shift\\ 
~& ~ &cg\_res\_lapse\\ 
~& ~ &cg\_red\_all\\ 
~ & Type:  & function \\ 
\end{tabular*} 


\vspace{5mm}

\noindent {\bf NoExcision\_CGSmoothing}   (conditional) 

\hspace{5mm} applybcs 

\hspace{5mm}{\it apply boundary conditions (symmetries) 1 } 


\hspace{5mm}

 \begin{tabular*}{160mm}{cll} 
~ & After:  & noexcision\_cgapplysym\_p1 \\ 
~ & Type:  & group \\ 
\end{tabular*} 


\vspace{5mm}

\noindent {\bf NoExcision\_CGSmoothing}   (conditional) 

\hspace{5mm} noexcision\_cginit\_2 

\hspace{5mm}{\it initialise the conjugate gradient method 2 } 


\hspace{5mm}

 \begin{tabular*}{160mm}{cll} 
~ & After:  & noexcision\_cgapplysym\_p1 \\ 
~ & Language:  & fortran \\ 
~ & Options:  & level \\ 
~ & Reads:  & admbase::alp(everywhere) \\ 
~& ~ &admbase::gxx(everywhere)\\ 
~ & Type:  & function \\ 
\end{tabular*} 


\subsection*{Aliased Functions}

\hspace{5mm}

 \begin{tabular*}{160mm}{ll} 

{\bf Alias Name:} ~~~~~~~ & {\bf Function Name:} \\ 
ApplyBCs & NoExcision\_CGApplyBCs\_p1 \\ 
NoExcision\_CGApplySym & NoExcision\_CGApplySym\_p1 \\ 
\end{tabular*} 



\vspace{5mm}\parskip = 10pt 
\end{document}
