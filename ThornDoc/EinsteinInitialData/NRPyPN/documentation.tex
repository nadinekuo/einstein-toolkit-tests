% *======================================================================*
%  Cactus Thorn template for ThornGuide documentation
%  Author: Ian Kelley
%  Date: Sun Jun 02, 2002
%  $Header$
%
%  Thorn documentation in the latex file doc/documentation.tex
%  will be included in ThornGuides built with the Cactus make system.
%  The scripts employed by the make system automatically include
%  pages about variables, parameters and scheduling parsed from the
%  relevant thorn CCL files.
%
%  This template contains guidelines which help to assure that your
%  documentation will be correctly added to ThornGuides. More
%  information is available in the Cactus UsersGuide.
%
%  Guidelines:
%   - Do not change anything before the line
%       % START CACTUS THORNGUIDE",
%     except for filling in the title, author, date, etc. fields.
%        - Each of these fields should only be on ONE line.
%        - Author names should be separated with a \\ or a comma.
%   - You can define your own macros, but they must appear after
%     the START CACTUS THORNGUIDE line, and must not redefine standard
%     latex commands.
%   - To avoid name clashes with other thorns, 'labels', 'citations',
%     'references', and 'image' names should conform to the following
%     convention:
%       ARRANGEMENT_THORN_LABEL
%     For example, an image wave.eps in the arrangement CactusWave and
%     thorn WaveToyC should be renamed to CactusWave_WaveToyC_wave.eps
%   - Graphics should only be included using the graphicx package.
%     More specifically, with the "\includegraphics" command.  Do
%     not specify any graphic file extensions in your .tex file. This
%     will allow us to create a PDF version of the ThornGuide
%     via pdflatex.
%   - References should be included with the latex "\bibitem" command.
%   - Use \begin{abstract}...\end{abstract} instead of \abstract{...}
%   - Do not use \appendix, instead include any appendices you need as
%     standard sections.
%   - For the benefit of our Perl scripts, and for future extensions,
%     please use simple latex.
%
% *======================================================================*
%
% Example of including a graphic image:
%    \begin{figure}[ht]
% 	\begin{center}
%    	   \includegraphics[width=6cm]{/home/runner/work/einstein-toolkit-tests/einstein-toolkit-tests/arrangements/EinsteinInitialData/NRPyPN/doc/MyArrangement_MyThorn_MyFigure}
% 	\end{center}
% 	\caption{Illustration of this and that}
% 	\label{MyArrangement_MyThorn_MyLabel}
%    \end{figure}
%
% Example of using a label:
%   \label{MyArrangement_MyThorn_MyLabel}
%
% Example of a citation:
%    \cite{MyArrangement_MyThorn_Author99}
%
% Example of including a reference
%   \bibitem{MyArrangement_MyThorn_Author99}
%   {J. Author, {\em The Title of the Book, Journal, or periodical}, 1 (1999),
%   1--16. {\tt http://www.nowhere.com/}}
%
% *======================================================================*

% If you are using CVS use this line to give version information
% $Header$

\documentclass{article}

% Use the Cactus ThornGuide style file
% (Automatically used from Cactus distribution, if you have a
%  thorn without the Cactus Flesh download this from the Cactus
%  homepage at www.cactuscode.org)
\usepackage{../../../../../doc/latex/cactus}

\newlength{\tableWidth} \newlength{\maxVarWidth} \newlength{\paraWidth} \newlength{\descWidth} \begin{document}

% The author of the documentation
\author{Zachariah B.~Etienne \textless zachetie *at* gmail *dot* com\textgreater}

% The title of the document (not necessarily the name of the Thorn)
\title{\texttt{NRPyPN}: Validated Post-Newtonian Expressions for Input
  into Wolfram Mathematica, SymPy, or Highly-Optimized C Codes. Also
  provides low-eccentricity binary black hole initial data parameters
  for Bowen-York initial data (e.g., TwoPunctures)}

% the date your document was last changed, if your document is in CVS,
% please use:
%    \date{$ $Date$ $}
\date{December 22, 2020}

\maketitle

% Do not delete next line
% START CACTUS THORNGUIDE

% Add all definitions used in this documentation here
%   \def\mydef etc

% Add an abstract for this thorn's documentation
Post-Newtonian theory results in some of the longest and most complex
mathematical expressions ever derived by humanity.

These expressions form the core of most gravitational wave data
analysis codes, as well as codes used to construct low-eccentricity
initial data for numerical relativity simulations of compact binary
systems (e.g., using \texttt{TwoPunctures}), but generally the
expressions are written in a format 
that is either inaccessible by others (i.e., closed-source) or written
directly in e.g., C code and thus incompatible with symbolic algebra
packages like Wolfram Mathematica or the
Python-based SymPy.

Once in a symbolic algebra package, these expressions could be
manipulated, extended, and output as {\it more} optimized C codes
(e.g., using SymPy/NRPy+, thus speeding up gravitational wave data
analysis and making it easier to quickly get needed low-eccentricity
parameters for setting up binary black hole initial data).

NRPyPN aims to provide a trusted source for validated
Post-Newtonian expressions useful for gravitational wave astronomy,
using the open-source SymPy computer algebra
system, so that expressions can be output into Wolfram
Mathematica or highly optimized C codes using the SymPy-based
\texttt{NRPy+}.

In particular, \texttt{NRPyPN} exists within the \texttt{Einstein
  Toolkit} to provide a convenient interface for computing
parameters for low-eccentricity binary black hole initial data, as
computed by the \texttt{TwoPunctures} thorn. The interface for
accomplishing this is provided by at the bottom of the
well-documented, interactive Jupyter notebook located at
\texttt{NRPyPN.ipynb} in the Einstein Toolkit. Instructions for
getting the Jupyter notebook up and running are provided in the
\texttt{README} file.

If you do not have Jupyter installed the online version
at~\cite{EinsteinInitialData_NRPyPN_MyBinder} may be used instead.

\begin{thebibliography}{9}

\bibitem{EinsteinInitialData_NRPyPN_MyBinder}
MyBinder online NRPyPN notebook,
\url{https://mybinder.org/v2/gh/zachetienne/nrpytutorial/master?filepath=NRPyPN%2FNRPyPN.ipynb}.

\end{thebibliography}


% Do not delete next line
% END CACTUS THORNGUIDE



\section{Parameters} 


\parskip = 0pt
\parskip = 10pt 

\section{Schedule} 


\parskip = 0pt


\noindent This section lists all the variables which are assigned storage by thorn EinsteinInitialData/NRPyPN.  Storage can either last for the duration of the run ({\bf Always} means that if this thorn is activated storage will be assigned, {\bf Conditional} means that if this thorn is activated storage will be assigned for the duration of the run if some condition is met), or can be turned on for the duration of a schedule function.


\subsection*{Storage}NONE

\vspace{5mm}\parskip = 10pt 
\end{document}
