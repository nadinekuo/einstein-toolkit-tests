\documentclass{article}

% Use the Cactus ThornGuide style file
% (Automatically used from Cactus distribution, if you have a 
%  thorn without the Cactus Flesh download this from the Cactus
%  homepage at www.cactuscode.org)
\usepackage{../../../../../doc/latex/cactus}

\newlength{\tableWidth} \newlength{\maxVarWidth} \newlength{\paraWidth} \newlength{\descWidth} \begin{document}

\title{SystemStatistics}
\author{Ian Hinder}
\date{$ $Date: 2010/04/06 11:00:00 $ $}

\maketitle

% Do not delete next line
% START CACTUS THORNGUIDE
\begin{abstract}
  Thorn {\bf SystemStatistics} collects information about the system
  on which a Cactus process is running and stores it in Cactus
  variables.  These variables can then be output and reduced using the
  standard Cactus methods such as IOBasic and IOScalar.
\end{abstract}


\section{Purpose}
When running a Cactus simulation, there are often features of the
system on which it is running which can affect the simulation.  For
example, Cactus processes take a certain amount of memory, and if
this exceeds the memory available on the system, the Cactus process
will either start to swap, significantly reducing performance, or will
terminate.  It can be useful to determine quantities such as memory
usage of a Cactus process, and that is the purpose of this thorn.
%
The thorn currently provides grid arrays which are filled with
memory-related information.

\section{{\bf SystemStatistics} Parameters}
%
This thorn has no parameters.
%
%
\section{Examples}
%
\begin{verbatim}
IOBasic::outInfo_every                  = 256
IOBasic::outInfo_reductions             = "minimum maximum"
IOBasic::outInfo_vars                   = "SystemStatistics::maxrss_mb"

IOScalar::outScalar_every               = 256
IOScalar::outScalar_reductions          = "minimum maximum average"
IOScalar::outScalar_vars                = "SystemStatistics::process_memory_mb"
\end{verbatim}

The resulting screen output would look like this:
\begin{verbatim}
-----------------------------------------
Iteration      Time | *TISTICS::maxrss_mb
                    |   minimum   maximum
-----------------------------------------
        0     0.000 |        52        54
      256     0.005 |        53        55
      512     0.010 |        53        55
      768     0.015 |        53        55
     1024     0.020 |        53        55
\end{verbatim}
%
and the reduced scalar output in {\tt
  \verb|systemstatistics::process_memory_mb.maximum.asc|} would look like
this:
\begin{verbatim}
# SYSTEMSTATISTICS::maxrss_mb (systemstatistics::process_memory_mb)
# 1:iteration 2:time 3:data
# data columns: 3:maxrss_mb 4:majflt_mb 5:arena_mb 6:ordblks_mb 7:hblks_mb 8:hblkhd_mb 9:uordblks_mb 10:fordblks_mb 11:keepcost_mb
0 0 54 7 50 0 0 0 39 12 11
256 0.00498610682309355 55 7 51 0 0 0 39 12 11
512 0.0099722136461871 55 7 51 0 0 0 39 12 11
768 0.0149583204692806 55 7 51 0 0 0 39 12 11
1024 0.0199444272923742 55 7 51 0 0 0 39 12 11
\end{verbatim}

\section{Comments}

\begin{itemize}
\item There has been some discussion concerning whether it would be
  better to use Cactus clocks to store this information.
\item The current implementation stores memory usage in \verb$CCTK_INT$
  variables.  Since memory sizes of many gigabytes are common, this
  could lead to overflow.  It would probably be better to use
  \verb$CCTK_REAL$ variables instead.  
\item The thorn currently has variables for bytes, kilobytes and
  megabytes for each quantity.  It might be better to remove these and
  let the output mechanism format the variables in an easy to
  understand form.
\item This thorn uses information from the /proc filesystem where it
  is available.  The Resident Set Size (RSS) is also read correctly on
  Mac OS, where /proc is not available.
\end{itemize}

% Do not delete next line
% END CACTUS THORNGUIDE



\section{Parameters} 


\parskip = 0pt
\parskip = 10pt 

\section{Interfaces} 


\parskip = 0pt

\vspace{3mm} \subsection*{General}

\noindent {\bf Implements}: 

systemstatistics
\vspace{2mm}
\subsection*{Grid Variables}
\vspace{5mm}\subsubsection{PRIVATE GROUPS}

\vspace{5mm}

\begin{tabular*}{150mm}{|c|c@{\extracolsep{\fill}}|rl|} \hline 
~ {\bf Group Names} ~ & ~ {\bf Variable Names} ~  &{\bf Details} ~ & ~\\ 
\hline 
process\_memory &  & compact & 0 \\ 
 & maxrss & description & Process memory statistics \\ 
 & majflt & dimensions & 1 \\ 
 & arena & distribution & CONSTANT \\ 
 & ordblks & group type & ARRAY \\ 
 & hblks & size & 1 \\ 
 & hblkhd & tags & Checkpoint="no" \\ 
 & uordblks & timelevels & 1 \\ 
 & fordblks & variable type & REAL \\ 
\hline 
process\_memory\_mb &  & compact & 0 \\ 
 & maxrss\_mb & description & Run memory statistics in MB \\ 
 & majflt\_mb & dimensions & 1 \\ 
 & arena\_mb & distribution & CONSTANT \\ 
 & ordblks\_mb & group type & ARRAY \\ 
 & hblks\_mb & size & 1 \\ 
 & hblkhd\_mb & tags & Checkpoint="no" \\ 
 & uordblks\_mb & timelevels & 1 \\ 
 & fordblks\_mb & variable type & INT \\ 
\hline 
process\_memory\_kb &  & compact & 0 \\ 
 & maxrss\_kb & description & Run memory statistics in KB \\ 
 & majflt\_kb & dimensions & 1 \\ 
 & arena\_kb & distribution & CONSTANT \\ 
 & ordblks\_kb & group type & ARRAY \\ 
 & hblks\_kb & size & 1 \\ 
 & hblkhd\_kb & tags & Checkpoint="no" \\ 
 & uordblks\_kb & timelevels & 1 \\ 
 & fordblks\_kb & variable type & INT \\ 
\hline 
\end{tabular*} 



\vspace{5mm}\parskip = 10pt 

\section{Schedule} 


\parskip = 0pt


\noindent This section lists all the variables which are assigned storage by thorn CactusUtils/SystemStatistics.  Storage can either last for the duration of the run ({\bf Always} means that if this thorn is activated storage will be assigned, {\bf Conditional} means that if this thorn is activated storage will be assigned for the duration of the run if some condition is met), or can be turned on for the duration of a schedule function.


\subsection*{Storage}

\hspace{5mm}

 \begin{tabular*}{160mm}{ll} 

{\bf Always:}&  ~ \\ 
 process\_memory process\_memory\_mb process\_memory\_kb & ~\\ 
~ & ~\\ 
\end{tabular*} 


\subsection*{Scheduled Functions}
\vspace{5mm}

\noindent {\bf CCTK\_ANALYSIS} 

\hspace{5mm} systemstatistics\_collect 

\hspace{5mm}{\it collect system statistics } 


\hspace{5mm}

 \begin{tabular*}{160mm}{cll} 
~ & Language:  & c \\ 
~ & Options:  & global \\ 
~ & Type:  & function \\ 
~ & Writes:  & systemstatistics::process\_memory \\ 
~& ~ &process\_memory\_mb\\ 
~& ~ &process\_memory\_kb\\ 
\end{tabular*} 



\vspace{5mm}\parskip = 10pt 
\end{document}
