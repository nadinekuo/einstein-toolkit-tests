\documentclass{article}

\usepackage{../../../../doc/latex/cactus}

\begin{document}

\author{Barry Wardell and Ian Hinder}
\title{EinsteinExact}
\date{10 May 2012}

\maketitle

% Do not delete next line
% START CACTUS THORNGUIDE

\begin{abstract}
The \textsc{EinsteinExact} arrangement provides a collection of thorns which compute
exact solutions to the Einstein equation and set the $3+1$ ADM variables
provided by the \textsc{ADMBase} thorn. The solutions may be computed either as initial
data, or at each iteration of an evolution. The computation is done analytically
from expressions for the metric which are taken from a database of solutions
provided by the \textsc{Metrics} package for \textsc{Mathematica}. Adding new
solutions to \textsc{EinsteinExact} only requires that the metric be added to this
database.
\end{abstract}

\section{Introduction}
The \textsc{EinsteinExact} code relies on the fact that a given analytic
expression for the $4$-metric $g_{ab}$ (and its inverse $g^{ab}$) of a spacetime
may be decomposed into the $3+1$ ADM variables,
\begin{equation}
g_{ab} =
\left(
\begin{array}{cc}
 -\alpha^2 + \beta_k \beta^k & \beta_i \\
 \beta_i & \gamma_{ij}\\
\end{array}
\right),\quad
g^{ab} =
\left(
\begin{array}{cc}
 -1/\alpha^2 & \beta^i / \alpha^2 \\
 \beta^i / \alpha^2 & \gamma^{ij} - \beta^i \beta^j / \alpha^2\\
\end{array}
\right),
\end{equation}
where $\alpha$ is the lapse, $\beta^i$ is the shift and $\gamma_{ij}$
is the three metric on the slice. The \textsc{ADMBase} variables for the
lapse, shift and three metric may therefore be directly obtained from
components of the four metric and its inverse:
\begin{equation}
\alpha = \frac{1}{\sqrt{-g_{00}}},\quad
\beta^i = \alpha^2 g^{0i},\quad
\gamma_{ij} = g_{ij}.
\end{equation}
Another \textsc{ADMBase} variable which must be set is the extrinsic
curvature $K_{ij}$. This may be written as
\begin{equation}
K_{ij} = -\frac{1}{2\alpha} \mathcal{L}_{\alpha \vec{n}} \gamma_{ij}
\end{equation}
where $\mathcal{L}$ is the Lie derivative and $\vec{n}$ is the normal to the
slice. Explicitly expanding this out gives
\begin{align}
K_{ij} = & \frac{1}{2\alpha} (-\partial_t \gamma_{ij} + D_i \beta_j + D_j \beta_i) \nonumber \\
 = &  \frac{1}{2\alpha} (-\partial_t \gamma_{ij}
        + \beta^k \partial_k \gamma_{ij}
        + \gamma_{ik} \partial_j \beta^k
        + \gamma_{kj} \partial_i \beta^k
      )
\end{align}
which is readily computed from the previous analytic expressions for $\alpha$,
$\beta^i$ and $\gamma_{ij}$. Finally, the \textsc{ADMBase} variables for the time
derivatives for the lapse and shift are computed by analytically differentiating
$\alpha$ and $\beta^i$ with respect to $t$.

\section{Obtaining EinsteinExact}
The latest version of \textsc{EinsteinExact} will always be available from the
project git repository:

\code{git clone git://github.com/barrywardell/EinsteinExact}

\section{Using EinsteinExact}
The \textsc{EinsteinExact} thorns are designed to work with \textsc{Cactus} as
part of the \textsc{Einstein Toolkit}. All requirements are therefore already included
with the \textsc{Einstein Toolkit}.

To use an \textsc{EinsteinExact} thorn, include the thorn for your desired spacetime in your
thornlist and in the \code{ActiveThorns} section of your parameter file and set either
the \code{ADMBase::initial\_data} or \code{ADMBase::evolution\_method}
parameters (or both) described below. It is also necessary to set the \code{ADMBase::initial\_lapse},
\code{ADMBase::initial\_dtlapse}, \code{ADMBase::initial\_shift} and \code{ADMBase::initial\_dtshift}
parameters in order for storage to be allocated for all \textsc{ADMBase} variables set by EinsteinExact.

As each thorn sets the ADM variables for a
specific spacetime in some particular coordinate system it only makes sense to
have one thorn in the arrangement activated at a time.

\section{Common parameters}
Although there is a separate \textsc{Cactus} thorn for each supported spacetime, all
thorns are generated from the same \textsc{Kranc} script and therefore share some
common parameters. The parameters supported by all thorns are listed in this
section.

\subsubsection{\code{ADMBase::initial\_data}}
The ADM variables can be set by \textsc{EinsteinExact}
as initial data for an evolution. Each
\textsc{EinsteinExact} thorn extends the \code{ADMBase::initial\_data}
keyword parameter with another option corresponding to the name of the thorn.
If the parameter is set to the thorn name, then initial data for the ADM
variables will be set by the thorn in the \code{ADMBase\_InitialData} schedule
bin.

\subsubsection{\code{ADMBase::evolution\_method}}
The setting of the ADM variables can optionally be done at each step in an
evolution. Each \textsc{EinsteinExact} thorn extends the
\code{ADMBase::evolution\_method}
keyword parameter with another option corresponding to the name of the thorn.
If the parameter is set to the thorn name, then the ADM variables will be set
by the thorn at each iteration during the \code{ANALYSIS} schedule bin.

\subsubsection{\code{ADMBase::initial\_lapse}}
Each EinsteinExact thorn extends this keyword parameter with another option
corresponding to the name of the thorn. This parameter is currently ignored
by the EinsteinExact code but must be set nonetheless in order to ensure
storage is allocated for the lapse variable.

\subsubsection{\code{ADMBase::initial\_shift}}
Each EinsteinExact thorn extends this keyword parameter with another option
corresponding to the name of the thorn. This parameter is currently ignored
by the EinsteinExact code but must be set nonetheless in order to ensure
storage is allocated for the shift variables.

\subsubsection{\code{ADMBase::initial\_dtlapse}}
Each EinsteinExact thorn extends this keyword parameter with another option
corresponding to the name of the thorn. This parameter is currently ignored
by the EinsteinExact code but must be set nonetheless in order to ensure
storage is allocated for the dtlapse variable.

\subsubsection{\code{ADMBase::initial\_dtshift}}
Each EinsteinExact thorn extends this keyword parameter with another option
corresponding to the name of the thorn. This parameter is currently ignored
by the EinsteinExact code but must be set nonetheless in order to ensure
storage is allocated for the dtshift variables.

\subsubsection{Origin: \code{timeoffset}, \code{positionx}, \code{positiony}, \code{positionz}}
These parameters specify the position at which the origin of the spacetime
should be located.

\subsubsection{Lapse: \code{lapsefactor}}
Factor that should rescale the lapse such that
$\alpha' = \tfrac{1}{\code{lapsefactor}}~ \alpha$. This has the effect of rescaling the time
coordinate so that $t' = \code{lapsefactor}~ t$. Note that this also has the
effect of rescaling the shift by a factor $\tfrac{1}{\code{lapsefactor}}$ (see xform.nb
for more details).

\subsubsection{Shift: \code{shiftaddx}, \code{shiftaddy}, \code{shiftaddz}}
Shift that should be added to the metric. This has the effect of adding an
offset to the spatial coordinates so that $x' = x - \code{shiftaddx}~ t'$,
$y' = y - \code{shiftaddy}~ t'$,  $z' = z - \code{shiftaddz}~ t'$ (see xform.nb
for more details).

\subsubsection{Lorentz Boost: \code{boostx}, \code{boosty}, \code{boostz}}
Lorentz-boost the spacetime by a velocity
$v^i = [\code{boostx}, \code{boosty}, \code{boostz}]$. This applies a
Lorentz boost transformation to the coordinates of the form
\begin{equation}
\begin{bmatrix}
t' \\
\mathbf{r'}
\end{bmatrix}
=
\begin{bmatrix}
\gamma & - \gamma \boldsymbol{\beta}^{\mathrm{T}} \\
-\gamma\boldsymbol{\beta} & \mathbf{I} + \gamma^2 / (\gamma+1) \boldsymbol{\beta}\boldsymbol{\beta}^{\mathrm{T}}  \\
\end{bmatrix}
\begin{bmatrix}
t  \\
\mathbf{r}
\end{bmatrix}.
\end{equation}


\subsubsection{Rotation: \code{theta}, \code{phi}, \code{psi}}
The spacetime may have an arbitrary $3$D rigid rotation applied to it. This
rotation is specified by the \code{theta}, \code{phi} and \code{psi} parameters
which correspond to the Euler angles of the rotation. The conventions used are
such that the rotations are given by the three rotation matrices \cite{EinsteinExact_Euler}:
\begin{align}
R_\phi =& 
\left(
\begin{array}{ccc}
 \cos (\phi ) & \sin (\phi ) & 0 \\
 -\sin (\phi ) & \cos (\phi ) & 0 \\
 0 & 0 & 1 \\
\end{array}
\right), \nonumber \\
R_\theta =&
\left(
\begin{array}{ccc}
 1 & 0 & 0 \\
 0 & \cos (\theta ) & \sin (\theta ) \\
 0 & -\sin (\theta ) & \cos (\theta ) \\
\end{array}
\right), \nonumber \\
R_\psi =&
\left(
\begin{array}{ccc}
 \cos (\psi ) & \sin (\psi ) & 0 \\
 -\sin (\psi ) & \cos (\psi ) & 0 \\
 0 & 0 & 1 \\
\end{array}
\right).
\end{align}

\section{Included spacetimes}

This section lists all spacetimes currently supported by \textsc{EinsteinExact}.


\subsection{GaugeWave}
{\bf Description:}  \text{4D flat spacetime in Cartesian gauge wave coordinates} \\
{\bf Thorn name:} GaugeWave \\
{\bf Parameters:} \code{\text{amp}}, \code{\text{period}} \\
{\bf Metric:} \\
\begin{equation}
g_{ab} =
\left(
\begin{array}{cccc}
 \text{amp} \sin \left(\frac{2 \pi  (x-t)}{\text{period}}\right)-1 & 0 & 0 & 0 \\
 0 & 1-\text{amp} \sin \left(\frac{2 \pi  (x-t)}{\text{period}}\right) & 0 & 0 \\
 0 & 0 & 1 & 0 \\
 0 & 0 & 0 & 1 \\
\end{array}
\right)
\end{equation}

\subsection{KerrSchild}
{\bf Description:}  \text{Kerr spacetime in Kerr-Schild coordinates} \\
{\bf Thorn name:} KerrSchild \\
{\bf Parameters:} \code{M}, \code{a} \\
{\bf Metric:} \\
\begin{equation}
g_{ab} =
\left(
\begin{array}{cccc}
 \frac{2 M r^3}{a^2 z^2+r^4}-1 & \frac{2 M r^3 (a y+r x)}{\left(a^2+r^2\right) \left(a^2 z^2+r^4\right)} & \frac{2 M r^3 (r y-a x)}{\left(a^2+r^2\right) \left(a^2 z^2+r^4\right)} & \frac{2 M r^2 z}{a^2 z^2+r^4} \\
 \frac{2 M r^3 (a y+r x)}{\left(a^2+r^2\right) \left(a^2 z^2+r^4\right)} & \frac{2 M r^3 (a y+r x)^2}{\left(a^2+r^2\right)^2 \left(a^2 z^2+r^4\right)}+1 & \frac{2 M r^3 (a y+r x) (r y-a x)}{\left(a^2+r^2\right)^2 \left(a^2 z^2+r^4\right)} & \frac{2 M r^2 z (a y+r x)}{\left(a^2+r^2\right) \left(a^2 z^2+r^4\right)} \\
 \frac{2 M r^3 (r y-a x)}{\left(a^2+r^2\right) \left(a^2 z^2+r^4\right)} & \frac{2 M r^3 (a y+r x) (r y-a x)}{\left(a^2+r^2\right)^2 \left(a^2 z^2+r^4\right)} & \frac{2 M r^3 (r y-a x)^2}{\left(a^2+r^2\right)^2 \left(a^2 z^2+r^4\right)}+1 & \frac{2 M r^2 z (r y-a x)}{\left(a^2+r^2\right) \left(a^2 z^2+r^4\right)} \\
 \frac{2 M r^2 z}{a^2 z^2+r^4} & \frac{2 M r^2 z (a y+r x)}{\left(a^2+r^2\right) \left(a^2 z^2+r^4\right)} & \frac{2 M r^2 z (r y-a x)}{\left(a^2+r^2\right) \left(a^2 z^2+r^4\right)} & \frac{2 M r z^2}{a^2 z^2+r^4}+1 \\
\end{array}
\right)
\end{equation}
where
\begin{equation}
r(x,y,z)=\frac{\sqrt{\sqrt{\left(-a^2+x^2+y^2+z^2\right)^2+4 a^2 z^2}-a^2+x^2+y^2+z^2}}{\sqrt{2}}
\end{equation}

\subsection{Minkowski}
{\bf Description:}  \text{4D flat spacetime in Cartesian coordinates} \\
{\bf Thorn name:} Minkowski \\
{\bf Metric:} \\
\begin{equation}
g_{ab} =
\left(
\begin{array}{cccc}
 -1 & 0 & 0 & 0 \\
 0 & 1 & 0 & 0 \\
 0 & 0 & 1 & 0 \\
 0 & 0 & 0 & 1 \\
\end{array}
\right)
\end{equation}

\subsection{ShiftedGaugeWave}
{\bf Description:}  \text{4D flat spacetime in Cartesian shifted gauge wave coordinates} \\
{\bf Thorn name:} ShiftedGaugeWave \\
{\bf Parameters:} \code{\text{amp}}, \code{\text{period}} \\
{\bf Metric:} \\
\begin{equation}
g_{ab} =
\left(
\begin{array}{cccc}
 \text{amp} \sin \left(\frac{2 \pi  (x-t)}{\text{period}}\right)-1 & -\text{amp} \sin \left(\frac{2 \pi  (x-t)}{\text{period}}\right) & 0 & 0 \\
 -\text{amp} \sin \left(\frac{2 \pi  (x-t)}{\text{period}}\right) & \text{amp} \sin \left(\frac{2 \pi  (x-t)}{\text{period}}\right)+1 & 0 & 0 \\
 0 & 0 & 1 & 0 \\
 0 & 0 & 0 & 1 \\
\end{array}
\right)
\end{equation}

\subsection{Vaidya}
{\bf Description:}  \text{A Vaidya spacetime with mass function m(v) = M (1 + Tanh[v dM / M]${}^{\wedge}$2)} \\
{\bf Thorn name:} Vaidya2 \\
{\bf Parameters:} \code{M}, \code{\text{dM}} \\
{\bf Metric:} \\
\begin{equation}
g_{ab} =
\left(
\begin{array}{cccc}
 \frac{2 m}{r}-1 & \frac{2 m x}{r^2} & \frac{2 m y}{r^2} & \frac{2 m z}{r^2} \\
 \frac{2 m x}{r^2} & \frac{2 m x^2}{r^3}+1 & \frac{2 m x y}{r^3} & \frac{2 m x z}{r^3} \\
 \frac{2 m y}{r^2} & \frac{2 m x y}{r^3} & \frac{2 m y^2}{r^3}+1 & \frac{2 m y z}{r^3} \\
 \frac{2 m z}{r^2} & \frac{2 m x z}{r^3} & \frac{2 m y z}{r^3} & \frac{2 m z^2}{r^3}+1 \\
\end{array}
\right)
\end{equation}
where
\begin{equation}
r(x,y,z)=\sqrt{x^2+y^2+z^2}
\end{equation}
\begin{equation}
m(t,x,y,z)=M \left(\tanh ^2\left(\frac{\text{dM} \left(t+\sqrt{x^2+y^2+z^2}\right)}{M}\right)+1\right)
\end{equation}

\subsection{ModifiedSchwarzschildBL}
{\bf Description:}  \text{Modified Schwarzschild spacetime in Brill-Lindquist coordinates with a radially dependent mass inside the black hole} \\
{\bf Thorn name:} ModifiedSchwarzschildBL \\
{\bf Parameters:} \code{M}, \code{epsilon} \\
{\bf Metric:} \\
\begin{equation}
g_{ab} =
\left(
\begin{array}{cccc}
 -1 & 0 & 0 & 0 \\
 0 & \left(1+\frac{M(r)}{2 r}\right)^4 & 0 & 0 \\
 0 & 0 & \left(1+\frac{M(r)}{2 r}\right)^4 & 0 \\
 0 & 0 & 0 & \left(1+\frac{M(r)}{2 r}\right)^4 \\
\end{array}
\right)
\end{equation}
where
\begin{equation}
r(x,y,z)=\sqrt{x^2+y^2+z^2}
\end{equation}
\begin{equation}
M(r) = \left\{
     \begin{array}{lr}
       4 r - \frac{8 \frac{r^2}{2} + [\cos(\frac{4 \pi r}{M}) - 1] \left(\frac{M}{4 \pi}\right)^2}{M} & : r < M/2\\
       M & : r \ge M/2
     \end{array}
   \right.
\end{equation}

\section{Modifying EinsteinExact}

\subsection{Generating the code}
The script which generates the thorns requires the \textsc{Kranc} and
\textsc{Metrics} \textsc{Mathematica} packages to be installed. \textsc{Kranc}
is included with the Einstein Toolkit and the \textsc{Metrics} package is
included in the \code{Metrics} directory.

To regenerate the code \code{make} from within the \code{m} directory.

\subsection{Adding a spacetime}
To add a new spacetime to \textsc{EinsteinExact}, simply create a new file
for the spacetime under \code{m/Metrics/metrics/} and then add this spacetime
to the \code{thorns = \ldots} and \code{spacetimes = \ldots} lines at the end
of the \textsc{Kranc} script in \code{m/EinsteinExact.m}. Also consider adding
a regression test to the \code{EinsteinExact\_test} thorn and correctness tests
under the \code{tests} directory.

If you think this spacetime would be of use to others, please consider 
submitting a patch. As all of the spacetimes in the \textsc{EinsteinExact}
arrangement come from them \textsc{Metrics} database it is highly preferable to
have new spacetimes added to the upstream repository for the \textsc{Metrics}
package. The version of the \textsc{Metrics} package used by
\textsc{EinsteinExact} can then be updated.

\subsection{Updating the Metrics database}
The \textsc{Metrics} database exists as a separate git repository. For
convenience, a version of it is included with \textsc{EinsteinExact} under
the \code{m/Metrics} directory. This is a version of the standalone
\textsc{Metrics} repository which has been merged using the \code{git-subtree}
command\footnote{\code{git-subtree} is included in the latest version of git.
For older versions of git, it may be obtained from
\url{http://github.com/apenwarr/git-subtree}}.

If a new version of the \textsc{Metrics} package becomes available, the version
in the \textsc{EinsteinExact} repository may be updated using the following
command:
\begin{verbatim}
git subtree pull --prefix=m/Metrics --squash git://github.com/barrywardell/Metrics master
\end{verbatim}
This will pull in new changes from the \textsc{Metrics} repository and merge
them with the \textsc{EinsteinExact} repository under the \code{m/Metrics}
directory.

\section{Tests}

\subsection{Correctness tests}
There are correctness tests for the \textsc{EinsteinExact} thorns under the
\code{tests} directory. These tests include:
\begin{itemize}
\item Checks that the solutions provided by the \textsc{Exact} thorn converge
  to those provided by each of the \textsc{EinsteinExact} thorns at the appropriate order.
\item Checks that the constraints for the solutions provided by each of the
 \textsc{EinsteinExact} converge to $0$ at the appropriate order.
\end{itemize}
In addition, the \textsc{Metrics} package has independent correctness tests
which check that the provided metrics are correct.

\subsection{Regression tests}
Each thorn provided by \textsc{EinsteinExact} has a corresponding regression
test in the \code{EinsteinExact\_code} thorn which checks that no new bugs are
introduced in the code.

\section{Acknowledgments}
Erik Schnetter contributed helpful comments and suggestions, along with some
patches for the code.

\begin{thebibliography}{{Pen}}

\bibitem{EinsteinExact_Euler}
{Eric W. Weisstein, {\em Euler Angles}, MathWorld--A Wolfram Web Resource. \url{http://mathworld.wolfram.com/EulerAngles.html}}

\end{thebibliography}

% END CACTUS THORNGUIDE

\end{document}
